%% contents/abstract.tex
%% Copyright 2021-2022 Tom M. Ragonneau
%
% This work may be distributed and/or modified under the
% conditions of the LaTeX Project Public License, either version 1.3
% of this license or (at your option) any later version.
% The latest version of this license is in
%   http://www.latex-project.org/lppl.txt
% and version 1.3 or later is part of all distributions of LaTeX
% version 2005/12/01 or later.
%
% This work has the LPPL maintenance status `maintained'.
%
% The Current Maintainer of this work is Tom M. Ragonneau.
\chapter*{Abstract}
\label{ch:abstract}
\addcontentsline{toc}{chapter}{\nameref*{ch:abstract}}
\markboth{\nameref*{ch:abstract}}{\nameref*{ch:abstract}}

In many applications, it is frequent to address optimization problems for which the objective function or the constraints require experiments or simulations.
The first-order information of these functions is often not assessable; hence, we need algorithms that use only function values, namely \glsxtrfull{dfo} algorithms.
This thesis focuses on a particular class of \glsxtrshort{dfo} methods, referred to as model-based methods.

Prof.\ M. J. D. Powell developed five \glsxtrshort{dfo} algorithms that are in the public domain: \gls{cobyqa}, \gls{uobyqa}, \gls{newuoa}, \gls{bobyqa}, and \gls{lincoa}.
He implemented these methods in Fortran 77; therefore, they may not be easily accessible to some users.
After first presenting some theory of optimization and polynomial interpolation, we introduce in this thesis the \gls{pdfo} package, developed to provide both Python and MATLAB interfaces to Powell's code, in joint work with Dr.\ Zaikun Zhang.
With \gls{pdfo}, users can call each Powell's algorithm directly based on their choice.
Alternatively, if the user does not specify any algorithm, the package can also select one automatically based on the problem characteristics.
When multiple algorithms can solve a problem, the package selects one based on experimental results obtained on the CUTEst problems.
We also share some observations about the behavior of Powell's algorithms on these problems.
Moreover, Powell's three most recent methods share similar default initial interpolation sets.
We show that, to some extent, this interpolation set is optimal.

Later in this thesis, we design a \glsxtrshort{dfo} algorithm for nonlinearly-constrained optimization problems.
The algorithm, named \gls{cobyqa}, is a derivative-free trust-region \glsxtrfull{sqp} method designed to tackle nonlinearly-constrained optimization problems that include equality and inequality constraints.
The method accepts bound constraints and always respects them.
Moreover, it builds the trust-region quadratic models using Powell's derivative-free symmetric Broyden updates.
We first discuss some theoretical properties of the traditional \glsxtrshort{sqp} method underlying \gls{cobyqa}.
In particular, we will show an intriguing interpretation of the \glsxtrshort{sqp} method.
Indeed, the objective function of the \glsxtrshort{sqp} subproblem turns out to be a quadratic approximation of the original objective function in the tangent space of a surface.
To solve the trust-region \glsxtrshort{sqp} subproblem, we employ a Byrd-Omojokun approach that is different from the traditional one, and numerical experiments show that our new approach significantly improves the performance of \gls{cobyqa}.
Finally, we introduce the Python implementation of \gls{cobyqa}, made publicly available.
Such an implementation is highly nontrivial due to the nature of \glsxtrshort{dfo} algorithms.
