%% contents/abstract.tex
%% Copyright 2021-2022 Tom M. Ragonneau
%
% This work may be distributed and/or modified under the
% conditions of the LaTeX Project Public License, either version 1.3
% of this license or (at your option) any later version.
% The latest version of this license is in
%   http://www.latex-project.org/lppl.txt
% and version 1.3 or later is part of all distributions of LaTeX
% version 2005/12/01 or later.
%
% This work has the LPPL maintenance status `maintained'.
%
% The Current Maintainer of this work is Tom M. Ragonneau.
\chapter*{Abstract}
\label{ch:abstract}
\addcontentsline{toc}{chapter}{\nameref*{ch:abstract}}
\markboth{\nameref*{ch:abstract}}{\nameref*{ch:abstract}}

This thesis studies \glsxtrfull{dfo}, particularly model-based methods and software.
These methods are motivated by optimization problems for which it is impossible or prohibitively expensive to access the first-order information of the objective function and possibly the constraint functions.
Such problems frequently arise in engineering and industrial applications, and keep emerging due to recent advances in data science and machine learning.

We first provide an overview of \glsxtrshort{dfo} and of interpolation models for \glsxtrshort{dfo} methods.
In particular, we show that an interpolation set employed by Powell for underdetermined quadratic interpolation is optimal in terms of well-poisedness.

We then present \gls{pdfo}, a package that we develop to provide both MATLAB and Python interfaces to Powell's model-based \glsxtrshort{dfo} solvers, namely \gls{cobyla}, \gls{uobyqa}, \gls{newuoa}, \gls{bobyqa}, and \gls{lincoa}.
They were implemented by Powell in Fortran 77, and hence, are becoming inaccessible to many users nowadays.
\Gls{pdfo} provides user-friendly interfaces to these solvers, so that users do not need to deal with the Fortran code.
In addition, it patches bugs in the original Fortran implementation.
We also share some observations about the behavior of Powell's solvers.

A major part of this thesis is devoted to the development of a new \glsxtrshort{dfo} method based on the \glsxtrfull{sqp} method.
Therefore, we first present an overview of the \glsxtrshort{sqp} method and provide some perspectives on its theory and practice.
In particular, we show that the objective function of the \glsxtrshort{sqp} subproblem is a natural quadratic approximation of the original objective function in the tangent space of a surface.
Moreover, we propose an extension of the Byrd-Omojokun approach for solving trust-region \glsxtrshort{sqp} subproblems with inequality constraints.
This extension works quite well in our experiments.

Finally, we elaborate on the development of our new \glsxtrshort{dfo} method, named \gls{cobyqa} after \textit{\glsdesc{cobyqa}}.
This is a derivative-free trust-region \glsxtrshort{sqp} method designed to tackle nonlinearly constrained optimization problems that admit equality and inequality constraints.
An important feature of \gls{cobyqa} is that it always respects bound constraints, if any, which is motivated by applications where the objective function is undefined when bounds are violated.
\Gls{cobyqa} builds quadratic trust-region models based on the derivative-free symmetric Broyden update proposed by Powell.
We provide a detailed description of \gls{cobyqa}, including its subproblem solvers, and introduce its Python implementation.
Finally, we expose extensive numerical experiments of \gls{cobyqa}, showing evident advantages of \gls{cobyqa} compared with Powell's \glsxtrshort{dfo} solvers.
These experiments demonstrate that \gls{cobyqa} is an excellent successor to \gls{cobyla} as a general-purpose \glsxtrshort{dfo} solver.
