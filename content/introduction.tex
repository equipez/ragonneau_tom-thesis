%% contents/introduction.tex
%% Copyright 2021-2022 Tom M. Ragonneau
%
% This work may be distributed and/or modified under the
% conditions of the LaTeX Project Public License, either version 1.3
% of this license or (at your option) any later version.
% The latest version of this license is in
%   http://www.latex-project.org/lppl.txt
% and version 1.3 or later is part of all distributions of LaTeX
% version 2005/12/01 or later.
%
% This work has the LPPL maintenance status `maintained'.
%
% The Current Maintainer of this work is Tom M. Ragonneau.
\chapter{Introduction}

\todo[noline]{Place the nomenclature items right after their first use.}
\nomenclature[Fa]{$\obj$}{Real-valued objective function defined on~$\R^n$}%
\nomenclature[Fb]{$\con{i}$}{Real-valued constraint function defined on~$\R^n$, with~$i \in \iub \cup \ieq$}%
\nomenclature[Fc]{$\lag$}{Lagrangian function}%
\nomenclature[Na]{$\in$}{Set membership notation}%
\nomenclature[Nb]{$\subseteq$}{Set inclusion notation}%
\nomenclature[Nc]{$\qedsymbol$}{Halmos symbol}%
\nomenclature[Nd]{$A^{\mathsf{T}}$,~$v^{\mathsf{T}}$}{Transpose of a matrix or a vector}%
\nomenclature[Ne]{$\mathcal{O}(\cdot)$}{Big-O notation}%
\nomenclature[Nf]{$o(\cdot)$}{Little-O notation}%
\nomenclature[Oa]{$[\cdot]_{+}$}{Elementwise positive-part operator}%
\nomenclature[Ob]{$[\cdot]_{-}$}{Elementwise negative-part operator}%
\nomenclature[Oc]{$\abs{\cdot}$}{Elementwise modulus operator}%
\nomenclature[Od]{$\inner{\cdot, \cdot}$}{Inner-product operator (may be subscripted for sake of clarity)}%
\nomenclature[Oe]{$\norm{\cdot}$}{Norm of a vector or a matrix (may be subscripted for sake of clarity)}%
\nomenclature[Of]{$\nabla$}{Gradient operator (elements~$\partial / \partial x_i$, with~$i \in \set{1, 2, \dots, n}$)}%
\nomenclature[Og]{$\nabla^2$}{Hessian operator (elements~$\partial^2 / \partial x_i \partial x_j$, with~$i, j \in \set{1, 2, \dots, n}$)}%
\nomenclature[Sa]{$\emptyset$}{Empty set}%
\nomenclature[Sb]{$[a, b]$}{Closed set~$\set{x \in \R : a \le x \le b}$ with~$a \le b$}%
\nomenclature[Sc]{$(a, b)$}{Open set~$\set{x \in \R : a < x < b}$ with~$a < b$}%
\nomenclature[Sd]{$[a, b)$}{Semi-open set~$\set{x \in \R : a \le x < b}$ with~$a < b$}%
\nomenclature[Se]{$(a, b]$}{Semi-open set~$\set{x \in \R : a < x \le b}$ with~$a < b$}%
\nomenclature[Sf]{$\R$}{Set of real numbers}%
\nomenclature[Sg]{$\R^n$}{Real coordinate space of dimension~$n$}%
\nomenclature[Sh]{$\Omega$}{Feasible set, included in~$\R^n$}%
\nomenclature[Si]{$\ieq$}{Set of indices of the equality constraints}%
\nomenclature[Sj]{$\iub$}{Set of indices of the inequality constraints}%

\section{Overview of \glsfmtlong{dfo}}

Optimization is the study of extremal points and values of mathematical functions.
Traditional optimization aims at minimizing (or maximizing) a real-valued function~$f$, referred to as the \emph{objective function}, within a given nonempty set of points~$\Omega \subseteq \R^n$, referred to as the \emph{feasible set}.
It is well known that the most useful information to optimization is embraced in the derivatives of~$f$ and of the curves of~$\Omega$.
However, such derivative information may not exist, be unassessable, be unreliable, or be prohibitively expensive to evaluate.
It is within this context, referred to as \gls{dfo}, that our research focuses.
We emphasize that we are \emph{not} studying nonsmooth optimization.
Classical or generalized derivatives of the functions defining the optimization problem may be well-defined, but we assume that they cannot be numerically assessed.

\begin{itemize}
    \item The leading complexity measure is the number of objective function evaluations.
\end{itemize}

\section{Examples of \glsfmtlong{dfo} applications}

\begin{itemize}
    \item Automatic error analysis~\cite{Higham_1993,Higham_2002}.
    \item Parameter tuning of nonlinear optimization methods~\cite{Audet_Orban_2006}.
    \item Machine learning~\cite{Ghanbari_Scheinberg_2017,Qian_Yu_2021}.
    \item Helicopter rotor blade design~\cite{Booker_Etal_1998a,Booker_Etal_1998b,Serafini_1998}.
    \item Molecular conformational analysis~\cite{Alberto_Etal_2004,Meza_Martinez_1994}.
    \item Groundwater supply and bioremediation engineering~\cite{Fowler_Etal_2008,Mugunthan_Shoemaker_Regis_2005,Yoon_Shoemaker_1999}.
    \item Aeroacoustic shape design~\cite{Marsden_2004,Marsden_Etal_2004}.
    \item Hydrodynamic design~\cite{Duvigneau_Visonneau_2004}.
    \item Registration in medical imaging~\cite{Oeuvray_2005,Oeuvray_Bierlaire_2007}.
    \item Dynamic pricing~\cite{Levina_Etal_2009}.
    \item Reservoir engineering and engine calibration~\cite{Langouet_2011}.
    \item Analog circuit design~\cite{Latorre_Etal_2019}.
    \item Aircraft engineering~\cite{Gazaix_Etal_2019}.
    \item Chemical product design~\cite{Sun_Etal_2020}.
\end{itemize}

\section{Optimality conditions for smooth optimization}

We consider the nonlinearly-constrained optimization problem
\begin{subequations}
    \label{eq:nlcp}
    \begin{align}
        \min        & \quad \obj(x) \label{eq:nlcp-obj}\\
        \text{s.t.} & \quad \con{i}(x) \le 0, ~ i \in \iub, \label{eq:nlcp-cub}\\
                    & \quad \con{i}(x) = 0, ~ i \in \ieq, \label{eq:nlcp-ceq}\\
                    & \quad x \in \R^n, \nonumber
    \end{align}
\end{subequations}
where the \emph{objective} and \emph{constraint functions}~$\obj$ and~$\con{i}$, with~$i \in \iub \cup \ieq$, are real-valued functions on~$\R^n$.
We define for sake of clarity the \emph{feasible set}~$\Omega \subseteq \R^n$ to be the set of points satisfying the constraints~\cref{eq:nlcp-cub,eq:nlcp-ceq}, that is
\begin{equation*}
    \Omega \eqdef \set{x \in \R^n : \con{i}(x) \le 0, ~ i \in \iub, ~ \con{i}(x) = 0, ~ i \in \ieq}.
\end{equation*}

\subsection{Local and global solutions}

\begin{definition}[Global solution]
    A point~$x^{\ast} \in \R^n$ is referred to as a \emph{global solution} to problem~\cref{eq:nlcp} if~$x^{\ast} \in \Omega$ and~$f(x) \ge f(x^{\ast})$ for all~$x \in \Omega$.
\end{definition}

\begin{definition}[Local solution]
    A points~$x^{\ast} \in \R^n$ is referred to as
    \begin{itemize}
        \item a \emph{local solution} to problem~\cref{eq:nlcp} if~$x^{\ast} \in \Omega$ and there exists an open neighborhood~$\mathcal{N} \subseteq \R^n$ of~$x^{\ast}$ such that~$f(x) \ge f(x^{\ast})$ for all~$x \in \mathcal{N} \cap \Omega$.
        \item a \emph{strict local solution} to problem~\cref{eq:nlcp} if~$x^{\ast} \in \Omega$ and there exists an open neighborhood~$\mathcal{N} \subseteq \R^n$ of~$x^{\ast}$ such that~$f(x) > f(x^{\ast})$ for all~$x \in \mathcal{N} \cap \Omega \setminus \set{x^{\ast}}$.
        \item an \emph{isolated local solution} to problem~\cref{eq:nlcp} if there exists an open neighborhood~$\mathcal{N} \subseteq \R^n$ of~$x^{\ast}$ such that it is the only local solution in~$\mathcal{N} \cap \Omega$.
    \end{itemize}
\end{definition}

\subsection{Constraint qualifications}

\begin{definition}[Active set]
    The \emph{active set}~$\mathcal{A}(x) \subseteq \iub \cup \ieq$ for problem~\cref{eq:nlcp} at a point~$x \in \Omega$ is defined by
    \begin{equation*}
        \mathcal{A}(x) \eqdef \ieq \cup \set{i \in \iub : \con{i}(x) \ge 0}.
    \end{equation*}
\end{definition}

\begin{definition}[Constraint qualification]
    Let~$x \in \Omega$ be any feasible point, denote~$\mathcal{A}(x)$ the active set for problem~\cref{eq:nlcp} at~$x$, and assume that the constraints function~$\con{i}$ are differentiable at~$x$ for all~$i \in \mathcal{A}(x)$.
    We say that
    \begin{itemize}
        \item the \gls{licq} holds at~$x$ if the gradients~$\nabla \con{i}(x)$ are linearly independent for all~$i \in \mathcal{A}(x)$, and
        \item the \gls{mfcq} holds at~$x$ if the gradients~$\nabla \con{i}(x)$ are linearly independent for all~$i \in \ieq$ and there exists a vector~$d \in \R^n$ such that
        \begin{equation*}
            \begin{cases}
                \inner{\nabla \con{i}(x), d} < 0    & \text{if~$i \in \mathcal{A}(x) \cap \iub$, and}\\
                \inner{\nabla \con{i}(x), d} = 0    & \text{if~$i \in \ieq$}.
            \end{cases}
        \end{equation*}
    \end{itemize}
\end{definition}

\begin{itemize}
    \item \gls{acq}.
    \item \gls{gcq}.
    \item \gls{lcq}.
    \item \gls{crcq}.
    \item \gls{cpld}.
    \item \gls{qncq}.
    \item \gls{sc}.
\end{itemize}

\subsection{First-order optimality conditions}

\subsection{Second-order optimality conditions}

\section{Methodology of \glsfmtlong{dfo} algorithms}

\subsection{Frameworks and algorithms for \glsfmtlong{dfo}}

\begin{itemize}
    \item Direct-search and model-based methods.
    \item Line-search and trust-region methods.
    \item Filtering methods and composite methods.
    \item Geometry of the interpolation sets.
\end{itemize}

\subsection{Prominent \glsfmtlong{dfo} methods}

\section{Performance profiles-distribution functions}
