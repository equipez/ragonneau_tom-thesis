%% contents/introduction.tex
%% Copyright 2021-2022 Tom M. Ragonneau
%
% This work may be distributed and/or modified under the
% conditions of the LaTeX Project Public License, either version 1.3
% of this license or (at your option) any later version.
% The latest version of this license is in
%   http://www.latex-project.org/lppl.txt
% and version 1.3 or later is part of all distributions of LaTeX
% version 2005/12/01 or later.
%
% This work has the LPPL maintenance status `maintained'.
%
% The Current Maintainer of this work is Tom M. Ragonneau.
\chapter{Introduction}

\nomenclature[Fa]{$\obj$}{Real-valued objective function defined on~$\R^n$}%
\nomenclature[Fb]{$\con{i}$}{Real-valued constraint function defined on~$\R^n$, with~$i \in \iub \cup \ieq$}%
\nomenclature[Fc]{$\lag$}{Lagrangian function}%
\nomenclature[Na]{$\in$}{Set membership notation}%
\nomenclature[Nb]{$\subseteq$}{Set inclusion notation}%
\nomenclature[Nc]{$\qedsymbol$}{Halmos symbol}%
\nomenclature[Nd]{$A^{\T}$,~$v^{\T}$}{Transpose of a matrix or a vector}%
\nomenclature[Ne]{$I_n$}{Identity matrix on~$\R^{n \times n}$}%
\nomenclature[Nf]{$e_k$}{Standard coordinate vector of~$\R^n$ ($k$-th column of~$I_n$), with~$1 \le k \le n$}%
\nomenclature[Ng]{$\mathcal{O}(\cdot)$}{Big-O notation}%
\nomenclature[Nh]{$o(\cdot)$}{Little-O notation}%
\nomenclature[Oa]{$[\cdot]_{+}$}{Elementwise positive-part operator}%
\nomenclature[Ob]{$[\cdot]_{-}$}{Elementwise negative-part operator}%
\nomenclature[Oc]{$\abs{\cdot}$}{Elementwise modulus operator}%
\nomenclature[Od]{$\inner{\cdot, \cdot}$}{Inner-product operator (may be subscripted for sake of clarity)}%
\nomenclature[Oe]{$\norm{\cdot}$}{Norm of a vector or a matrix (may be subscripted for sake of clarity)}%
\nomenclature[Of]{$\nabla$}{Gradient operator (elements~$\partial / \partial x_i$, with~$i \in \set{1, 2, \dots, n}$)}%
\nomenclature[Og]{$\nabla^2$}{Hessian operator (elements~$\partial^2 / \partial x_i \partial x_j$, with~$i, j \in \set{1, 2, \dots, n}$)}%
\nomenclature[Oh]{$\land$}{Logic and operator}%
\nomenclature[Sa]{$\emptyset$}{Empty set}%
\nomenclature[Sb]{$[a, b]$}{Closed set~$\set{x \in \R : a \le x \le b}$ with~$a \le b$}%
\nomenclature[Sc]{$(a, b)$}{Open set~$\set{x \in \R : a < x < b}$ with~$a < b$}%
\nomenclature[Sd]{$[a, b)$}{Semi-open set~$\set{x \in \R : a \le x < b}$ with~$a < b$}%
\nomenclature[Se]{$(a, b]$}{Semi-open set~$\set{x \in \R : a < x \le b}$ with~$a < b$}%
\nomenclature[Sf]{$\R$}{Set of real numbers}%
\nomenclature[Sg]{$\R^n$}{Real coordinate space of dimension~$n$}%
\nomenclature[Sh]{$\R^{m \times n}$}{Real matrix space of dimension~$m \times n$}%
\nomenclature[Si]{$\Omega$}{Feasible set, included in~$\R^n$}%
\nomenclature[Sj]{$\ieq$}{Set of indices of the equality constraints}%
\nomenclature[Sk]{$\iub$}{Set of indices of the inequality constraints}%
\todo[noline]{Replace the nomenclature items.}

\section{Overview of \glsfmtlong{dfo}}

Optimization is the study of extremal points and values of mathematical functions.
Traditional optimization aims at minimizing (or maximizing) a real-valued function~$\obj$, referred to as the \emph{objective function}, within a given nonempty set of points~$\Omega \subseteq \R^n$, referred to as the \emph{feasible set}.
It is well known that the most helpful information to optimization is embraced in the derivatives of~$\obj$ and the curves of~$\Omega$.
However, such derivative information may not exist, be unassessable, unreliable, or prohibitively expensive to evaluate.
It is within this context, referred to as \gls{dfo}~\cite{Audet_Hare_2017,Conn_Scheinberg_Vicente_2009b}, that our research focuses.
Problems for which derivative information is unavailable arise naturally when the objective function or the feasible set results from complex experiments or simulations.
We emphasize that we are \emph{not} studying nonsmooth optimization.
Classical or generalized derivatives of the optimization problem's functions may be well-defined, but we assume they cannot be numerically assessed.

The leading complexity measure we consider is the number of objective function evaluations.
We presume that an objective function evaluation dawdles and requires several minutes or even several days to complete.
For instance, a modern application of \gls{dfo} is hyperparameter tuning in machine learning~\cite{Ghanbari_Scheinberg_2017}, for which one objective function evaluation necessitates training a machine learning model (see~\cref{subsec:machine-learning}).
Hence, the algebraic complexity of the methods we consider is not our leading issue, although we will strive to maintain it acceptable.

Within this chapter, we consider the nonlinearly-constrained problem
\begin{subequations}
    \label{eq:nlcp-intro}
    \begin{align}
        \min        & \quad \obj(x)\\
        \text{s.t.} & \quad \con{i}(x) \le 0, ~ i \in \iub, \label{eq:nlcp-intro-cub}\\
                    & \quad \con{i}(x) = 0, ~ i \in \ieq, \label{eq:nlcp-intro-ceq}\\
                    & \quad x \in \R^n, \nonumber
    \end{align}
\end{subequations}
where the \emph{objective} and \emph{constraint functions}~$\obj$ and~$\con{i}$, with~$i \in \iub \cup \ieq$, are real-valued functions on~$\R^n$, and where the sets of indices~$\iub$ and~$\ieq$ are finite (perhaps empty) and disjoint.
The above-mentioned feasible set~$\Omega \subseteq \R^n$ is, therefore, the set of points satisfying the inequality constraints~\cref{eq:nlcp-intro-cub} and the equality constraints~\cref{eq:nlcp-intro-ceq}, that is
\begin{equation*}
    \Omega \eqdef \set{x \in \R^n : \con{i}(x) \le 0, ~ i \in \iub \land \con{i}(x) = 0, ~ i \in \ieq}.
\end{equation*}

\section{Examples of applications}

\subsection{Automatic error analysis}

A typical example of \gls{dfo} applications is automatic error analysis~\cite{Higham_1993,Higham_2002}, which formulates numerical computation's accuracies and stabilities using optimization problems.
Consider, for instance, the Gaussian elimination with partial pivoting of a matrix~$A \in \R^{n \times n}$, given in~\cref{alg:gepp}.

\begin{algorithm}[htp]
    \caption{Gaussian elimination with partial pivoting}
    \label{alg:gepp}
    \DontPrintSemicolon
    \KwData{Matrix~$A \in \R^{n \times n}$.}
    Initialize $A^{(0)} \gets A$\;
    \For{$k = 1, 2, \dots, n - 1$}{
        Determine the pivot index~$j = \argmax \set[\big]{\abs[\big]{A_{i, k}^{(k - 1)}} : k \le i \le n}$\;
        \eIf{$A_{j, k}^{(k - 1)} \neq 0$}{
            Exchange the~$k$th and the~$j$th rows of~$A^{(k - 1)}$\;
            Evaluate the multiplier~$\tau^k \in \R^n$ with components
            \begin{algoempheq}[left={\tau_i^k = \empheqlbrace}]{alignat*=2}
                & A_{i, k}^{(k - 1)} / A_{k, k}^{(k - 1)}   && \quad \text{if~$i > k$, and}\\
                & 0                                         && \quad \text{otherwise}
            \end{algoempheq}
            Update~$A^{(k)} \gets (I_n - \tau^k e_k^{\T})A^{(k - 1)}$\;
        }{
            Set~$A^{(k)} \gets A^{(k - 1)}$\;
        }
    }
\end{algorithm}

\Citeauthor{Wilkinson_1963}'s backward error analysis~\cite[eq. (25.14)]{Wilkinson_1963} demonstrates that the growth factor of the Gaussian elimination, defined by
\begin{equation}
    \label{eq:gepp-growth-factor}
    \rho_n(A) \eqdef \frac{\max_{0 \le k \le n - 1} \norm{A^{(k)}}_{\max}}{\norm{A}_{\max}},
\end{equation}
determinates the numerical stability of~\cref{alg:gepp}, where~$\norm{\cdot}_{\max}$ denotes the max norm of a matrix, i.e., the largest absolute value of the matrix's entries.
More specifically, the~$\ell_{\infty}$-norm of the backward error of the computed solution is bounded from above by a term proportional to~$\rho_n(A)$.
To study the worst-case scenario of~\cref{alg:gepp}, we wish to determine how large~$\rho_n$ can be and hence, to solve
\begin{equation}
    \label{eq:gepp-opti}
    \max_{A \in \R^{n \times n}} \rho_n(A).
\end{equation}
Note that~$\R^{n \times n}$ is isomorphic to~$\R^{n^2}$ and hence, problem~\cref{eq:gepp-opti} could straightforwardly be formulated as problem~\cref{eq:nlcp-intro}.
Besides, although the growth factor is defined everywhere, it may not be continuous at the points yielding a tie in the selection of the pivot element.
Moreover, it is not differentiable at the points yielding a tie in any maximum operator in equation~\cref{eq:gepp-growth-factor}.
Hence, optimization methods based on derivative information cannot be used for such a problem.
In such a case, \gls{dfo} methods can help determine the optimal solution to problem~\cref{eq:gepp-opti}, and \citeauthor{Higham_Higham_1989} have used \gls{dfo} methods to determine the set of matrices yielding the optimal solution~\cite{Higham_Higham_1989}.

\subsection{Tuning nonlinear optimization methods}

Another well-known example of \gls{dfo} applications is the parameter tuning of nonlinear optimization methods~\cite{Audet_Orban_2006}.
Consider, for example, the simplified version of the trust-region method for solving problem~\cref{eq:nlcp-intro} when~$\iub = \ieq = \emptyset$ given in~\cref{alg:trust-region}.

\begin{algorithm}[htp]
    \caption{Trust region for unconstrained optimization}
    \label{alg:trust-region}
    \DontPrintSemicolon
    \KwData{Objective function~$\obj$, initial guess~$x^0 \in \R^n$, initial trust-region radius~$\Delta_0 > 0$, and parameters~$0 < \eta_1 \le \eta_2 < 1$ and~$0 < \theta_1 < 1 < \theta_2$.}
    \For{$k = 0, 1, \dots$}{
        Define a model~$m_k$ of~$\obj$ around~$x^k$\;
        Set the trial step~$d^k$ to any solution to
        \begin{algomathdisplay}
            \begin{aligned}
                \min        & \quad m_k(x^k + d)\\
                \text{s.t.} & \quad \norm{d} \le \Delta_k,\\
                            & \quad d \in \R^n
            \end{aligned}
        \end{algomathdisplay}
        Evaluate the trust-region ratio
        \begin{algomathdisplay}
            \rho_k \gets \frac{\obj(x^k) - \obj(x^k + d^k)}{m_k(x^k) - m_k(x^k + d^k)}
        \end{algomathdisplay}
        \eIf{$\rho_k \ge \eta_1$}{
            Update the trial point~$x^{k + 1} \gets x^k + d^k$\;
        }{
            Retain the trial point~$x^{k + 1} \gets x^k$\;
        }
        Update the trust-region radius
        \begin{algoempheq}[left={\Delta_{k + 1} \gets \empheqlbrace}]{alignat*=2}
            & \theta_1 \Delta_k && \quad \text{if~$\rho_k < \eta_1$,}\\
            & \Delta_k          && \quad \text{if~$\eta_1 \le \rho_k \le \eta_2$, and}\\
            & \theta_2 \Delta_k && \quad \text{otherwise}
        \end{algoempheq}
    }
\end{algorithm}

Such an algorithm depends on four parameters, namely~$\eta_1$,~$\eta_2$,~$\theta_1$, and~$\theta_2$.
A complete trust-region method includes more parameters, but we simplified the algorithm for sake of clarity.
For example, the threshold on the trust-region ratio for accepting the trial step should necessarily be the same as the one for enlarging the trust-region radius (and can even be zero).
To choose those parameters, we wish to minimize some measure of the method's performance (e.g., the sum of the number of function evaluations required to solve a given set of optimization problems),~$\rho$ say.
In other words, we wish to solve the optimization problem
\begin{subequations}
    \label{eq:tuning-opti}
    \begin{align}
        \min        & \quad \rho(\eta_1, \eta_2, \theta_1, \theta_2)\\
        \text{s.t.} & \quad 0 \le \eta_1 \le \eta_2 < 1,\\
                    & \quad 0 < \theta_1 < 1 < \theta_2.
    \end{align}
\end{subequations}
Derivatives of~$\rho$ are unassessable and may not even exist.
Such a problem is then solved using \gls{dfo} methods.
For instance, the \gls{mads} with minimal granularity and controlled decimals method~\cite{Audet_Digabel_Tribes_2019} was proposed by~\citeauthor{Audet_Digabel_Tribes_2019} to solve problem~\cref{eq:tuning-opti} with a fixed number of significant digits.
Similarly, the \gls{bfo} solver~\cite{Porcelli_Toint_2017}, a \gls{dfo} method for bound-constrained problems mixing continuous and discrete variables, has been self-tuned using the method presented above.

\subsection{Hyperparameter tuning in machine learning}
\label{subsec:machine-learning}

A newfangled example of \gls{dfo} applications is hyperparameter tuning in machine learning~\cite{Ghanbari_Scheinberg_2017}.
For instance, researchers at Google solve internal parameter tuning problems using Google Vizier~\cite{Golovin_Etal_2017}, the Google-internal service for performing black-box optimization
To illustrate this example, we consider the following hyperparameter tuning problem of a \gls{svm} for binary classification.
Given a labeled dataset~$\set{(x_i, y_i)}_{i = 1, 2, \dots, m} \subseteq \R^n \times \set{\pm 1}$, we build a \gls{svm} to classify the data with their respective labels.
A binary classification is obtained using a~$C$-\gls{svc}~\cite{Chang_Lin_2011} by solving the optimization problem
\begin{subequations}
    \label{eq:csvc}
    \begin{align}
        \min        & \quad \frac{1}{2} \norm{\omega}_2^2 + C \norm{\xi}_1\\
        \text{s.t.} & \quad y_i (\beta + \inner{\omega, \varphi_{\gamma}(x_i)}) \ge 1 - \xi_i, ~ i \in \set{1, 2, \dots, m},\\
                    & \quad \xi \ge 0,\\
                    & \quad (\omega, \beta, \xi) \in \R^{\ell} \times \R \times \R^m,
    \end{align}
\end{subequations}
where~$\varphi_{\gamma}$ is a function mapping the data to a higher-dimensional space~$\R^{\ell}$ for some given parameters~$\gamma > 0$ and~$C > 0$.
Given~$(\omega^{\ast}, \beta^{\ast}, \xi^{\ast}) \in \R^{\ell} \times \R \times \R^m$ a solution to problem~\cref{eq:csvc}, the~$C$-\gls{svc} classifies any data~$x \in \R^n$ according to
\begin{equation*}
    \delta(x) \eqdef \sgn(\beta^{\ast} + \inner{\omega^{\ast}, \varphi_{\gamma}(x)}).
\end{equation*}
In other words, the function~$\delta$ maps an observation~$x \in \R^n$ to a certain label.
It is clear that~$\delta$ depends on the two parameters~$C$ and~$\gamma$, and we want to find the optimal parameters for the given dataset.
Do to so, we use a~$5$-fold cross-validation of our model, as presented in~\cref{alg:cross-validation}, for some model's performance measure.

\begin{algorithm}[htp]
    \caption{$k$-fold cross-validation of a~$C$-\glsfmtshort{svc}}
    \label{alg:cross-validation}
    \DontPrintSemicolon
    \KwData{Labelled dataset~$\set{(x_i, y_i)}_{i = 1, 2, \dots, m} \subseteq \R^n \times \set{\pm 1}$ and fold number~$k > 0$.}
    Split randomly the dataset into~$k$ balanced groups\;
    \For{$i = 1, 2, \dots, k$}{
        Train the~$C$-\gls{svc}~\cref{eq:csvc} with the all the data except those in the~$i$th group\;
        Evaluate the model's performance on the data in the~$i$th group\;
    }
    Summarize the model's performance using the~$k$ samples\;
\end{algorithm}

A typical example of model's performance used in the~$k$-fold cross-validation is the model's accuracy, i.e., the percentage of testing data correctly classified.
The~\gls{auc}~\cite{Hanley_Mcneil_1982} is another example of model's performance, particularly effective for imbalanced datasets~\cite{Bradley_1997}.
In this example, hyperparameter tuning maximizes the model's performance provided by the~$k$-fold cross-validation of the~$C$-\gls{svc}~\cref{eq:csvc} with respect to the parameters~$C$ and~$\gamma$, subject to~$C > 0$ and~$\gamma > 0$.
It is clear that derivatives of the objective function of such a problem cannot be easily evaluated and may even not exist.
Hyperparameter tuning problems may be solved numerically using \gls{dfo} methods.
In a similar fashion, reinforcement learning in machine learning necessitates solving optimization problems for which derivatives cannot be assessed~\cite{Qian_Yu_2021}.

\subsection{Some industrial and engineering applications}

In engineering studies, \gls{mdo} designates a field that uses optimization methods for solving design problems.
Examples of \gls{mdo} applications include molecular conformational analysis~\cite{Alberto_Etal_2004,Meza_Martinez_1994}, helicopter rotor blade design~\cite{Booker_Etal_1998a,Booker_Etal_1998b,Serafini_1998}, groundwater supply and bioremediation engineering~\cite{Fowler_Etal_2008,Mugunthan_Shoemaker_Regis_2005,Yoon_Shoemaker_1999}, aeroacoustic shape design~\cite{Marsden_2004,Marsden_Etal_2004}, hydrodynamic design~\cite{Duvigneau_Visonneau_2004}, registration in medical imaging~\cite{Oeuvray_2005,Oeuvray_Bierlaire_2007}, reservoir engineering and engine calibration~\cite{Langouet_2011}, analog circuit design~\cite{Latorre_Etal_2019}, aircraft engine engineering~\cite{Gazaix_Etal_2019}, and chemical product design~\cite{Sun_Etal_2020} for instance.
\Gls{mdo} problems are solved using \gls{dfo} solvers such as the Nelder-Mead simplex method~\cite{Nelder_Mead_1965}, underlying the MATLAB function \texttt{fminsearch}, open-source optimizers such as NOMAD~\cite{Digabel_2011}, or CONDOR~\cite{Berghen_Bersini_2004}, or proprietary software such as Boeing's Design Explorer~\cite{Cramer_Gablonsky_2004}.

\section{Optimality conditions for smooth optimization}

\subsection{Local and global solutions}

\begin{definition}[Global solution]
    A point~$x^{\ast} \in \R^n$ is referred to as a \emph{global solution} to problem~\cref{eq:nlcp-intro} if~$x^{\ast} \in \Omega$ and~$\obj(x) \ge \obj(x^{\ast})$ for all~$x \in \Omega$.
\end{definition}

\begin{definition}[Local solution]
    A points~$x^{\ast} \in \R^n$ is referred to as
    \begin{itemize}
        \item a \emph{local solution} to problem~\cref{eq:nlcp-intro} if~$x^{\ast} \in \Omega$ and there exists an open neighborhood~$\mathcal{N} \subseteq \R^n$ of~$x^{\ast}$ such that~$\obj(x) \ge \obj(x^{\ast})$ for all~$x \in \mathcal{N} \cap \Omega$.
        \item a \emph{strict local solution} to problem~\cref{eq:nlcp-intro} if~$x^{\ast} \in \Omega$ and there exists an open neighborhood~$\mathcal{N} \subseteq \R^n$ of~$x^{\ast}$ such that~$\obj(x) > \obj(x^{\ast})$ for all~$x \in \mathcal{N} \cap \Omega \setminus \set{x^{\ast}}$.
        \item an \emph{isolated local solution} to problem~\cref{eq:nlcp-intro} if there exists an open neighborhood~$\mathcal{N} \subseteq \R^n$ of~$x^{\ast}$ such that it is the only local solution in~$\mathcal{N} \cap \Omega$.
    \end{itemize}
\end{definition}

\subsection{Constraint qualifications}

\begin{definition}[Active set]
    The \emph{active set}~$\mathcal{A}(x) \subseteq \iub \cup \ieq$ for problem~\cref{eq:nlcp-intro} at a point~$x \in \Omega$ is defined by
    \begin{equation*}
        \mathcal{A}(x) \eqdef \ieq \cup \set{i \in \iub : \con{i}(x) \ge 0}.
    \end{equation*}
\end{definition}

\begin{definition}[Constraint qualification]
    Let~$x \in \Omega$ be any feasible point of problem~\cref{eq:nlcp-intro}, denote~$\mathcal{A}(x)$ the active set for problem~\cref{eq:nlcp-intro} at~$x$, and assume that the constraints function~$\con{i}$ are differentiable at~$x$ for all~$i \in \mathcal{A}(x)$.
    We say that
    \begin{itemize}
        \item the \gls{licq} holds at~$x$ if the gradients~$\nabla \con{i}(x)$ are linearly independent for all~$i \in \mathcal{A}(x)$, and
        \item the \gls{mfcq} holds at~$x$ if the gradients~$\nabla \con{i}(x)$ are linearly independent for all~$i \in \ieq$ and there exists a vector~$d \in \R^n$ such that
        \begin{empheq}[left=\empheqlbrace]{alignat*=2}
            & \inner{\nabla \con{i}(x), d} < 0  && \quad \text{if~$i \in \mathcal{A}(x) \cap \iub$, and}\\
            & \inner{\nabla \con{i}(x), d} = 0  && \quad \text{if~$i \in \ieq$}.
        \end{empheq}
    \end{itemize}
\end{definition}

\begin{itemize}
    \item \gls{acq}.
    \item \gls{gcq}.
    \item \gls{lcq}.
    \item \gls{crcq}.
    \item \gls{cpld}.
    \item \gls{qncq}.
    \item \gls{sc}.
\end{itemize}

\subsection{First-order optimality conditions}

\begin{equation*}
    \lag(x, \lambda) \eqdef \obj(x) + \sum_{\mathclap{i \in \iub \cup \ieq}} \lambda_i \con{i}(x),
\end{equation*}
where~$\lambda = (\lambda_i)_{i \in \iub \cup \ieq}$ with~$\lambda_i \in \R$ for all~$i \in \iub \cup \ieq$.

\begin{theorem}[First-order necessary conditions]
    Let~$x^{\ast} \in \Omega$ be a given local solution to problem~\cref{eq:nlcp-intro}, assume that \gls{licq} holds at~$x^{\ast}$ and that the functions~$\obj$ and~$\con{i}$, for all~$i \in \iub \cup \ieq$, are continuously differentiable in an open neighborhood or~$x^{\ast}$.
    Then there exists a Lagrange multiplier~$\lambda^{\ast} = (\lambda_i^{\ast})_{i \in \iub \cup \ieq}$ with~$\lambda_i^{\ast} \in \R$ for all~$i \in \iub \cup \ieq$ such that
    \begin{subequations}
        \label{eq:kkt-intro}
        \begin{empheq}[left=\empheqlbrace]{alignat=2}
            & \nabla_x \lag(x^{\ast}, \lambda^{\ast}) = 0,  && \label{eq:kkt-intro-sta}\\
            & \con{i}(x^{\ast}) \le 0,                      && \quad i \in \iub, \label{eq:kkt-intro-pfub}\\
            & \con{i}(x^{\ast}) = 0,                        && \quad i \in \ieq, \label{eq:kkt-intro-pfeq}\\
            & \lambda_i^{\ast} \con{i}(x^{\ast}) = 0,       && \quad i \in \iub, \label{eq:kkt-intro-csl}\\
            & \lambda_i^{\ast} \ge 0,                       && \quad i \in \iub. \label{eq:kkt-intro-df}
        \end{empheq}
    \end{subequations}
\end{theorem}

The conditions~\cref{eq:kkt-intro} are referred to as the \gls{kkt} conditions.
More specifically, condition~\cref{eq:kkt-intro-sta} is referred to as the \emph{stationarity} condition, conditions~\cref{eq:kkt-intro-pfub,eq:kkt-intro-pfeq} as the \emph{primal feasibility} conditions, condition~\cref{eq:kkt-intro-csl} as the \emph{complementary slackness} condition, and condition~\cref{eq:kkt-intro-df} as the \emph{dual feasibility} condition.
Note that this theorem holds for more general constraint qualifications, such as the \gls{mfcq}.
Moreover, the smoothness assumptions on the objective and constraint functions can be relaxed, using some generalized derivatives, but we want to keep this theoretical development as simple as possible for sake of clarity.

\subsection{Second-order optimality conditions}

\begin{theorem}[Second-order necessary conditions]
    Let~$x^{\ast} \in \Omega$ be a given local solution to problem~\cref{eq:nlcp-intro}, assume that \gls{licq} holds at~$x^{\ast}$ and that the functions~$\obj$ and~$\con{i}$, for all~$i \in \iub \cup \ieq$, are twice continuously differentiable in an open neighborhood or~$x^{\ast}$.
    Denote~$\mathcal{A}(x^{\ast})$ the active set for problem~\cref{eq:nlcp-intro} at~$x^{\ast}$.
    Let~$\lambda^{\ast} = (\lambda_i^{\ast})_{i \in \iub \cup \ieq}$ with~$\lambda_i^{\ast} \in \R$ for all~$i \in \iub \cup \ieq$ be a Lagrange multiplier satisfying the KKT condition~\cref{eq:kkt-intro}, and let~$z \in \R^n$ be any vector such that
    \begin{subequations}
        \label{eq:second-order-intro}
        \begin{empheq}[left=\empheqlbrace]{alignat*=2}
            & \inner{\nabla \con{i}(x^{\ast}), z} = 0,      && \quad \text{if~$i \in \ieq$,}\\
            & \inner{\nabla \con{i}(x^{\ast}), z} = 0,      && \quad \text{if~$i \in \mathcal{A}(x^{\ast}) \cap \iub$ and~$\lambda_i^{\ast} > 0$, and}\\
            & \inner{\nabla \con{i}(x^{\ast}), z} \ge 0,    && \quad \text{if~$i \in \mathcal{A}(x^{\ast}) \cap \iub$ and~$\lambda_i^{\ast} = 0$.}
        \end{empheq}
    \end{subequations}
    Then~$\inner{z, \nabla_{x, x}^2 \lag(x^{\ast}, \lambda^{\ast}) z} \ge 0$.
\end{theorem}

\begin{theorem}[Second-order sufficient conditions]
    Let~$x^{\ast} \in \Omega$ be any feasible point of problem~\cref{eq:nlcp-intro}, and assume that \gls{licq} holds at~$x^{\ast}$, that there exists a Lagrange multiplier~$\lambda^{\ast} = (\lambda_i^{\ast})_{i \in \iub \cup \ieq}$ with~$\lambda_i^{\ast} \in \R$ for all~$i \in \iub \cup \ieq$ satisfying the KKT condition~\cref{eq:kkt-intro}, and that the functions~$\obj$ and~$\con{i}$, for all~$i \in \iub \cup \ieq$, are twice continuously differentiable in an open neighborhood or~$x^{\ast}$.
    If for all nonzero vector~$z \in \R^n \setminus \set{0}$ satisfying the conditions~\cref{eq:second-order-intro} we have
    \begin{equation*}
        \inner{z, \nabla_{x, x}^2 \lag(x^{\ast}, \lambda^{\ast}) z} > 0,
    \end{equation*}
    then~$x^{\ast}$ is a strict local solution to problem~\cref{eq:nlcp-intro}.
\end{theorem}

\section{Methodology of \glsfmtlong{dfo} algorithms}

\subsection{Frameworks and algorithms for \glsfmtlong{dfo}}

\begin{itemize}
    \item Direct-search and model-based methods.
    \item Line-search and trust-region methods.
    \item Filter methods and hybrid methods.
    \item implicit filtering methods (hybrid between direct-search and line-search).
\end{itemize}

\subsection{Examples of \glsfmtlong{dfo} methods}

\begin{itemize}
    \item The earliest work on numerical \gls{dfo} is attributed to~\citeauthor{Fermi_Metropolis_1952}~\cite{Fermi_Metropolis_1952}, who developed a nonlinear least-squares solver using a \gls{dfo} coordinate search on MANIAC, an early computer based on the von Neumann architecture.
    \item Nelder-Mead method~\cite{Nelder_Mead_1965}.
    \item BFO~\cite{Porcelli_Toint_2017}.
    \item DFO~\cite{Conn_Scheinberg_Toint_1998}.
    \item NOMAD~\cite{Digabel_2011}.
    \item MNH~\cite{Wild_2008}
\end{itemize}

\section{Benchmarking tools for \glsfmtlong{dfo} methods}

\subsection{Performance profiles}

\subsection{Data profiles}
