%% contents/cobyqa-introduction.tex
%% Copyright 2021-2022 Tom M. Ragonneau
%
% This work may be distributed and/or modified under the
% conditions of the LaTeX Project Public License, either version 1.3
% of this license or (at your option) any later version.
% The latest version of this license is in
%   http://www.latex-project.org/lppl.txt
% and version 1.3 or later is part of all distributions of LaTeX
% version 2005/12/01 or later.
%
% This work has the LPPL maintenance status `maintained'.
%
% The Current Maintainer of this work is Tom M. Ragonneau.
\chapter{\glsfmttext{cobyqa} \textemdash\ a constrained \glsfmtlong{dfo} method}
\label{ch:cobyqa-introduction}

\section{Statement of the problem}

In this chapter, we introduce a new model-based \gls{dfo} method for solving nonlinearly-constrained problems of the form
\begin{subequations}
    \label{eq:problem-cobyqa}
    \begin{align}
        \min        & \quad \obj(x) \label{eq:problem-cobyqa-obj}\\
        \text{s.t.} & \quad \con{i}(x) \le 0, ~ i \in \iub, \label{eq:problem-cobyqa-ub}\\
                    & \quad \con{i}(x) = 0, ~ i \in \ieq,\\
                    & \quad \xl \le x \le \xu, \label{eq:problem-cobyqa-bd}\\
                    & \quad x \in \R^n, \nonumber
    \end{align}
\end{subequations}
where the objective and constraint functions~$\obj$ and~$\con{i}$, with~$i \in \iub \cup \ieq$, are real-valued functions on~$\R^n$, with the sets of indices~$\iub$ and~$\ieq$ being finite (perhaps empty) and disjoint, and where the bounds~$\xl \in (\R \cup \set{-\infty})^n$ and~$\xu \in (\R \cup \set{\infty})^n$ satisfy~$\xl < \xu$.
The requirement on the bounds is weak, as otherwise, the problem~\cref{eq:problem-cobyqa} would be either infeasible, or would admit fix variables.
The solver we develop, named~\gls{cobyqa} after \emph{\glsdesc{cobyqa}}, uses only function values of~$\obj$ and~$\con{i}$, with~$i \in \iub \cup \ieq$, but not derivatives.

\section{Management of the bound constraints}

From a theoretical standpoint, the bound constraints~\cref{eq:problem-cobyqa-bd} could be included in the general inequality constraints~\cref{eq:problem-cobyqa-ub}.
However, we consider them separatly, because they often represent inalienable physical or theoretical restrictions.
In other words, in many applications, the objective function~\cref{eq:problem-cobyqa-obj} is not defined if the bounds constraints~\cref{eq:problem-cobyqa-bd} are violated.
For instance, the tuning of nonlinear optimization methods (see \cref{subsec:tuning-nonlinear-optimization-methods}) involves a bounds that cannot be violated, as the optimization methods that are tuned may not be defined otherwise.
Very similar observations can be made on hyperparameter tuning in machine learning (see \cref{subsec:machine-learning}).

Therefore, every point that \gls{cobyqa} encounters always respects these bounds.
This is also the case for the \gls{bobyqa} method, presented in \cref{subsec:newuoa-bobyqa-lincoa}.
Note that they are very simple constraints.
It is, for example, trivial to check whether a point is feasible with respect to the bound constraints~\cref{eq:problem-cobyqa-bd}, and easy to project any point onto the bound constraints.


\section{The \glsfmtlong{sqp} method}
\label{sec:sqp-method}

For sake of clarity we assume throughout the \cref{sec:sqp-method} that no bound constraint is provided.
Therefore, the problem we consider is of the form
\begin{subequations}
    \label{eq:problem-cobyqa-no-bounds}
    \begin{align}
        \min        & \quad \obj(x)\\
        \text{s.t.} & \quad \con{i}(x) \le 0, ~ i \in \iub,\\
                    & \quad \con{i}(x) = 0, ~ i \in \ieq,\\
                    & \quad x \in \R^n. \nonumber
    \end{align}
\end{subequations}
As we mentioned earlier, if bound constraints are supplied, they can be included in the inequality constraints~\cref{eq:problem-cobyqa-ub} for theoretical purposes.
The Lagrangian function that we consider is defined by
\begin{equation*}
    \lag(x, \lambda) \eqdef \obj(x) + \sum_{\mathclap{i \in \iub \cup \ieq}} \lambda_i \con{i}(x), \quad \text{for~$x \in \R^n$ and~$\lambda_i \in \R$ for~$i \in \iub \cup \ieq$},
\end{equation*}
where~$\lambda = [\lambda_i]_{i \in \iub \cup \ieq}^{\T}$ is the dual variable of the considered problem.

\subsection{Overview of the method}

The \gls{sqp} method of \citeauthor{Wilson_1963}~\cite{Wilson_1963}, \citeauthor{Han_1976}~\cite{Han_1976,Han_1977}, and \citeauthor{Powell_1978a}~\cite{Powell_1978a,Powell_1978b} is known to be one of the most powerful method for solving the problem~\cref{eq:problem-cobyqa-no-bounds} when derivatives of~$\obj$ and~$\con{i}$, with~$i \in \iub \cup \ieq$, are available.
Given an iterate~$x^k \in \R^n$, it generates a step~$d^k \in \R^n$ by solving approximately
\begin{subequations}
    \label{eq:sqp-subproblem}
    \begin{align}
        \min        & \quad \nabla \obj(x^k)^{\T} d + \frac{1}{2} d^{\T} \nabla_{x, x}^2 \lag(x^k, \lambda^k) d \label{eq:sqp-subproblem-obj}\\
        \text{s.t.} & \quad \con{i}(x^k) + \nabla \con{i}(x^k)^{\T} d \le 0, ~ i \in \iub,\\
                    & \quad \con{i}(x^k) + \nabla \con{i}(x^k)^{\T} d = 0, ~ i \in \ieq,\\
                    & \quad d \in \R^n,
    \end{align}
\end{subequations}
for some Lagrange multiplier~$\lambda^k = [\lambda_i^k]_{i \in \iub \cup \ieq}^{\T}$ with~$\lambda_i^k \in \R$ for~$i \in \iub \cup \ieq$.
Further, it sets the next iterate~$x^{k + 1}$ to~$x^k + d^k$.
If the second-order derivative of~$\lag$ with respect to the decision variables is unavailable, the term~$\nabla^2 \lag_{x, x}(x^k, \lambda^k)$ can be replace with an approximation of it (e.g., a quasi-Newton approximation).
Under some mild assumptions on the regularity of the objective and constraint functions, the \gls{sqp} method is locally Q-superlinearly convergent.

\subsection{Interpretation of the subproblem}

To get some insight into the origin of the \gls{sqp} method, we now interpret the \gls{sqp} subproblem~\cref{eq:sqp-subproblem}.
Note that the second-order term of the objective function~\cref{eq:sqp-subproblem-obj} involves the Lagrangian function of problem~\cref{eq:problem-cobyqa-no-bounds} and not only its objective function.
It is in fact necessary due to the nonlinearity of the constraints.
To comprehend this fact, we consider the~$2$-dimensional example of \citeauthor{Boggs_Tolle_1995}~\cite{Boggs_Tolle_1995}
\begin{align*}
    \min        & \quad -x_1 - \frac{x_2^2}{2}\\
    \text{s.t.} & \quad \norm{x}^2 - 1 = 0,\\
                & \quad x \in \R^2,
\end{align*}
whose solution is~$[1, 0]^{\T}$.
Given a perturbation~$\epsilon > 0$ and an iterate~$x^k = [1 + \epsilon, 0]^{\T}$, if the second-order term in~\cref{eq:sqp-subproblem-obj} included only~$\nabla^2 \obj(x^k)$, the \gls{sqp} subproblem would be
\begin{align*}
    \min        & \quad -d_1 - \frac{d_2^2}{2}\\
    \text{s.t.} & \quad d_1 = -\frac{\epsilon (2 + \epsilon)}{2 (1 + \epsilon)},\\
                & \quad d \in \R^2,
\end{align*}
which is unbounded from below, regardless of the magnitude of~$\epsilon$.
Therefore, in this example, including the Lagrangian function in the second-order term of the objective function~\cref{eq:sqp-subproblem-obj} is crucial.
This behavior is in fact not specific to this example, but is much more general (see, e.g., \cite[ch.~18]{Nocedal_Wright_2006}).

\subsubsection{Approximation of the \glsfmtlong{kkt} conditions}

According to \cref{thm:first-order-necessary-conditions}, if~$x^{\ast} \in \R^n$ is a local solution to the problem~\cref{eq:problem-cobyqa-no-bounds}, under some mild assumptions, there exists a Lagrange multiplier~$\lambda^{\ast} = [\lambda_i^{\ast}]_{i \in \iub \cup \ieq}^{\T}$ with~$\lambda_i^{\ast} \in \R$ for all~$i \in \iub \cup \ieq$ such that
\begin{subequations}
    \begin{empheq}[left=\empheqlbrace]{alignat=2}
        & \nabla_x \lag(x^{\ast}, \lambda^{\ast}) = 0,  && \\
        & \con{i}(x^{\ast}) \le 0,                      && \quad \text{if~$i \in \iub$,}\\
        & \con{i}(x^{\ast}) = 0,                        && \quad \text{if~$i \in \ieq$,}\\
        & \lambda_i^{\ast} \con{i}(x^{\ast}) = 0,       && \quad \text{if~$i \in \iub$,} \label{eq:sqp-kkt-complementary-slackness}\\
        & \lambda_i^{\ast} \ge 0,                       && \quad \text{if~$i \in \iub$.}
    \end{empheq}
\end{subequations}
Let~$(x^k, \lambda^k)$ be some approximation of~$(x^{\ast}, \lambda^{\ast})$, and let~$(d^k, u^k)$ satisfy
\begin{subequations}
    \label{eq:sqp-kkt-step}
    \begin{empheq}[left=\empheqlbrace]{alignat=2}
        & \nabla_x \lag(x^{k}, \lambda^{k} + u^k) + \nabla_{x, x}^2 \lag(x^{k}, \lambda^{k}) d^k = 0,   && \\
        & \con{i}(x^k) + \nabla \con{i}(x^k)^{\T} d^k \le 0,                                            && \quad \text{if~$i \in \iub$,}\\
        & \con{i}(x^k) + \nabla \con{i}(x^k)^{\T} d^k = 0,                                              && \quad \text{if~$i \in \ieq$,}\\
        & (\lambda_i^{k} + u_i^k) [\con{i}(x^k) + \nabla \con{i}(x^k)^{\T} d^k] = 0,                && \quad \text{if~$i \in \iub$,} \label{eq:sqp-kkt-step-complementary-slackness}\\
        & \lambda_i^{k} + u_i^k \ge 0,                                                                  && \quad \text{if~$i \in \iub$.}
    \end{empheq}
\end{subequations}
Note that this conditions resemble the first-order Taylor approximation at~$(x^k, \lambda^k)$, in which case~$(d^k, u^k)$ would be a Newton step.
However, the condition~\cref{eq:sqp-kkt-step-complementary-slackness} is not directly the linearization of~\cref{eq:sqp-kkt-complementary-slackness}, but it includes the second-order term~$u_i^k \nabla \con{i}(x^k)^{\T} d^k$.
We remark that the conditions~\cref{eq:sqp-kkt-step} are nothing but the \gls{kkt} conditions of the \gls{sqp} subproblem~\cref{eq:sqp-subproblem} associated with the Lagrange multiplier~$\lambda^k + u^k$.

\subsubsection{Approximation of a modified Lagrangian}

A second interpretation of the \gls{sqp} subproblem~\cref{eq:sqp-subproblem} is as follows.
Given an iterate~$x^k \in \R^n$, let~$\lag[k]$ be the modified Lagrangian function
\begin{equation*}
    \lag[k](x, \lambda) \eqdef \obj(x) + \sum_{\mathclap{i \in \iub \cup \ieq}} \lambda_i \delta_i^k(x), \quad \text{for~$x \in \R^n$ and~$\lambda_i \in \R$ for~$i \in \iub \cup \ieq$},
\end{equation*}
where~$\lambda = [\lambda_i]_{i \in \iub \cup \ieq}^{\T}$, and where~$\delta_i^k$, for~$i \in \iub \cup \ieq$, denotes the departure from linearity\footnote{When~$\con{i}$ is strictly convex, it is the Bregman distance~\cite{Bregman_1967} associated with~$\con{i}$ for the point~$x^k$.}~\cite{Robinson_1972,Hoek_1982} associated with~$\con{i}$ for the point~$x^k$, defined by
\begin{equation*}
    \delta_i^k(x) \eqdef \con{i}(x) - \con{i}(x^k) - \nabla \con{i}(x^k)^{\T} (x - x^k), \quad \text{for~$x \in \R^n$.}
\end{equation*}
The \gls{sqp} subproblem~\cref{eq:sqp-subproblem} can then be seen as the minimization of the second-order Taylor approximation of~$\lag[k]$ subject to the linearized constraints, i.e.,
\begin{align}
    \min        & \quad \nabla_x \lag[k](x^k, \lambda^k)^{\T} d + \frac{1}{2} d^{\T} \nabla_{x, x}^2 \lag[k](x^k, \lambda^k) d\\
    \text{s.t.} & \quad \con{i}(x^k) + \nabla \con{i}(x^k)^{\T} d \le 0, ~ i \in \iub,\\
                & \quad \con{i}(x^k) + \nabla \con{i}(x^k)^{\T} d = 0, ~ i \in \ieq,\\
                & \quad d \in \R^n.
\end{align}

\subsubsection{Approximation of the augmented Lagrangian}

The original problem~\cref{eq:problem-cobyqa-no-bounds} can be reformulated by introducing a slack variable~$s$ as
\begin{align}
    \min        & \quad \obj(x)\\
    \text{s.t.} & \quad \con{i}(x) + s_i = 0, ~ i \in \iub,\\
                & \quad \con{i}(x) = 0, ~ i \in \ieq,\\
                & \quad x \in \R^n, ~ s \ge 0.
\end{align}
We assume here without loss of generality that~$\ieq = \emptyset$.
Let~$\lag[\mathsf{A}]$ be the augmented Lagrangian of this problem, i.e.,
\begin{equation*}
    \lag[\mathsf{A}](x, s, \lambda) \eqdef \obj(x) + \sum_{i \in \iub} \bigg[ \lambda_i (\con{i}(x) + s_i) + \frac{\gamma}{2} (\con{i}(x) + s_i)^2 \bigg],
\end{equation*}
for~$x \in \R^n$ and~$s_i, \lambda_i \in \R$ for~$i \in \iub$, where~$\gamma \ge 0$ is a given penalty parameter.
We then remark that the augmented Lagrangian function of the \gls{sqp} subproblem~\cref{eq:sqp-subproblem} is the Taylor-like approximation of~$\lag[\textsf{A}]$ obtained by replacing~$\obj$ by its second-order approximation,~$\con{i}$ in the left part of the summed term by its second-order approximation, and~$\con{i}$ in the right part of the summed term by its first-order approximation. The function~$\con{i}$, with~$i \in \iub$, in the right part of the summed term is replaced only a linear function to obtain a quadratic approximation of~$\lag[\mathsf{A}]$, and not a quartic one.

\subsection{A derivative-free \glsfmtlong{sqp} method}

At the~$k$th iteration, \gls{cobyqa} builds the models~$\objm[k]$ and~$\conm[k]{i}$ of the objective function~$\obj$ and the constraints functions~$\con{i}$, with~$i \in \iub \cup \ieq$, using the derivative-free Broyden updates presented in~\cref{subsec:symmetric-broyden-updates}.
A poised interpolation set~$\xpt[k] \subseteq \R^n$ is maintained, whose cardinal number is fixed.
The only restriction is
\begin{equation*}
    n + 2 \le \card(\xpt[k]) \le \frac{1}{2} (n + 1) (n + 2),
\end{equation*}
as otherwise, the set~$\xpt[k]$ cannot be poised (see \cref{subsec:symmetric-broyden-updates} for details).
The derivative-free variation of the \gls{sqp} method we consider then has the subproblem
\begin{align*}
    \min        & \quad \nabla \objm[k](x^k)^{\T} d + \frac{1}{2} d^{\T} \nabla_{x, x}^2 \lagm[k](x^k, \lambda^k) d \label{eq:sqp-subproblem-obj}\\
    \text{s.t.} & \quad \conm[k]{i}(x^k) + \nabla \conm[k]{i}(x^k)^{\T} d \le 0, ~ i \in \iub,\\
                & \quad \conm[k]{i}(x^k) + \nabla \conm[k]{i}(x^k)^{\T} d = 0, ~ i \in \ieq,\\
                & \quad d \in \R^n,
\end{align*}
where~$\lagm[k]$ is defined by
\begin{equation*}
    \lagm[k](x, \lambda) \eqdef \objm[k](x) + \sum_{\mathclap{i \in \iub \cup \ieq}} \lambda_i \conm[k]{i}(x), \quad \text{for~$x \in \R^n$ and~$\lambda_i \in \R$ for~$i \in \iub \cup \ieq$}.
\end{equation*}

\section{The trust-region framework}

The \gls{sqp} framework suffers from one major deffect: it is a local method.
Therefore, the convergence of the method cannot be ensured for any initial guess~$x^0 \in \R^n$.
To cope with this difficulty, the method is usually embedded in a globalization strategy, such as a line-search or a trust-region framework.
We consider in this thesis trust-region methods only.
When combining all the information, the trust-region \gls{sqp} subproblem considered by \gls{cobyqa} is
\begin{subequations}
    \begin{align}
        \min        & \quad \nabla \objm[k](x^k)^{\T} d + \frac{1}{2} d^{\T} \nabla_{x, x}^2 \lagm[k](x^k, \lambda^k) d \label{derivative-free-trust-region-sqp-subproblem-obj}\\
        \text{s.t.} & \quad \conm[k]{i}(x^k) + \nabla \conm[k]{i}(x^k)^{\T} d \le 0, ~ i \in \iub,\\
                    & \quad \conm[k]{i}(x^k) + \nabla \conm[k]{i}(x^k)^{\T} d = 0, ~ i \in \ieq,\\
                    & \quad \xl \le x^k + d \le \xu,\\
                    & \quad \norm{d} \le \Delta^k,\\
                    & \quad d \in \R^n,
    \end{align}
\end{subequations}
where~$\norm{\cdot}$ denotes the Euclidean norm,~$\Delta^k > 0$ is the current trust-region radius, and~$\lagm[k]$ denotes the Lagrangian function of the problem~\cref{eq:problem-cobyqa}.
Note that only the second-order derivative of this Lagrangian with respect to~$x$ intervene in this subproblem.
Therefore, the bound constraints do not play any role in the objective function~\cref{derivative-free-trust-region-sqp-subproblem-obj}.

\subsection{Merit functions and penalty coefficients}

\subsection{Composite-step approach}

\section{Outline of the \glsfmttext{cobyqa} method}

\subsection{Interpolation-based quadratic models}

\subsection{Geometry of the interpolation set}

\subsection{Estimation of the Lagrange multipliers}

\subsection{Maratos effect and \glsfmtlong{soc}}
