%% contents/cobyqa-introduction.tex
%% Copyright 2021-2022 Tom M. Ragonneau
%
% This work may be distributed and/or modified under the
% conditions of the LaTeX Project Public License, either version 1.3
% of this license or (at your option) any later version.
% The latest version of this license is in
%   http://www.latex-project.org/lppl.txt
% and version 1.3 or later is part of all distributions of LaTeX
% version 2005/12/01 or later.
%
% This work has the LPPL maintenance status `maintained'.
%
% The Current Maintainer of this work is Tom M. Ragonneau.
\chapter{\glsfmttext{cobyqa} \textemdash\ a constrained \glsfmtlong{dfo} method}
\label{ch:cobyqa-introduction}

\section{Statement of the problem}

In this chapter, we introduce a new model-based \gls{dfo} method for solving nonlinearly-constrained problems of the form
\begin{subequations}
    \label{eq:problem-cobyqa}
    \begin{align}
        \min        & \quad \obj(\iter) \label{eq:problem-cobyqa-obj}\\
        \text{s.t.} & \quad \con{i}(\iter) \le 0, ~ i \in \iub, \label{eq:problem-cobyqa-ub}\\
                    & \quad \con{i}(\iter) = 0, ~ i \in \ieq,\\
                    & \quad \iter \in \R^n, \nonumber
    \end{align}
\end{subequations}
where the objective and constraint functions~$\obj$ and~$\con{i}$, with~$i \in \iub \cup \ieq$, are real-valued functions on~$\R^n$, with the sets of indices~$\iub$ and~$\ieq$ being finite (perhaps empty) and disjoint.
The solver we develop, named~\gls{cobyqa} after \emph{\glsdesc{cobyqa}}, uses only function values of~$\obj$ and~$\con{i}$, with~$i \in \iub \cup \ieq$, but not derivatives.

\section{Management of bound constraints}

Note that the general inequality constraints~\cref{eq:problem-cobyqa-ub} may include bound constraints.
In general, the problem~\cref{eq:problem-cobyqa} can be formulated as
\begin{subequations}
    \label{eq:problem-cobyqa-bounds}
    \begin{align}
        \min        & \quad \obj(\iter) \label{eq:problem-cobyqa-bounds-obj}\\
        \text{s.t.} & \quad \con{i}(\iter) \le 0, ~ i \in \iub, \label{eq:problem-cobyqa-bounds-ub}\\
                    & \quad \con{i}(\iter) = 0, ~ i \in \ieq,\\
                    & \quad \xl \le x \le \xu, \label{eq:problem-cobyqa-bounds-bd}\\
                    & \quad \iter \in \R^n, \nonumber
    \end{align}
\end{subequations}
where the bounds~$\xl \in (\R \cup \set{-\infty})^n$ and~$\xu \in (\R \cup \set{\infty})^n$ satisfy~$\xl < \xu$.
The requirement on the bounds is weak, as otherwise, the problem~\cref{eq:problem-cobyqa-bounds} would be either infeasible, or would admit fix variables.
Of course, when written in this form, the indices of the inequality constraints in~$\iub$ might be different from the one in~\cref{eq:problem-cobyqa}, not to include the bound constraints anymore.
Note that they are very simple constraints.
It is, for example, trivial to check whether a point is feasible with respect to the bound constraints~\cref{eq:problem-cobyqa-bounds-bd}, and easy to project any point onto the bound constraints.

In the theoretical development presented hereinafter, we consider problems of the form~\cref{eq:problem-cobyqa} for clarity.
However, when developing the implementation of \gls{cobyqa}, we consider the bound constraints separatly from the inequality constraints~\cref{eq:problem-cobyqa-bounds-ub}, because they often represent inalienable physical or theoretical restrictions.
In other words, in many applications, the objective function~\cref{eq:problem-cobyqa-bounds-obj} is not defined if the bounds constraints~\cref{eq:problem-cobyqa-bounds-bd} are violated.
For instance, the tuning of nonlinear optimization methods (see \cref{subsec:tuning-nonlinear-optimization-methods}) involves a bounds that cannot be violated, as the optimization methods that are tuned may not be defined otherwise.
Very similar observations can be made on hyperparameter tuning in machine learning (see \cref{subsec:machine-learning}).
Therefore, every point that \gls{cobyqa} encounters always respects these bounds.
This is also the case for the \gls{bobyqa} method, presented in \cref{subsec:newuoa-bobyqa-lincoa}.

\section{The \glsfmtlong{sqp} method}

Throughout this section, we consider problems of the form~\cref{eq:problem-cobyqa}.
The Lagrangian function of the problem~\cref{eq:problem-cobyqa} is defined by
\begin{equation*}
    \lag(\iter, \lambda) \eqdef \obj(\iter) + \sum_{\mathclap{i \in \iub \cup \ieq}} \lambda_i \con{i}(\iter), \quad \text{for~$\iter \in \R^n$ and~$\lambda_i \in \R$, with~$i \in \iub \cup \ieq$,}
\end{equation*}
where~$\lambda = [\lambda_i]_{i \in \iub \cup \ieq}^{\T}$ is the dual variable of the considered problem.

\subsection{Overview of the method}

The \gls{sqp} method of \citeauthor{Wilson_1963}~\cite{Wilson_1963}, \citeauthor{Han_1976}~\cite{Han_1976,Han_1977}, and \citeauthor{Powell_1978a}~\cite{Powell_1978a,Powell_1978b} is known to be one of the most powerful method for solving the problem~\cref{eq:problem-cobyqa} when derivatives of~$\obj$ and~$\con{i}$, with~$i \in \iub \cup \ieq$, are available.
Given an iterate~$\iter[k] \in \R^n$, it generates a step~$d^k \in \R^n$ by solving approximately
\begin{subequations}
    \label{eq:sqp-subproblem}
    \begin{align}
        \min        & \quad \nabla \obj(\iter[k])^{\T} \step + \frac{1}{2} \step^{\T} \nabla_{x, x}^2 \lag(\iter[k], \lambda^k) \step \label{eq:sqp-subproblem-obj}\\
        \text{s.t.} & \quad \con{i}(\iter[k]) + \nabla \con{i}(\iter[k])^{\T} \step \le 0, ~ i \in \iub,\\
                    & \quad \con{i}(\iter[k]) + \nabla \con{i}(\iter[k])^{\T} \step = 0, ~ i \in \ieq,\\
                    & \quad \step \in \R^n,
    \end{align}
\end{subequations}
for some Lagrange multiplier~$\lambda^k = [\lambda_i^k]_{i \in \iub \cup \ieq}^{\T}$ with~$\lambda_i^k \in \R$ for~$i \in \iub \cup \ieq$.
Further, it sets the next iterate~$\iter[k + 1]$ to~$\iter[k] + d^k$.
If the second-order derivative of~$\lag$ with respect to the decision variables is unavailable, the term~$\nabla^2 \lag_{x, x}(\iter[k], \lambda^k)$ can be replace with an approximation of it (e.g., a quasi-Newton approximation).
Under some mild assumptions on the regularity of the objective and constraint functions, the \gls{sqp} method is locally Q-superlinearly convergent.

\subsection{Interpretation of the subproblem}

To get some insight into the origin of the \gls{sqp} method, we now interpret the \gls{sqp} subproblem~\cref{eq:sqp-subproblem}.
Note that the second-order term of the objective function~\cref{eq:sqp-subproblem-obj} involves the Lagrangian function of problem~\cref{eq:problem-cobyqa} and not only its objective function.
It is in fact necessary due to the possible nonlinearity of the constraints.
To comprehend this fact, we consider the~$2$-dimensional example of \citeauthor{Boggs_Tolle_1995}~\cite{Boggs_Tolle_1995}
\begin{align*}
    \min        & \quad -x_1 - \frac{x_2^2}{2}\\
    \text{s.t.} & \quad \norm{x}^2 - 1 = 0,\\
                & \quad \iter \in \R^2,
\end{align*}
whose solution is~$[1, 0]^{\T}$.
Given a perturbation~$\epsilon > 0$ and an iterate~$\iter[k] = [1 + \epsilon, 0]^{\T}$, if the second-order term in~\cref{eq:sqp-subproblem-obj} included only~$\nabla^2 \obj(\iter[k])$, the \gls{sqp} subproblem would be
\begin{align*}
    \min        & \quad -\step_1 - \frac{\step_2^2}{2}\\
    \text{s.t.} & \quad \step_1 = -\frac{\epsilon (2 + \epsilon)}{2 (1 + \epsilon)},\\
                & \quad \step \in \R^2,
\end{align*}
which is unbounded from below, regardless of the magnitude of~$\epsilon$.
Therefore, in this example, including the Lagrangian function in the second-order term of the objective function~\cref{eq:sqp-subproblem-obj} is crucial.
This behavior is in fact not specific to this example, but is much more general (see, e.g., \cite[ch.~18]{Nocedal_Wright_2006}).

\subsubsection{Approximation of the \glsfmtlong{kkt} conditions}

According to \cref{thm:first-order-necessary-conditions}, if~$\iter[\ast] \in \R^n$ is a local solution to the problem~\cref{eq:problem-cobyqa}, under some mild assumptions, there exists a Lagrange multiplier~$\lambda^{\ast} = [\lambda_i^{\ast}]_{i \in \iub \cup \ieq}^{\T}$ with~$\lambda_i^{\ast} \in \R$ for all~$i \in \iub \cup \ieq$ such that
\begin{subequations}
    \label{eq:sqp-kkt}
    \begin{empheq}[left=\empheqlbrace]{alignat=2}
        & \nabla_x \lag(\iter[\ast], \lambda^{\ast}) = 0,   && \\
        & \con{i}(\iter[\ast]) \le 0,                       && \quad \text{if~$i \in \iub$,}\\
        & \con{i}(\iter[\ast]) = 0,                         && \quad \text{if~$i \in \ieq$,}\\
        & \lambda_i^{\ast} \con{i}(\iter[\ast]) = 0,        && \quad \text{if~$i \in \iub$,} \label{eq:sqp-kkt-complementary-slackness}\\
        & \lambda_i^{\ast} \ge 0,                           && \quad \text{if~$i \in \iub$.}
    \end{empheq}
\end{subequations}
Let~$(\iter[k], \lambda^k)$ be some approximation of~$(\iter[\ast], \lambda^{\ast})$, and let~$(d^k, u^k)$ satisfy
\begin{subequations}
    \label{eq:sqp-kkt-step}
    \begin{empheq}[left=\empheqlbrace]{alignat=2}
        & \nabla_x \lag(\iter[k], \lambda^k + u^k) + \nabla_{x, x}^2 \lag(\iter[k], \lambda^k) d^k = 0,   && \\
        & \con{i}(\iter[k]) + \nabla \con{i}(\iter[k])^{\T} d^k \le 0,                                            && \quad \text{if~$i \in \iub$,}\\
        & \con{i}(\iter[k]) + \nabla \con{i}(\iter[k])^{\T} d^k = 0,                                              && \quad \text{if~$i \in \ieq$,}\\
        & (\lambda_i^{k} + u_i^k) [\con{i}(\iter[k]) + \nabla \con{i}(\iter[k])^{\T} d^k] = 0,                && \quad \text{if~$i \in \iub$,} \label{eq:sqp-kkt-step-complementary-slackness}\\
        & \lambda_i^{k} + u_i^k \ge 0,                                                                  && \quad \text{if~$i \in \iub$.}
    \end{empheq}
\end{subequations}
Note that this conditions resemble closely the first-order Taylor approximation at~$(\iter[k], \lambda^k)$ of~\cref{eq:sqp-kkt}, in which case~$(d^k, u^k)$ would be a Newton-Raphson step.
However, the condition~\cref{eq:sqp-kkt-step-complementary-slackness} is not directly the linearization of~\cref{eq:sqp-kkt-complementary-slackness}, but it includes the second-order term~$u_i^k \nabla \con{i}(\iter[k])^{\T} d^k$.
We thus remark that the conditions~\cref{eq:sqp-kkt-step} are nothing but the \gls{kkt} conditions of the \gls{sqp} subproblem~\cref{eq:sqp-subproblem} associated with the Lagrange multiplier~$\lambda^k + u^k$.

\subsubsection{Approximation of a modified Lagrangian}

A second interpretation of the \gls{sqp} subproblem~\cref{eq:sqp-subproblem} is as follows.
Given an iterate~$\iter[k] \in \R^n$, let~$\lag[k]$ be the modified Lagrangian function
\begin{equation*}
    \lag[k](\iter, \lambda) \eqdef \obj(\iter) + \sum_{\mathclap{i \in \iub \cup \ieq}} \lambda_i \delta_i^k(\iter), \quad \text{for~$\iter \in \R^n$ and~$\lambda_i \in \R$, with~$i \in \iub \cup \ieq$},
\end{equation*}
where~$\lambda = [\lambda_i]_{i \in \iub \cup \ieq}^{\T}$, and where~$\delta_i^k$, for~$i \in \iub \cup \ieq$, denotes the departure from linearity\footnote{When~$\con{i}$ is strictly convex, it is the Bregman distance~\cite{Bregman_1967} associated with~$\con{i}$ for the point~$\iter[k]$.}~\cite{Robinson_1972,Hoek_1982} associated with~$\con{i}$ for the point~$\iter[k]$, defined by
\begin{equation*}
    \delta_i^k(\iter) \eqdef \con{i}(\iter) - \con{i}(\iter[k]) - \nabla \con{i}(\iter[k])^{\T} (\iter - \iter[k]), \quad \text{for~$\iter \in \R^n$.}
\end{equation*}
The \gls{sqp} subproblem~\cref{eq:sqp-subproblem} can then be seen as the minimization of the second-order Taylor approximation of~$\lag[k]$ subject to the linearized constraints, i.e.,
\begin{align}
    \min        & \quad \nabla_x \lag[k](\iter[k], \lambda^k)^{\T} \step + \frac{1}{2} \step^{\T} \nabla_{x, x}^2 \lag[k](\iter[k], \lambda^k) \step\\
    \text{s.t.} & \quad \con{i}(\iter[k]) + \nabla \con{i}(\iter[k])^{\T} \step \le 0, ~ i \in \iub,\\
                & \quad \con{i}(\iter[k]) + \nabla \con{i}(\iter[k])^{\T} \step = 0, ~ i \in \ieq,\\
                & \quad \step \in \R^n.
\end{align}

\subsubsection{Approximation of the augmented Lagrangian}

For clarity, we assume here without loss of generality that~$\ieq = \emptyset$.
The problem~\cref{eq:problem-cobyqa} can be reformulated by introducing a slack variable~$s = [s_i]_{i \in \iub}^{\T}$ as
\begin{align*}
    \min        & \quad \obj(\iter)\\
    \text{s.t.} & \quad \con{i}(\iter) + s_i = 0, ~ i \in \iub,\\
                & \quad s_i \ge 0, ~ i \in \iub,\\
                & \quad \iter \in \R^n.
\end{align*}
Let~$\lag[\mathsf{A}]$ be the augmented Lagrangian function of this problem, i.e.,
\begin{equation}
    \label{eq:augmented-lagrangian-inequality}
    \lag[\mathsf{A}](\iter, s, \lambda) \eqdef \obj(\iter) + \sum_{i \in \iub} \lambda_i [\con{i}(\iter) + s_i] + \frac{\gamma}{2} \sum_{i \in \iub} [\con{i}(\iter) + s_i]^2,
\end{equation}
for~$\iter \in \R^n$ and~$s_i, \lambda_i \in \R$ for~$i \in \iub$, where~$\gamma \ge 0$ is a given penalty parameter.
Given an iterate~$\iter[k] \in \R^n$, a slack variable~$s = [s_i]_{i \in \iub}^{\T}$, and a Lagrange multiplier~$\lambda^k = [\lambda_i^k]_{i \in \iub}^{\T}$, we approximate the augmented Lagrangian function~\cref{eq:augmented-lagrangian-inequality} by the following Taylor-like development.
We replace~$\obj$ by its second-order approximation, and~$\con{i}$ by its second-order approximation in the first sum and by its first-order approximation in the second sum, that is,
\begin{align*}
    \lag[\mathsf{A}](\iter[k] + \step, s^k, \lambda^k)  & \approx \obj(\iter[k]) + \nabla \obj(\iter[k])^{\T} \step + \frac{1}{2} \step^{\T} \nabla^2 \obj(\iter[k]) \step\\
                                                        & \qquad + \sum_{i \in \iub} \lambda_i^k \bigg[ \con{i}(\iter[k]) + \nabla \con{i}(\iter[k])^{\T} \step + \frac{1}{2} \step^{\T} \nabla^2 \con{i}(\iter[k]) \step + s_i \bigg]\\
                                                        & \qquad + \frac{\gamma}{2} \sum_{i \in \iub} [\con{i}(\iter[k]) + \nabla \con{i}(\iter[k])^{\T} \step + s_i]^2\\
                                                        & \approx \obj(\iter[k]) + \nabla \obj(\iter[k])^{\T} \step + \frac{1}{2} \step^{\T} \nabla^2 \lag(\iter[k], \lambda^k) \step\\
                                                        & \qquad + \sum_{i \in \iub} \lambda_i^k [\con{i}(\iter[k]) + \nabla \con{i}(\iter[k])^{\T} \step + s_i]\\
                                                        & \qquad + \frac{\gamma}{2} \sum_{i \in \iub} [\con{i}(\iter[k]) + \nabla \con{i}(\iter[k])^{\T} \step + s_i]^2,
\end{align*}
for~$\step \in \R^n$.
We thus remark the right-hand side of this equation is nothing but the augmented Lagrangian function of the \gls{sqp} subproblem~\cref{eq:sqp-subproblem}.

Moreover, the second-order Taylor approximation of~$\lag[\mathsf{A}]$ provides
\begin{align*}
    \lag[\mathsf{A}](\iter[k] + d, s^k, \lambda^k)   & \approx \obj(\iter[k]) + \nabla \obj(\iter[k])^{\T} d + \frac{1}{2} d^{\T} B_k(\Lambda^k) d\\
                                                & \qquad + \sum_{i \in \iub} \lambda_i^k [\con{i}(\iter[k]) + \nabla \con{i}(\iter[k])^{\T} d + s_i]\\
                                                & \qquad + \frac{\gamma}{2} \sum_{i \in \iub} [\con{i}(\iter[k]) + \nabla \con{i}(\iter[k])^{\T} d + s_i]^2\\
\end{align*}
where~$B_k$ is defined by
\begin{equation*}
    B_k(\lambda) \eqdef \nabla \obj(\iter[k]) + \sum_{i \in \iub} [\lambda_i + \gamma (\con{i}(\iter[k]) + s_i)] \nabla^2 \con{i}(\iter[k]), \quad \text{for~$\lambda_i \in \R$, with~$i \in \iub$.}
\end{equation*}
This approximation of~$\lag[\mathsf{A}]$ is the augmented Lagrangian of the \gls{sqp} subproblem~\cref{eq:sqp-subproblem} if the second-order term in~\cref{eq:sqp-subproblem-obj} included~$B_k(\lambda^k)$ instead of~$\nabla_{x, x}^2 \lag(\iter[k], \lambda^k)$.
Interestingly, a traditional augmented Lagrangian method for solving the problem~\cref{eq:problem-cobyqa} would update the dual variable~$\lambda^k$ as
\begin{equation*}
    \lambda_i^{k + 1} = \lambda_i^k + \gamma (\con{i}(\iter[k]) + s_i), \quad \text{for~$i \in \iub$}.
\end{equation*}
Therefore, this approximation of~$\lag[\mathsf{A}]$ can be interpreted as the augmented Lagrangian of the \gls{sqp} subproblem~\cref{eq:sqp-subproblem} if the second-order term in~\cref{eq:sqp-subproblem-obj} included the updated term~$\nabla_{x, x}^2 \lag(\iter[k], \lambda^{k + 1})$ instead of~$\nabla_{x, x}^2 \lag(\iter[k], \lambda^k)$.

\subsection{A derivative-free \glsfmtlong{sqp} method}

At the~$k$th iteration, \gls{cobyqa} builds the models~$\objm[k]$ and~$\conm[k]{i}$ of the objective function~$\obj$ and the constraints functions~$\con{i}$, with~$i \in \iub \cup \ieq$, using the derivative-free Broyden updates presented in~\cref{subsec:symmetric-broyden-updates}.
A poised interpolation set~$\xpt[k] \subseteq \R^n$ is maintained, whose cardinal number is fixed.
The only restriction is
\begin{equation*}
    n + 2 \le \card(\xpt[k]) \le \frac{1}{2} (n + 1) (n + 2),
\end{equation*}
as otherwise, the set~$\xpt[k]$ cannot be poised (see \cref{subsec:symmetric-broyden-updates} for details).
More details on this interpolation set is given in \cref{subsec:cobyqa-models}.
The derivative-free variation of the \gls{sqp} method we consider then has the subproblem
\begin{align*}
    \min        & \quad \nabla \objm[k](\iter[k])^{\T} \step + \frac{1}{2} \step^{\T} \nabla_{x, x}^2 \lagm[k](\iter[k], \lambda^k) \step\\
    \text{s.t.} & \quad \conm[k]{i}(\iter[k]) + \nabla \conm[k]{i}(\iter[k])^{\T} \step \le 0, ~ i \in \iub,\\
                & \quad \conm[k]{i}(\iter[k]) + \nabla \conm[k]{i}(\iter[k])^{\T} \step = 0, ~ i \in \ieq,\\
                & \quad \step \in \R^n,
\end{align*}
where~$\lagm[k]$ is defined by
\begin{equation*}
    \lagm[k](\iter, \lambda) \eqdef \objm[k](\iter) + \sum_{\mathclap{i \in \iub \cup \ieq}} \lambda_i \conm[k]{i}(\iter), \quad \text{for~$\iter \in \R^n$ and~$\lambda_i \in \R$, with~$i \in \iub \cup \ieq$}.
\end{equation*}

\section{The trust-region framework}

The \gls{sqp} framework suffers from one major deffect: it is a local method.
Therefore, the convergence of the method cannot be ensured for any initial guess~$\iter[0] \in \R^n$.
To cope with this difficulty, the method is usually embedded in a globalization strategy, such as a line-search or a trust-region framework.
We consider in this thesis trust-region methods only.
When combining all the information, the trust-region \gls{sqp} subproblem considered by \gls{cobyqa} is
\begin{subequations}
    \begin{align}
        \min        & \quad \nabla \objm[k](\iter[k])^{\T} \step + \frac{1}{2} \step^{\T} \nabla_{x, x}^2 \lagm[k](\iter[k], \lambda^k) \step \label{derivative-free-trust-region-sqp-subproblem-obj}\\
        \text{s.t.} & \quad \conm[k]{i}(\iter[k]) + \nabla \conm[k]{i}(\iter[k])^{\T} \step \le 0, ~ i \in \iub,\\
                    & \quad \conm[k]{i}(\iter[k]) + \nabla \conm[k]{i}(\iter[k])^{\T} \step = 0, ~ i \in \ieq,\\
                    & \quad \norm{\step} \le \rad[k],\\
                    & \quad \step \in \R^n,
    \end{align}
\end{subequations}
where~$\norm{\cdot}$ denotes the Euclidean norm,~$\rad[k] > 0$ is the current trust-region radius, and~$\lagm[k]$ denotes the Lagrangian function of the problem~\cref{eq:problem-cobyqa}.

A solution to this subproblem provides a trial step~$d^k \in \R^k$.
Further, this trial step is either accepted or not, i.e.,~$\iter[k + 1]$ is either set to~$\iter[k] + d^k$ or to~$\iter[k]$, depending on the value of the trust-region ratio.
Given a merit function~$\merit[k]$ and its modeled counterpart~$\meritm[k]$ (see \cref{subsec:cobyqa-merit-function}), the \emph{trust-region ratio}~$\ratio[k]$ is defined by
\begin{equation*}
    \ratio[k] \eqdef \frac{\merit[k](\iter[k]) - \merit[k](\iter[k] + d^k)}{\meritm[k](\iter[k]) - \meritm[k](\iter[k] + d^k)}.
\end{equation*}
The closer from one~$\ratio[k]$ is, the best the~$k$th models perform.

\subsection{Updating the trust-region radii}

\begin{itemize}
    \item We employ two trust-region radii.
\end{itemize}

\begin{algorithm}[ht]
    \caption{Updating the trust-region radius}
    \DontPrintSemicolon
    \KwData{Current lower bound on the trust-region radius~$\radlb[k] > 0$, current trust-region radius~$\rad[k] > 0$, current trust-region ratio~$\ratio[k] \in \R$, current trial step~$d^k \in \R^n$, and parameters~$0 < \eta_1 \le \eta_2 < 1 \le \eta_3$ and~$0 < \theta_1 < 1 < \theta_2$}
    Set the value of~$\rad[k + 1]$ to
    \begin{algoempheq}[left={\rad[k + 1] \gets \empheqlbrace}]{alignat*=2}
        & \theta_1 \rad[k]                                                                      && \quad \text{if~$\ratio[k] \le \eta_1$,}\\
        & \min \set{\theta_1 \rad[k], \norm{d^k}}                                               && \quad \text{if~$\eta_1 < \ratio[k] \le \eta_2$,}\\
        & \min \set{\theta_2 \rad[k], \max \set{\theta_1 \rad[k], \theta_1^{-1} \norm{d^k}}}    && \quad \text{otherwise}
    \end{algoempheq}
    \If{$\rad[k + 1] \le \eta_3 \radlb[k]$}{
        $\rad[k + 1] \gets \radlb[k]$\;
    }
\end{algorithm}

\begin{itemize}
    \item \Cref{alg:reducing-lower-bound-trust-region-radius} is entertained only at the end of a trust-region iteration, if~$\rad[k] = \radlb[k]$, the trial step~$\norm{\step[k]}$ is small compared with~$\rad[k]$, and the trust-region ratio~$\ratio[k]$ is small.
\end{itemize}

\begin{algorithm}[ht]
    \caption{Reducing the lower bound on the trust-region radius}
    \label{alg:reducing-lower-bound-trust-region-radius}
    \DontPrintSemicolon
    \KwData{Final trust-region radius~$\radlb[\infty] > 0$, current lower bound on the trust-region radius~$\radlb[k] \ge \radlb[\infty]$}
    \If{$\radlb[k] = \radlb[\infty]$}{
        Terminate the optimization method\;
    }
    Set the value of~$\radlb[k + 1]$ to
    \begin{algoempheq}[left={\radlb[k + 1] \gets \empheqlbrace}]{alignat*=2}
        & \theta_3 \radlb[k]                && \quad \text{if~$\eta_5 < \radlb[k] / \radlb[\infty]$,}\\
        & \sqrt{\radlb[k] \radlb[\infty]}   && \quad \text{if~$\eta_4 < \radlb[k] / \radlb[\infty] \le \eta_5$,}\\
        & \radlb[\infty]                    && \quad \text{otherwise}
    \end{algoempheq}
    $\rad[k + 1] \gets \max \set{\rad[k + 1], \radlb[k + 1]}$\;
\end{algorithm}

\subsection{Merit functions and penalty coefficients}
\label{subsec:cobyqa-merit-function}

\begin{itemize}
    \item What is a merit function?
    \item Examples of merit functions (smooth, exact nonsmooth, augmented Lagrangian).
    \item We choose the~$\ell_2$-merit function (exact nonsmooth).
    \item The modeled merit function we choose is not directly the Taylor approximation.
    \item The modeled merit function must ensure a crucial decrease property (holds because of what follows).
    \item The best point so far considered by \gls{cobyqa} is the one with the least merit value.
    \item What is the trust-region ratio?
    \item \Cite[thm.~14.5.1]{Conn_Gould_Toint_2000} provides a lower bound on the penalty coefficient.
    \item Detail the algorithm for increasing the penalty parameter.
    \item We decrease the penalty parameter as in~\cite{Powell_1994} (detail the modification made for the equality constraints).
\end{itemize}

\subsection{Composite-step approach}

\section{Outline of the \glsfmttext{cobyqa} method}

\subsection{Interpolation-based quadratic models}
\label{subsec:cobyqa-models}

\begin{itemize}
    \item What is the initial interpolation set?
    \item How is this interpolation set updated?
    \item Approximations of linear function are exact (theorem).
\end{itemize}

\subsection{Geometry of the interpolation set}

\subsection{Estimation of the Lagrange multipliers}

\subsection{Maratos effect and \glsfmtlong{soc}}

\subsection{Summary of the method}
