%% contents/cobyqa-introduction.tex
%% Copyright 2021-2022 Tom M. Ragonneau
%
% This work may be distributed and/or modified under the
% conditions of the LaTeX Project Public License, either version 1.3
% of this license or (at your option) any later version.
% The latest version of this license is in
%   http://www.latex-project.org/lppl.txt
% and version 1.3 or later is part of all distributions of LaTeX
% version 2005/12/01 or later.
%
% This work has the LPPL maintenance status `maintained'.
%
% The Current Maintainer of this work is Tom M. Ragonneau.
\chapter{\glsfmttext{cobyqa} \textemdash\ a new \glsfmtlong{dfo} method}
\label{ch:cobyqa-introduction}

\section{Statement of the problem}

In this chapter, we introduce a new model-based \gls{dfo} method for solving nonlinearly-constrained problems of the form
\begin{subequations}
    \label{eq:problem-cobyqa}
    \begin{align}
        \min        & \quad \obj(\iter) \label{eq:problem-cobyqa-obj}\\
        \text{s.t.} & \quad \con{i}(\iter) \le 0, ~ i \in \iub, \label{eq:problem-cobyqa-ub}\\
                    & \quad \con{i}(\iter) = 0, ~ i \in \ieq, \label{eq:problem-cobyqa-eq}\\
                    & \quad \xl \le x \le \xu, \label{eq:problem-cobyqa-bd}\\
                    & \quad \iter \in \R^n, \nonumber
    \end{align}
\end{subequations}
where~$\obj$ and~$\con{i}$ represent the objective and constraint functions, with~$i \in \iub \cup \ieq$ and the sets of indices~$\iub$ and~$\ieq$ being finite and disjoint, but possibly empty. 
The lower bounds~$\xl \in (\R \cup \set{-\infty})^n$ and the upper bounds~$\xu \in (\R \cup \set{\infty})^n$ satisfy~$\xl < \xu$.
Note that the bound constraints~\cref{eq:problem-cobyqa-bd} are not included in the inequality constraints~\cref{eq:problem-cobyqa-ub}, because they will be handled separately, as detailed in \cref{sec:simple-constraints}.

We will develop a derivative-free trust-region \gls{sqp} method for the problem~\cref{eq:problem-cobyqa}.
The method, named~\gls{cobyqa} after \emph{\glsdesc{cobyqa}}, does not use derivatives of the objective function or the nonlinear constraint functions, but models them using underdetermined interpolation based on the derivative-free symmetric Broyden update (see \cref{subsec:symmetric-broyden-updates}).
This chapter presents the framework of the method, while the subproblems and the Python implementation will be discussed in \cref{ch:cobyqa-subproblems,ch:cobyqa-implementation}, respectively.

\section{The derivative-free trust-region \glsfmtshort{sqp} method}

We presented in \cref{ch:sqp} the basic trust-region \gls{sqp} method.
We now adapt this framework to derivative-free settings.

\subsection{Interpolation-based quadratic models}
\label{subsec:interpolation-based-quadratic-models}

Recall that the basic trust-region \gls{sqp} method presented in \cref{alg:trust-region-sqp} uses the gradients and the Hessian matrices of the objective function~$\obj$ and the constraint functions~$\con{i}$, with~$i \in \iub \cup \ieq$.
Since we do not have access to such information, we use quadratic models of them, for which all derivative information are known.
As detailed in \cref{ch:interpolation}, to increase the performance of the method, we use quadratic models obtained by underdetermined interpolation.

At the~$k$th iteration, we note by~$\objm[k]$ the quadratic model of~$\obj$, and by~$\conm[k]{i}$ the quadratic model of~$\con{i}$, for all~$i \in \iub \cup \ieq$.
These models are built on an interpolation set~$\xpt[k] \subseteq \R^n$, which is maintained automatically by the method.
It changes at most one point of~$\xpt[k]$ at each iteration, and ensures that~$\iter[k] \in \xpt[k]$, where~$\iter[k] \in \R^n$ denotes the~$k$th iterate.
Inspired by the performances of the solvers \gls{newuoa}, \gls{bobyqa}, and \gls{lincoa} of Powell, we decide to base our models~$\objm[k]$ and~$\conm[k]{i}$, for~$i \in \iub \cup \ieq$, on the derivative-free symmetric Broyden update (see \cref{subsec:symmetric-broyden-updates}).

The initial models~$\objm[0]$ and~$\conm[0]{i}$, for~$i \in \iub \cup \ieq$, are built on the initial interpolation set~$\xpt[0] \subseteq \R^n$, defined as follows and proposed by \citeauthor{Powell_2009}~\cite{Powell_2009}.
We are given a number~$m$ of interpolation points, satisfying
\begin{equation*}
    n + 2 \le m \le \frac{1}{2} (n + 1) (n + 2),
\end{equation*}
provided by the user, together with an initial guess~$\iter[0] \in \R^n$ satisfying~$\xl \le \iter[0] \le \xu$ and an initial trust-region radius~$\rad[0] > 0$.
The first~$n + 1$ points of~$\xpt[0] = \set{y^1, y^2, \dots, y^m}$ are defined by
\begin{empheq}[left={y^i \eqdef \empheqlbrace}]{alignat*=2}
    & \iter[0]                      && \quad \text{if~$i = 1$,}\\
    & \iter[0] + \rad[0] e_{i - 1}  && \quad \text{if~$2 \le i \le n + 1$ and~$\xl_{i - 1} \le \iter[0]_{i - 1} < \xu_{i - 1}$,}\\
    & \iter[0] - \rad[0] e_{i - 1}  && \quad \text{if~$2 \le i \le n + 1$ and~$\iter[0]_{i - 1} = \xu_{i - 1}$,}
\end{empheq}
and the following~$\min \set{n, m - n - 1}$ are defined by
\begin{empheq}[left={y^i \eqdef \empheqlbrace}]{alignat*=2}
    & \iter[0] - \rad[0] e_{i - n - 1}  && \quad \text{if~$n + 2 \le i \le \min \set{2n + 1, m}$ and~$\xl_{i - n - 1} < \iter[0]_{i - n - 1} \le \xu_{i - n - 1}$,}\\
    & \iter[0] + 2\rad[0] e_{i - n - 1} && \quad \text{if~$n + 2 \le i \le \min \set{2n + 1, m}$ and~$\iter[0]_{i - n - 1} = \xl_{i - n - 1}$,}\\
    & \iter[0] - 2\rad[0] e_{i - n - 1} && \quad \text{if~$n + 2 \le i \le \min \set{2n + 1, m}$ and~$\iter[0]_{i - n - 1} = \xu_{i - n - 1}$,}
\end{empheq}
where~$e_i$ denotes the~$i$th standard coordinate vector of~$\R^n$, i.e., the~$i$th column of~$I_n$.
If~$m > 2n + 1$, for~$i \in \set{2n + 2, 2n + 3, \dots, m}$, we set
\begin{equation*}
    y^i \eqdef y^{p(i) + 1} + y^{q(i) + 1} - \iter[0],
\end{equation*}
where~$p$ and~$q$ are defined by
\begin{equation*}
    p(i) \eqdef i - n - 1 - n \kappa(i) \quad \text{with} \quad \kappa(i) \eqdef \floor[\bigg]{\frac{i - n - 2}{n}},
\end{equation*}
and
\begin{empheq}[left={q(i) \eqdef \empheqlbrace}]{alignat*=2}
    & p(i) + \kappa(i)      && \quad \text{if~$p(i) + \kappa(i) \le n$,}\\
    & p(i) + \kappa(i) - n  && \quad \text{otherwise.}
\end{empheq}
We point out that if~$m \le 2n + 1$ and~$\xl < \iter[0] < \xl$, then the interpolation set~$\xpt[0]$ is exactly a scaled variation of the interpolation set studied in \cref{sec:optimal-interpolation-set}.
We showed to some extend that this set was optimal for~$m = 2n + 1$.
Therefore, as Powell did, we encourage such a value for~$m$.

An advantage of choosing such an initial interpolation set is that the coefficients of the minimum-norm quadratic interpolation model can be directly evaluated.
See~\cite[\S~9]{Powell_2009} for the details calculations.

\subsection{A derivative-free trust-region \glsfmtshort{sqp} framework}

We present in this section the derivative-free trust-region \gls{sqp} framework employed by \gls{cobyqa}.
As before, the merit function we consider is the~$\ell_2$-merit function, defined for a given penalty parameter~$\gamma^k \ge 0$ by
\begin{equation*}
    \merit[k](\iter) \eqdef \obj(x) + \gamma^k \sqrt{\sum_{i \in \iub} \posp{\con{i}(\iter)}^2 + \sum_{i \in \ieq} \abs{\con{i}(\iter)}^2}, \quad \text{for~$\iter \in \R^n$.}
\end{equation*}
Once again, we denote by~$\meritm[k]$ the~$\ell_2$-merit function computed on the \gls{sqp} subproblem, i.e.,
\begin{align*}
    \meritm[k](\step)   & \eqdef \nabla \obj(\iter[k])^{\T} \step + \frac{1}{2} \step^{\T} \nabla_{x, x}^2 \lag(\iter[k], \lm[k]) \step\\
                        & \qquad + \gamma^k \sqrt{\sum_{i \in \iub} \posp{\con{i}(\iter[k]) + \nabla \con{i}(\iter[k])^{\T} \step}^2 + \sum_{i \in \ieq} [\con{i}(\iter[k]) + \nabla \con{i}(\iter[k])^{\T} \step]^2}, \quad \text{for~$\step \in \R^n$.}
\end{align*}
The framework is given in \cref{alg:derivative-free-trust-region-sqp}, which hides many details, including the definition of \enquote{convergence} in the loop or the update of~$\lm[k]$, which will be specified hereinafter.

\begin{algorithm}
    \caption{Derivative-free trust-region \glsfmtshort{sqp} method}
    \label{alg:derivative-free-trust-region-sqp}
    \DontPrintSemicolon
    \KwData{Initial guess~$\iter[0] \in \R^n$, estimated Lagrange multiplier~$\lm[0] = [\lm[0]_i]_{i \in \iub \cup \ieq}^{\T}$, initial trust-region radius~$\rad[0] > 0$, and parameters~$0 < \eta_1 \le \eta_2 < 1$ and~$0 < \theta_1 < 1 < \theta_2$.}
    Set the penalty parameter~$\gamma^{-1} \gets  0$\;
    Build the initial interpolation set~$\xpt[0] \subseteq \R^n$ described in \cref{subsec:interpolation-based-quadratic-models}\;
    \For{$k = 0, 1, \dots$ until convergence}{
        Evaluate~$\objm[k]$ and~$\conm[k]{i}$ for~$i \in \iub \cup \ieq$ by underdetermined interpolation on~$\xpt[k]$\; \nllabel{alg:derivative-free-trust-region-sqp-models}
        Set the trial step~$\step[k] \in \R^n$ to an approximate solution to
        \begin{subequations}
            \label{eq:derivative-free-trust-region-sqp-subproblem}
            \begin{algomathalign}
                \min        & \quad \nabla \objm[k](\iter[k])^{\T} \step + \frac{1}{2} \step^{\T} \nabla_{x, x}^2 \lagm[k](\iter[k], \lm[k]) \step\\
                \text{s.t.} & \quad \conm[k]{i}(\iter[k]) + \nabla \conm[k]{i}(\iter[k])^{\T} \step \le 0, ~ i \in \iub,\\
                            & \quad \conm[k]{i}(\iter[k]) + \nabla \conm[k]{i}(\iter[k])^{\T} \step = 0, ~ i \in \ieq,\\
                            & \quad \norm{\step} \le \rad[k],\\
                            & \quad \step \in \R^n, \nonumber
            \end{algomathalign}
        \end{subequations}
        Set a penalty parameter~$\gamma^k \ge \max \set{\gamma^{k - 1}, \norm{\lm[k]}}$ providing~$\meritm[k](\step[k]) < \meritm[k](0)$ \nllabel{alg:derivative-free-trust-region-sqp-penalty}\;
        Evaluate the trust-region ratio
        \begin{algomathdisplay}
            \ratio[k] \gets \frac{\merit[k](\iter[k]) - \merit[k](\iter[k] + \step[k])}{\meritm[k](0) - \meritm[k](\step[k])}
        \end{algomathdisplay}
        \eIf{$\ratio[k] \ge 0$}{
            Choose a point~$\bar{y} \in \xpt[k]$ to remove from~$\xpt[k]$\;
        }{
            Choose a point~$\bar{y} \in \xpt[k] \setminus \set{\iter[k]}$ to remove from~$\xpt[k]$\;
        }
        Update the interpolation set~$\xpt[k + 1] \gets (\xpt[k] \setminus \set{\bar{y}}) \cup \set{\iter[k] + \step[k]}$\;
        Update the current iterate~$\iter[k + 1]$ to a solution to~$\min_{y \in \xpt[k + 1]} \merit[k](y)$\;
        Estimate the Lagrange multiplier~$\lm[k + 1] = [\lm[k + 1]_i]_{i \in \iub \cup \ieq}$\; \nllabel{alg:derivative-free-trust-region-sqp-multipliers}
        Update the trust-region radius
        \begin{algoempheq}[left={\rad[k + 1] \gets \empheqlbrace}]{alignat*=2}
            & \theta_1 \rad[k],  && \quad \text{if~$\ratio[k] \le \eta_1$,}\\
            & \rad[k],           && \quad \text{if~$\eta_1 < \ratio[k] \le \eta_2$,}\\
            & \theta_2 \rad[k],  && \quad \text{otherwise}
        \end{algoempheq} \label{alg:derivative-free-trust-region-sqp-radius}
    }
\end{algorithm}

If we were naively adapting the trust-region framework for unconstrained optimization to our setting, the choice of~$\iter[k + 1]$ would be either~$\iter[k] + \step[k]$ if~$\ratio[k] \ge 0$, and~$\iter[k]$ otherwise.
However, this would not take into account the fact that~$\merit[k]$ might differ from~$\merit[k - 1]$.
Therefore,~$\iter[k + 1]$ might not be the best point in~$\xpt[k + 1]$ according to~$\merit[k]$.
In \cref{alg:derivative-free-trust-region-sqp} however,~$\iter[k + 1]$ is always the best point in~$\xpt[k + 1]$.
Moreover, note that~$\xpt[k + 1]$ necessarily contains~$\iter[k] + \step[k]$, and also contain~$\iter[k]$ if~$\ratio[k] < 0$.

In a practical implementation, the quadratic models on \cref{alg:derivative-free-trust-region-sqp-models} of \cref{alg:derivative-free-trust-region-sqp} are not computed from scratch, because only one point differs~$\xpt[k + 1]$ from~$\xpt[k]$.
Details on this update mechanism are given in \cref{subsec:implementation-symmetric-broyden-update}.

To solve the trust-region \gls{sqp} subproblem~\cref{eq:derivative-free-trust-region-sqp-subproblem}, we employ a Byrd-Omojokun approach.
It first generates a normal step~$\nstep[k]$ by solving approximately
\begin{align*}
    \min        & \quad \sum_{i \in \iub} \posp{\con{i}(\iter[k]) + \nabla \con{i}(\iter[k])^{\T} \step}^2 + \sum_{i \in \ieq} [\con{i}(\iter[k]) + \nabla \con{i}(\iter[k])^{\T} \step]^2\\
    \text{s.t.} & \quad \xl \le \iter[k] + \step \le \xu,\\
                & \quad \norm{\step} \le \zeta \rad[k],\\
                & \quad \step \in \R^n,
\end{align*}
\todo[noline]{Finish the explanation}

% Theoretically, it is possible to evaluate the composite step directly.
% However, the methods used to solve the subproblems usually require the trust-region center to be feasible, which is not the case for the composite subproblem.
% Therefore, we modify the tangential subproblem to make the origin the trust-region center, which is then feasible.
% Therefore, we build explicitely the tangential step.
% See Trust-Region Methods, p.~660.

\subsection{Least-squares Lagrange multipliers}

We now provide details on the strategy employed by \gls{cobyqa} to estimate the Lagrange multipliers on \cref{alg:derivative-free-trust-region-sqp-multipliers} of \cref{alg:derivative-free-trust-region-sqp}.

Recall that the \gls{sqp} method can be regarded as an approximate Newton method on the \gls{kkt} system.
Therefore, given the iterate~$\iter[k + 1]$, it is natural to set the estimated Lagrange multipliers~$\lm[k + 1]$ to the one that attempt to satisfy the \gls{kkt} conditions as much as possible using the information available so far.
We then let~$\lm[k + 1]$ be the least-norm solution to
\begin{subequations}
    \begin{align}
        \min        & \quad \norm[\bigg]{\nabla \objm[k](\iter[k + 1]) + \sum_{i \in \iub \cup \ieq} \lm_i \nabla \conm[k]{i}(\iter[k + 1])}\\
        \text{s.t.} & \quad \lm_i \conm[k]{i}(\iter[k + 1]) = 0, ~ i \in \iub \label{eq:least-squares-lagrange-multipliers-complementary-slackness}\\
                    & \quad \lm_i \ge 0, ~ i \in \iub,\\
                    & \quad \lm = [\lm_i]_{i \in \iub \cup \ieq},
    \end{align}
\end{subequations}
which is then referred to as the \emph{least-squares Lagrange multiplier}~\cite{Dussault_1995}.
Note that this problem can be easily simplified as follows.
For all the indices~$i \in \iub$ such that~$\conm[k]{i}(\iter[k + 1]) \neq 0$, we necessarily have~$\lm[k + 1]_i = 0$.
Therefore, the linear constraints~\cref{eq:least-squares-lagrange-multipliers-complementary-slackness} can be removed from the problem by considering only the indices in
\begin{equation*}
    \ieq \cup \set{i \in \iub : \conm[k]{i}(\iter[k + 1]) = 0},
\end{equation*}
and by setting the remaining components of the least-squares Lagrange multiplier to zero.
% Another possibility, leading to a slightly different problem, is to consider only the indices in
% \begin{equation*}
%     \ieq \cup \set{i \in \iub : \conm[k]{i}(\iter[k + 1]) \le 0},
% \end{equation*}
% and to set the remaining components of the least-squares Lagrange multiplier to zero.
This is the method employed by \gls{cobyqa}.

\subsection{Geometry of the interpolation set}

As we mentioned in \cref{ch:interpolation}, the interpolation set~$\xpt[k]$ is adequate only if it is poised, i.e., only if the interpolation problem admits a solution for any function that is interpolated.
Moreover, we may wish that~$\xpt[k]$ is~$\Lambda$-poised in
\begin{equation*}
    \set{\iter \in \R^n : \norm{x - \iter[k]} \le \radlb[k]},
\end{equation*}
for some reasonably low~$\Lambda$ (see \cref{subsec:symmetric-broyden-updates}).
However, \cref{alg:derivative-free-trust-region-sqp} never ensures that such properties hold.
It is known that model-based methods tends in practice to lose the poisedness property of their interpolation set as the iterations progress.
To cope with this difficulty, model-based \gls{dfo} methods are usually modified to include geometry-improving mechanisms~\cite{Conn_Scheinberg_Vicente_2008a,Conn_Scheinberg_Vicente_2008b,Fasano_Morales_Nocedal_2009}.

We present in what follows the geometry-improving mechanism employed by \gls{cobyqa}, which is adapted from \gls{bobyqa}~\cite{Powell_2009}.
If the step~$\step[k]$

\subsection{Managing the trust-region radius}

Inspired by the performance of \gls{uobyqa}, \gls{newuoa}, \gls{bobyqa}, and \gls{lincoa} (see \cref{subsec:uobyqa,subsec:newuoa-bobyqa-lincoa}), we employ the following paradigm for managing the trust-region radius in \gls{cobyqa}, proposed by Powell~\cite{Powell_2002,Powell_2006,Powell_2009}, in place of \cref{alg:derivative-free-trust-region-sqp-radius} or \cref{alg:derivative-free-trust-region-sqp}.
It consists in maintaining both the trust-region radius~$\rad[k] > 0$ and a lower-bound of it~$\radlb[k] > 0$.
The idea behind this technique is to use~$\rad[k]$ as the trust-region radius in the trust-region subproblem~\cref{eq:trust-region-sqp-subproblem}, and~$\radlb[k]$ to maintain an adequate distance between the points in~$\xpt[k]$.
Further, the method never increases~$\radlb[k]$, but adapt the value of~$\rad[k]$ in a typical trust-region way.
Of course, the method always ensures that~$\rad[k] \ge \radlb[k]$.
As noted by \citeauthor{Powell_2002}, allowing trial step to have a length larger than~$\radlb[k]$ prevent loss in efficiency that occurred otherwise in his software \gls{uobyqa}.

The update of the trust-region radius~$\rad[k]$ is given in \cref{alg:update-trust-region-radius}.
The parameters chosen in \gls{cobyqa} are~$\eta_1 = 0.1$,~$\eta_2 = 0.7$,~$\eta_3 = 1.4$,~$\theta_1 = 0.5$, and~$\theta_2 = \sqrt{2}$.
This update is entertained at each iteration.

\begin{algorithm}
    \caption{Updating the trust-region radius}
    \label{alg:update-trust-region-radius}
    \DontPrintSemicolon
    \KwData{Current lower bound on the trust-region radius~$\radlb[k] > 0$, current trust-region radius~$\rad[k] \ge \radlb[k]$, current trust-region ratio~$\ratio[k] \in \R$, current trial step~$\step \in \R^n$, and parameters~$0 < \eta_1 \le \eta_2 < 1 \le \eta_3$ and~$0 < \theta_1 < 1 < \theta_2$.}
    \KwResult{Updated trust-region radius~$\rad[k + 1]$.}
    Update the trust-region radius
    \begin{algoempheq}[left={\rad[k + 1] \gets \empheqlbrace}]{alignat*=2}
        & \theta_1 \rad[k]                                                                      && \quad \text{if~$\ratio[k] \le \eta_1$,}\\
        & \min \set{\theta_1 \rad[k], \norm{\step}}                                             && \quad \text{if~$\eta_1 < \ratio[k] \le \eta_2$,}\\
        & \min \set{\theta_2 \rad[k], \max \set{\theta_1 \rad[k], \theta_1^{-1} \norm{\step}}}  && \quad \text{otherwise}
    \end{algoempheq}
    \If{$\rad[k + 1] \le \eta_3 \radlb[k]$}{
        $\rad[k + 1] \gets \radlb[k]$\;
    }
\end{algorithm}

As we mentioned already, the method maintains~$\rad[k] \ge \radlb[k]$ and it never increases~$\radlb[k]$.
Therefore, the value of~$\radlb[k]$ is decreased only if~$\rad[k] = \radlb[k]$.
Moreover, since~$\radlb[k]$ is designed to maintain a good distance between the interpolation points, it must be decreased only if the performance of the models is poor.
Hence, it is decreased only if the trial step~$\norm{\step[k]}$ is small compared with~$\rad[k]$ and the trust-region ratio~$\ratio[k]$ is small.
\Cref{alg:reducing-lower-bound-trust-region-radius} presents the method employed by \gls{cobyqa} to reduce~$\radlb[k]$.
The parameters chosen in \gls{cobyqa} are~$\eta_4 = 16$,~$\eta_5 = 250$, and~$\theta_3 = 0.1$.

\begin{algorithm}
    \caption{Reducing the lower bound on the trust-region radius}
    \label{alg:reducing-lower-bound-trust-region-radius}
    \DontPrintSemicolon
    \KwData{Final trust-region radius~$\radlb[\infty] > 0$, current lower bound on the trust-region radius~$\radlb[k] \ge \radlb[\infty]$, updated trust-region radius~$\rad[k + 1] \ge \radlb[k]$, and parameters~$1 \le \eta_4 < \eta_5$ and~$0 < \theta_3 < 1$.}
    \KwResult{Reduced lower bound on trust-region radius~$\radlb[k + 1]$ and modified trust-region radius~$\rad[k + 1]$.}
    \If{$\radlb[k] = \radlb[\infty]$}{
        Terminate the optimization method\; \nllabel{alg:reducing-lower-bound-trust-region-radius-stop}
    }
    Update the lower bound on the trust-region radius
    \begin{algoempheq}[left={\radlb[k + 1] \gets \empheqlbrace}]{alignat*=2}
        & \theta_3 \radlb[k]                && \quad \text{if~$\eta_5 < \radlb[k] / \radlb[\infty]$,}\\
        & \sqrt{\radlb[k] \radlb[\infty]}   && \quad \text{if~$\eta_4 < \radlb[k] / \radlb[\infty] \le \eta_5$,}\\
        & \radlb[\infty]                    && \quad \text{otherwise}
    \end{algoempheq}
    Update the trust-region radius~$\rad[k + 1] \gets \max \set{\rad[k + 1], \radlb[k + 1]}$\;
\end{algorithm}

If \cref{alg:reducing-lower-bound-trust-region-radius-stop} of \cref{alg:reducing-lower-bound-trust-region-radius} is reached, we consider that the optimization method should stop, and that the method is successful.
This is because~$\radlb[k]$ measures to some extend the distance between the interpolation points.

\section{Management of bound and linear constraints}
\label{sec:simple-constraints}

The implementation of \gls{cobyqa} accepts three types of constraints, namely bound constraints, linear constraints, and nonlinear constraints.
From a theoretical standpoint, problems written in the form
\begin{align*}
    \min        & \quad \obj(\iter)\\
    \text{s.t.} & \quad \con{i}(\iter) \le 0, ~ i \in \iub,\\
                & \quad \con{i}(\iter) = 0, ~ i \in \ieq,\\
                & \quad \iter \in \R^n,
\end{align*}
may have bound and linear constraints included in the constraints~$\set{\con{i}}_{i \in \iub \cup \ieq}$.
However, our implementation handles these types of constraints separately, for the following reasons.

% First of all, note that in the general form of nonlinearly-constrained problems~\cref{eq:problem-cobyqa}, we did not include the bound constraints~\cref{eq:problem-cobyqa-bd} in the inequality constraints~\cref{eq:problem-cobyqa-ub}.
First of all, bound constraints often represent inalienable physical or theoretical restrictions.
In many applications for which \gls{cobyqa} is designed, the objective function~\cref{eq:problem-cobyqa-obj} is not defined if the bounds~\cref{eq:problem-cobyqa-bd} are violated.
% For instance, the tuning of nonlinear optimization methods (see \cref{subsec:tuning-nonlinear-optimization-methods}) involves bounds that cannot be violated, as the optimization methods that are tuned may not be defined otherwise.
For an example, see the hyperparameter tuning problem given in \cref{subsec:machine-learning}.
For this reason, every point that \gls{cobyqa} encounters always respects these bounds, as is also the case for the \gls{bobyqa} method, presented in \cref{subsec:newuoa-bobyqa-lincoa}.
% When establishing the problem~\cref{eq:problem-cobyqa}, we assumed that~$\xl < \xu$.
% Note that this requirement is weak, as otherwise, the problem~\cref{eq:problem-cobyqa} would be either infeasible, or admit fix variables.
% Thus, note also that they are very simple constraints.
% It is, for example, trivial to check whether a point is feasible with respect to the bound constraints~\cref{eq:problem-cobyqa-bd}, and easy to project any point onto the bound constraints.
Therefore, \gls{cobyqa} handles bound constraints separately.
% This is another reason why \gls{cobyqa} handles them separately.

The linear constraints are usually much less restrictive.
In applications, the objective function is often well-defined even at points that are infeasible with respect to the linear constraints.
Therefore, we do not enforce \gls{cobyqa} to always respect the linear constraints.
However, when evaluating a model~$\conm[k]{i}$ of a linear constraint~$\con{i}$, we enforce~$\conm[k]{i} \equiv \con{i}$.
This reduces the computational complexity of evaluating all the models, and it also suppresses all damages that could be generated by computer rounding errors.

\section{Merit function and update of the penalty parameters}

\section{Summary of the method}
