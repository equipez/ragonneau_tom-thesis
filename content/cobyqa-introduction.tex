%% contents/cobyqa-introduction.tex
%% Copyright 2021-2022 Tom M. Ragonneau
%
% This work may be distributed and/or modified under the
% conditions of the LaTeX Project Public License, either version 1.3
% of this license or (at your option) any later version.
% The latest version of this license is in
%   http://www.latex-project.org/lppl.txt
% and version 1.3 or later is part of all distributions of LaTeX
% version 2005/12/01 or later.
%
% This work has the LPPL maintenance status `maintained'.
%
% The Current Maintainer of this work is Tom M. Ragonneau.
\chapter{\glsfmttext{cobyqa} \textemdash\ a constrained \glsfmtlong{dfo} method}
\label{ch:cobyqa-introduction}

\section{Statement of the problem}

In this chapter, we introduce a new model-based \gls{dfo} method for solving nonlinearly-constrained problems of the form
\begin{subequations}
    \label{eq:problem-cobyqa}
    \begin{align}
        \min        & \quad \obj(x)\\
        \text{s.t.} & \quad \con{i}(x) \le 0, ~ i \in \iub, \label{eq:problem-cobyqa-ub}\\
                    & \quad \con{i}(x) = 0, ~ i \in \ieq,\\
                    & \quad \xl \le x \le \xu, \label{eq:problem-cobyqa-bd}\\
                    & \quad x \in \R^n, \nonumber
    \end{align}
\end{subequations}
where the objective and constraint functions~$\obj$ and~$\con{i}$, with~$i \in \iub \cup \ieq$, are real-valued functions on~$\R^n$, with the sets of indices~$\iub$ and~$\ieq$ being finite (perhaps empty) and disjoint, and where the bounds~$\xl \in (\R \cup \set{-\infty})^n$ and~$\xu \in (\R \cup \set{\infty})^n$ satisfy~$\xl < \xu$.
The requirement on the bounds is weak, as otherwise, the problem~\cref{eq:problem-cobyqa} would be either infeasible, or would admit a fix variables.
The solver we develop, named~\gls{cobyqa} after \emph{\glsdesc{cobyqa}}, uses only function values of~$\obj$ and~$\con{i}$, with~$i \in \iub \cup \ieq$, but not derivatives.

\section{Management of the bound constraints}

\begin{itemize}
    \item Theoretically, they could be included in~\cref{eq:problem-cobyqa-ub}.
    \item They are very simple constraints: it is trivial to check whether a point is feasible with respect to the bound constraints~\cref{eq:problem-cobyqa-bd}, it is easy to project any point onto the bound constraints, etc.
    \item \gls{cobyqa} nevers violates the bound constraints~\cref{eq:problem-cobyqa-bd}, as they often represent inalienable physical or theoretical restrictions.
\end{itemize}

\section{The \glsfmtlong{sqp} method}

For sake of clarity we assume in this section that no bound constraint is provided.
Therefore, the problem we consider is of the form
\begin{subequations}
    \label{eq:problem-cobyqa-no-bounds}
    \begin{align}
        \min        & \quad \obj(x)\\
        \text{s.t.} & \quad \con{i}(x) \le 0, ~ i \in \iub,\\
                    & \quad \con{i}(x) = 0, ~ i \in \ieq,\\
                    & \quad x \in \R^n. \nonumber
    \end{align}
\end{subequations}
As we mentioned earlier, if bound constraints are supplied, they can be included in the inequality constraints~\cref{eq:problem-cobyqa-ub} for theoretical purposes.
The Lagrangian function that we consider is defined by
\begin{equation*}
    \lag(x, \lambda) \eqdef \obj(x) + \sum_{\mathclap{i \in \iub \cup \ieq}} \lambda_i \con{i}(x), \quad \text{for~$x \in \R^n$ and~$\lambda_i \in \R$ for~$i \in \iub \cup \ieq$},
\end{equation*}
where~$\lambda = [\lambda_i]_{i \in \iub \cup \ieq}^{\T}$.

\subsection{Overview of the method}

The \gls{sqp} method of \citeauthor{Wilson_1963}~\cite{Wilson_1963}, \citeauthor{Han_1976}~\cite{Han_1976,Han_1977}, and \citeauthor{Powell_1978a}~\cite{Powell_1978a,Powell_1978b} is known to be one of the most powerful method for solving the problem~\cref{eq:problem-cobyqa-no-bounds} when derivatives of~$\obj$ and~$\con{i}$, with~$i \in \iub \cup \ieq$, are available.
Given an iterate~$x^k \in \R^n$, it generates a step~$d^k \in \R^n$ by solving approximately
\begin{subequations}
    \label{eq:sqp-subproblem}
    \begin{align}
        \min        & \quad \inner{\nabla \obj(x^k), d} + \frac{1}{2} \inner{d, \nabla_{x, x}^2 \lag(x^k, \lambda^k) d} \label{eq:sqp-subproblem-obj}\\
        \text{s.t.} & \quad \con{i}(x^k) + \inner{\nabla \con{i}(x^k), d} \le 0, ~ i \in \iub,\\
                    & \quad \con{i}(x^k) + \inner{\nabla \con{i}(x^k), d} = 0, ~ i \in \ieq,\\
                    & \quad d \in \R^n,
    \end{align}
\end{subequations}
for some Lagrange multiplier~$\lambda^k = [\lambda_i^k]_{i \in \iub \cup \ieq}^{\T}$ with~$\lambda_i^k \in \R$ for~$i \in \iub \cup \ieq$.
Further, it sets the next iterate~$x^{k + 1}$ to~$x^k + d^k$.
If the second-order derivative of~$\lag$ with respect to the decision variables is unavailable, the term~$\nabla^2 \lag_{x, x}(x^k, \lambda^k)$ can be replace with an approximation of it (e.g., a quasi-Newton approximation).
Under some mild assumptions on the regularity of the objective and constraint functions, the \gls{sqp} method is locally Q-superlinearly convergent.

\subsection{Interpretation of the subproblem}

To get some insight into the origin of the \gls{sqp} method, we now interpret the \gls{sqp} subproblem~\cref{eq:sqp-subproblem}.
Note that the second-order term of the objective function~\cref{eq:sqp-subproblem-obj} involves the Lagrangian function of problem~\cref{eq:problem-cobyqa-no-bounds} and not only its objective function.
It is in fact necessary due to the nonlinearity of the constraints.
To comprehend this fact, we consider the~$2$-dimensional example of \citeauthor{Boggs_Tolle_1995}~\cite{Boggs_Tolle_1995}
\begin{align*}
    \min        & \quad -x_1 - \frac{x_2^2}{2}\\
    \text{s.t.} & \quad \norm{x}^2 - 1 = 0,\\
                & \quad x \in \R^2,
\end{align*}
whose solution is~$[1, 0]^{\T}$.
Given a perturbation~$\epsilon > 0$ and an iterate~$x^k = [1 + \epsilon, 0]^{\T}$, if the second-order term in~\cref{eq:sqp-subproblem-obj} included only~$\nabla^2 \obj(x^k)$, the \gls{sqp} subproblem would be
\begin{align*}
    \min        & \quad -d_1 - \frac{d_2^2}{2}\\
    \text{s.t.} & \quad d_1 = -\frac{\epsilon (2 + \epsilon)}{2 (1 + \epsilon)},\\
                & \quad d \in \R^2,
\end{align*}
which is unbounded from below, regardless of the magnitude of~$\epsilon$.

\subsubsection{Approximation of the \glsfmtlong{kkt} conditions}

According to \cref{thm:first-order-necessary-conditions}, if~$x^{\ast} \in \R^n$ is a local solution to the problem~\cref{eq:problem-cobyqa-no-bounds}, under some mild assumptions, there exists a Lagrange multiplier~$\lambda^{\ast} = [\lambda_i^{\ast}]_{i \in \iub \cup \ieq}^{\T}$ with~$\lambda_i^{\ast} \in \R$ for all~$i \in \iub \cup \ieq$ such that
\begin{subequations}
    \begin{empheq}[left=\empheqlbrace]{alignat=2}
        & \nabla_x \lag(x^{\ast}, \lambda^{\ast}) = 0,  && \\
        & \con{i}(x^{\ast}) \le 0,                      && \quad \text{if~$i \in \iub$,}\\
        & \con{i}(x^{\ast}) = 0,                        && \quad \text{if~$i \in \ieq$,}\\
        & \lambda_i^{\ast} \con{i}(x^{\ast}) = 0,       && \quad \text{if~$i \in \iub$,} \label{eq:sqp-kkt-complementary-slackness}\\
        & \lambda_i^{\ast} \ge 0,                       && \quad \text{if~$i \in \iub$.}
    \end{empheq}
\end{subequations}
Let~$(x^k, \lambda^k)$ be some approximation of~$(x^{\ast}, \lambda^{\ast})$, and let~$(d^k, u^k)$ satisfy
\begin{subequations}
    \label{eq:sqp-kkt-step}
    \begin{empheq}[left=\empheqlbrace]{alignat=2}
        & \nabla_x \lag(x^{k}, \lambda^{k} + u^k) + \nabla_{x, x}^2 \lag(x^{k}, \lambda^{k}) d^k = 0,   && \\
        & \con{i}(x^k) + \inner{\nabla \con{i}(x^k), d^k} \le 0,                                        && \quad \text{if~$i \in \iub$,}\\
        & \con{i}(x^k) + \inner{\nabla \con{i}(x^k), d^k} = 0,                                          && \quad \text{if~$i \in \ieq$,}\\
        & (\lambda_i^{k} + u_i^k) [\con{i}(x^k) + \inner{\nabla \con{i}(x^k), d^k}] = 0,                && \quad \text{if~$i \in \iub$,} \label{eq:sqp-kkt-step-complementary-slackness}\\
        & \lambda_i^{k} + u_i^k \ge 0,                                                                  && \quad \text{if~$i \in \iub$.}
    \end{empheq}
\end{subequations}
Note that this conditions resemble the first-order Taylor approximation at~$(x^k, \lambda^k)$, in which case~$(d^k, u^k)$ would be a Newton step.
However, the condition~\cref{eq:sqp-kkt-step-complementary-slackness} is not directly the linearization of~\cref{eq:sqp-kkt-complementary-slackness}, but it includes the second-order term~$u_i^k \inner{\nabla \con{i}(x^k), d^k}$.
We remark that the conditions~\cref{eq:sqp-kkt-step} are nothing but the \gls{kkt} conditions of the \gls{sqp} subproblem~\cref{eq:sqp-subproblem} associated with the Lagrange multiplier~$\lambda^k + u^k$.

\subsubsection{Approximation of a modified Lagrangian}

A second interpretation of the \gls{sqp} subproblem~\cref{eq:sqp-subproblem} is as follows.
Given an iterate~$x^k \in \R^n$, let~$\lag[k]$ be the modified Lagrangian function
\begin{equation*}
    \lag[k](x, \lambda) \eqdef \obj(x) + \sum_{\mathclap{i \in \iub \cup \ieq}} \lambda_i \delta_i^k(x), \quad \text{for~$x \in \R^n$ and~$\lambda_i \in \R$ for~$i \in \iub \cup \ieq$},
\end{equation*}
where~$\lambda = [\lambda_i]_{i \in \iub \cup \ieq}^{\T}$, and where~$\delta_i^k$, for~$i \in \iub \cup \ieq$, denotes the departure from linearity\footnote{When~$\con{i}$ is strictly convex, it is the Bregman distance~\cite{Bregman_1967} associated with~$\con{i}$ for the point~$x^k$.}~\cite{Robinson_1972,Hoek_1982} associated with~$\con{i}$ for the point~$x^k$, defined by
\begin{equation*}
    \delta_i^k(x) \eqdef \con{i}(x) - \con{i}(x^k) - \inner{\nabla \con{i}(x^k), x - x^k}, \quad \text{for~$x \in \R^n$.}
\end{equation*}
The \gls{sqp} subproblem~\cref{eq:sqp-subproblem} can then be seen as the minimization of the second-order Taylor approximation of~$\lag[k]$ subject to the linearized constraints, i.e.,
\begin{align}
    \min        & \quad \inner{\nabla_x \lag[k](x^k, \lambda^k), d} + \frac{1}{2} \inner{d, \nabla_{x, x}^2 \lag[k](x^k, \lambda^k) d}\\
    \text{s.t.} & \quad \con{i}(x^k) + \inner{\nabla \con{i}(x^k), d} \le 0, ~ i \in \iub,\\
                & \quad \con{i}(x^k) + \inner{\nabla \con{i}(x^k), d} = 0, ~ i \in \ieq,\\
                & \quad d \in \R^n.
\end{align}

\subsubsection{Approximation of the augmented Lagrangian}

The original problem~\cref{eq:problem-cobyqa-no-bounds} can be reformulated by introducing a slack variable~$s$ as
\begin{align}
    \min        & \quad \obj(x)\\
    \text{s.t.} & \quad \con{i}(x) + s_i = 0, ~ i \in \iub,\\
                & \quad \con{i}(x) = 0, ~ i \in \ieq,\\
                & \quad x \in \R^n, ~ s \ge 0.
\end{align}
We assume here without loss of generality that~$\ieq = \emptyset$.
Let~$\lag[\mathsf{A}]$ be the augmented Lagrangian of this problem, i.e.,
\begin{equation*}
    \lag[\mathsf{A}](x, s, \lambda) \eqdef \obj(x) + \sum_{i \in \iub} \bigg[ \lambda_i (\con{i}(x) + s_i) + \frac{\gamma}{2} (\con{i}(x) + s_i)^2 \bigg],
\end{equation*}
for~$x \in \R^n$ and~$s_i, \lambda_i \in \R$ for~$i \in \iub$, where~$\gamma \ge 0$ is a given penalty parameter.

\section{The trust-region framework}

\subsection{Merit functions and penalty coefficients}

\subsection{Composite-step approach}

\section{Outline of the \glsfmttext{cobyqa} method}

\subsection{Interpolation-based quadratic models}

\subsection{Geometry of the interpolation set}

\subsection{Estimation of the Lagrange multipliers}

\subsection{Maratos effect and \glsfmtlong{soc}}
