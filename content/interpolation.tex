%% contents/interpolation.tex
%% Copyright 2021-2022 Tom M. Ragonneau
%
% This work may be distributed and/or modified under the
% conditions of the LaTeX Project Public License, either version 1.3
% of this license or (at your option) any later version.
% The latest version of this license is in
%   http://www.latex-project.org/lppl.txt
% and version 1.3 or later is part of all distributions of LaTeX
% version 2005/12/01 or later.
%
% This work has the LPPL maintenance status `maintained'.
%
% The Current Maintainer of this work is Tom M. Ragonneau.
\chapter{Interpolation models for \glsfmtlong{dfo}}

\section{Introduction and motivation}

As mentioned in \cref{ch:introduction}, model-based \gls{dfo} methods necessitate to approximate locally the functions involved in optimization problems by simple functions, referred to as \emph{models} or \emph{surrogates}.
These models are used to construct subproblems that are in turn approximately minimized.
Examples of such functions commonly used in the literature are polynomials and \glspl{rbf}~\cite{Powell_2004a}.
In this thesis, we focus on linear and quadratic polynomial models, i.e., on polynomials of degree at most one and two, respectively.

Let~$\lpoly$ and~$\qpoly$ denote the spaces of linear and quadratic polynomials on~$\R^n$, respectively.
In a \gls{dfo} context, models from~$\lpoly$ or~$\qpoly$ are built for a real-valued function~$\obj$ without using derivatives.
This can be done by interpolation schemes based on function values.
Given a finite set of points~$\xpt = \set{y^1, y^2, \dots, y^m} \subseteq \R^n$, we construct a model~$\objm$ that interpolates the function~$\obj$ on~$\xpt$, i.e.,
\begin{equation}
    \label{eq:interpolation-conditions}
    \objm(y^i) = \obj(y^i), \quad \text{for~$i \in \set{1, 2, \dots, m}$}.
\end{equation}

The conditions~\cref{eq:interpolation-conditions} may be inconsistent.
In such a case, models can be built using regression schemes.
For example, a least-square regression model~$\objm$ minimizes
\begin{equation*}
    \sum_{i = 1}^m [\obj(y^i) - \objm(y^i)]^2.
\end{equation*}
Although there are successful methods that use regression models (see, e.g.,~\cite{Billups_Larson_Graf_2013,Conn_Scheinberg_Vicente_2008b}), the \gls{dfo} methods we present and develop in this thesis use interpolation models and ensure that the interpolation conditions are consistent and well-conditioned (see \cref{ch:pdfo,ch:cobyqa-introduction}).

It is also possible to use polynomials of degree higher than two.
However, we do not consider such models in this thesis due to the following observations.
\begin{enumerate}
    \item As shown in~\cite[thm.~2.5]{Wendland_2005}, the space of polynomials on~$\R^n$ of degree at most~$k$ has a dimension of
    \begin{equation*}
        \binom{n + k}{n} = \frac{1}{k!} \prod_{i = 1}^k (n + i) \ge \frac{n^k}{k!}.
    \end{equation*}
    Therefore, to determine a model from this space merely by the interpolation conditions~\cref{eq:interpolation-conditions}, we need in general~$\bigo(n^k)$ function values.
    This amount is unacceptable in a \gls{dfo} context unless~$k$ is small.
    It is possible to reduce this number with underdetermined interpolation, which is used by several optimization methods for~$k \le 2$ (see \cref{sec:underdetermined-interpolation}), including the method \gls{cobyqa} developed in this thesis (see \cref{ch:cobyqa-introduction}).
    Using underdetermined interpolation models with~$k \ge 3$ is out of the scope of this thesis, although it is an interesting research direction.
    \item The \gls{dfo} methods need to solve approximately subproblems (e.g., trust-region subproblems) built upon these models.
    Sophisticated models usually lead to complicated subproblems to solve.
    On the other hand, even with models that are not quadratic, practical algorithms normally solve the subproblems based on first- or second-order approximations of the models, e.g., calculate the approximate Cauchy point discussed in~\cite[\S~6.3.3]{Conn_Gould_Toint_2000}.
    Therefore, building polynomial models of degree higher than two may not be necessary.
\end{enumerate}

Although we do not study \gls{rbf} models, we mention that there also exist many \gls{dfo} methods based on these models.
Examples of such methods include \gls{orbit}~\cite{Wild_Regis_Shoemaker_2008}, \gls{conorbit}~\cite{Regis_Wild_2017}, and \gls{boosters}~\cite{Oeuvray_Bierlaire_2009}.

\section{Elementary concepts of multivariate interpolation}
\label{sec:multivariate-interpolation}

The space in which~$\objm$ lies in this section is either~$\lpoly$ or~$\qpoly$.
Moreover, we assume in this section that number of interpolation points~$m$ equals the dimension of the chosen space, i.e., either~$n + 1$ or~$(n + 1)(n + 2) / 2$, respectively.

\subsection{Poisedness of interpolation sets}

Before studying properties of multivariate interpolation, we must introduce the following notion of poisedness.

\begin{definition}[Poisedness]
    The set~$\xpt$ is \emph{poised} for interpolation on~$\lpoly$ if the interpolation system~\cref{eq:interpolation-conditions} has a unique solution in~$\lpoly$ for any real-valued function~$\obj$.
    Similarly, it is \emph{poised} for interpolation on~$\qpoly$ if the system has a unique solution in~$\qpoly$ for any real-valued function~$\obj$.
\end{definition}

Let us first consider the problem of finding a linear model~$\objm \in \lpoly$ satisfying the interpolation system~\cref{eq:interpolation-conditions} whenever~$m = \dim \lpoly = n + 1$.
In the natural basis of~$\lpoly$, the system~\cref{eq:interpolation-conditions} can be reformulated as
\begin{equation}
    \label{eq:linear-interpolation-conditions}
    \alpha + \inner{g, y^i} = \obj(y^i), \quad \text{for~$i \in \set{1, 2, \dots, m}$},
\end{equation}
where~$\alpha \in \R$ and~$g \in \R^n$ are the coefficients to determine.
If~$\xpt$ is poised for linear interpolation, given~$(\alpha^{\ast}, g^{\ast})$ the unique solution to the system~\cref{eq:linear-interpolation-conditions}, the linear model~$\objm$ is defined by
\begin{equation*}
    \objm(x) = \alpha^{\ast} + \inner{g^{\ast}, x}, \quad \text{for~$x \in \R^n$},
\end{equation*}
and the vector~$\nabla \objm \equiv g^{\ast}$ is referred to as the \emph{simplex gradient} of~$\obj$ for the interpolation set~$\xpt$.
Such models are used for instance by \gls{cobyla}~\cite{Powell_1994}, a \gls{dfo} method for nonlinearly-constrained optimization detailed in \cref{subsec:cobyla}.

The problem of finding a quadratic model~$\objm \in \qpoly$ satisfying the interpolation system~\cref{eq:interpolation-conditions} when~$m = \dim \qpoly = (n + 1)(n + 2) / 2$ is very similar.
This kind of models are used for instance by \gls{uobyqa}~\cite{Powell_2002}, a \gls{dfo} method for unconstrained optimization detailed in \cref{subsec:uobyqa}.
The advantage of quadratic models is that they capture curvature information of the function~$\obj$, and are more precise than linear models.
In the remaining of this chapter, we focus our discussions on quadratic models.

\subsection{Lagrange polynomials}
\label{sec:lagrange-polynomials}

Throughout this section, we fix a poised interpolation set~$\xpt = \set{y^1, y^2, \dots, y^m} \subseteq \R^n$ of~$m = \dim \qpoly = (n + 1)(n + 2) / 2$ points.
We introduce the Lagrange polynomials for the interpolation problem
\begin{equation}
    \label{eq:interpolation-conditions-quadratic}
    \objm(y^i) = \obj(y^i), \quad \text{for~$i \in \set{1, 2, \dots, m}$}, \quad \text{with~$\objm \in \qpoly$}.
\end{equation}

\begin{definition}[Lagrange polynomials for~\cref{eq:interpolation-conditions-quadratic}]
    \label{def:lagrange-polynomials}
    For each~$i \in \set{1, 2, \dots, m}$, the~$i$th Lagrange polynomial~$\lagp[i]$ for the interpolation problem~\cref{eq:interpolation-conditions-quadratic} is the unique quadratic polynomial that satisfies
    \begin{empheq}[left={\lagp[i](y^j) = \empheqlbrace}]{alignat*=2}
        & 1,    && \quad \text{if~$j = i$,}\\
        & 0,    && \quad \text{if~$j \in \set{1, 2, \dots, m} \setminus \set{i}$.}
    \end{empheq}
\end{definition}

We note that in \cref{def:lagrange-polynomials}, the poisedness of the interpolation set ensures the existence and the uniqueness of the Lagrange polynomials.
Further, it is well-known that the interpolant~$\objm \in \qpoly$ of~$\obj$ on~$\xpt$ can be formulated as a linear combinaison of the Lagrange polynomials, as detailed in \cref{thm:lagrange-polynomials-basis}

\begin{theorem}
    \label{thm:lagrange-polynomials-basis}
    The Lagrange polynomials~$\set{\lagp[1], \lagp[2], \dots, \lagp[m]}$ form a basis of~$\qpoly$.
    Moreover, the quadratic interpolant~$\objm$ of~$\obj$ on~$\xpt$ is given by
    \begin{equation*}
        \objm(x) = \sum_{i = 1}^m \obj(y^i) \lagp[i](x), \quad \text{for~$x \in \R^n$}.
    \end{equation*}
\end{theorem}

One of the most important application of the Lagrange polynomials is to measure the well-poisedness of the interpolation set~$\xpt$, as we discuss in the next section.

\section{Well-poisedness of interpolation sets}
\label{sec:poisedness}

As we mentioned above, the interpolation set~$\xpt$ is poised if it admits a unique interpolant in~$\qpoly$.
However, given a poised set, how good is this interpolation set?
This section will briefly discuss this topic and introduce the concept of~$\Lambda$-poisedness.

Suppose that~$\obj$ is thrice differentiable on~$\R^n$.
It is well-known that for any~$x \in \R^n$, we have
\begin{equation}
    \label{eq:Ciarlet_Raviart_0}
    \abs{\obj(x) - \objm(x)} \le \frac{\nu}{6} \sum_{i = 1}^m \abs{\lagp[i](x)} \norm{x - y^i}^3,
\end{equation}
where~$\nu$ is an upper bound on the absolute value of the third-order directional derivatives of~$\obj$~\cite[thm.~2]{Powell_2001}.
A similar bound is established in~\cite[thm.~2]{Ciarlet_Raviart_1972}.
For any compact set~$\mathcal{C} \subseteq \R^n$, we then clearly have
\begin{equation}
    \label{eq:ciarlet-raviard-bound}
    \max_{x \in \mathcal{C}} \abs{\obj(x) - \objm(x)} \le \frac{m \nu \Lambda}{6} \max_{1 \le i \le m} \max_{x \in \mathcal{C}} \norm{x - y^i}^3,
\end{equation}
where~$\Lambda$ is defined by
\begin{equation}
    \label{eq:lambda-poisedness-convex-hull}
    \Lambda \eqdef \max_{1 \le i \le m} \max_{x \in \mathcal{C}} \abs{\lagp[i](x)}.
\end{equation}
In this sense,~$\Lambda$ measures the well-poisedness of the interpolation set~$\xpt$ with respect to the set~$\mathcal{C}$.
This motivates the following concept of~$\Lambda$-poisedness.

\begin{definition}[$\Lambda$-poisedness~\cite{Conn_Scheinberg_Vicente_2009b}] % def.~3.6
    \label{def:lambda-poisedness}
    A poised interpolation set~$\xpt \subseteq \R^n$ is said to be~$\Lambda$-poised in a compact set~$\mathcal{C} \subseteq \R^n$, for some~$\Lambda > 0$, if
    \begin{equation*}
        \Lambda \ge \max_{1 \le i \le m} \max_{x \in \mathcal{C}} \abs{\lagp[i](x)}.
    \end{equation*}
\end{definition}

Note that in this definition, the compact set~$\mathcal{C} \subseteq \R^n$ may or may not contain the interpolation points in~$\xpt$.
The compactness of~$\mathcal{C}$ can be relaxed if we take the supremum instead of the maximum over~$\mathcal{C}$.
If~$\xpt$ is~$\Lambda_0$-poised in~$\mathcal{C}$, it is obviously~$\Lambda$-poised in~$\mathcal{C}$ for any~$\Lambda \ge \Lambda_0$.
Alternative definitions of the~$\Lambda$-poisedness of sets are given in~\cite[\S~3.3]{Conn_Scheinberg_Vicente_2009b}.

As discussed in~\cite[\S~3.3]{Conn_Scheinberg_Vicente_2009b}, the~$\Lambda$-poisedness measures how good an interpolation set is.
A lower~$\Lambda$ indicates a better interpolation set.
Indeed, if~$\xpt$ is~$\Lambda$-poised in a compact set~$\mathcal{C} \subseteq \R^n$ and~$\Lambda$ is reasonably low, then~\cref{eq:ciarlet-raviard-bound} shows that the quadratic interpolant~$\objm$ represents~$\obj$ reasonably well in~$\mathcal{C}$.

The notion of~$\Lambda$-poisedness is intrinsically related to the well-conditioning of the interpolation system.
Note that the interpolation system~\cref{eq:interpolation-conditions-quadratic} is linear with respect to the coefficients of the interpolant~$\objm$.
One can show that the set~$\xpt$ is~$\Lambda$-poised if and only if the condition number of the coefficient matrix of this linear system is bounded by some terms proportional to~$\Lambda$~\cite[thm.~3.14]{Conn_Scheinberg_Vicente_2009b}.
Therefore, the lower~$\Lambda$ is, the better the conditioning of the interpolation system is and hence, the better the interpolation set is from a numerical standpoint.

In practice, however, we do not use the~$\Lambda$-poisedness of interpolation sets directly.
This is because determining whether a set is~$\Lambda$-poised is a difficult problem, even if the compact set~$\mathcal{C} \subseteq \R^n$ is simple (e.g., a ball).
It is a theoretical tool that we mostly use to justify design choices of algorithms.

\section{Underdetermined quadratic interpolation}
\label{sec:underdetermined-interpolation}

As before, we focus on quadratic interpolation in this section.
Building a quadratic model by the interpolation system~\cref{eq:interpolation-conditions-quadratic} requires~$\mathcal{O}(n^2)$ function evaluations.
Therefore, if a \gls{dfo} method employs such models, its initialization already costs~$\mathcal{O}(n^2)$ function evaluations, which are needed to build the first model.
To reduce this amount of function evaluations, we use underdetermined interpolation, as presented hereinafter.

Underdetermined interpolation works as follows.
Assume that we are given an interpolation set~$\xpt \subseteq \R^n$ such that the interpolation conditions~\cref{eq:interpolation-conditions} are consistent.
The number~$m$ of interpolation points may, however, be lower than~$\dim \qpoly$.
In such a case, the interpolation conditions may \emph{not} define a unique interpolant.
To select one, we choose a functional~$\mathcal{F} : \qpoly \to \R$ that reflects a desired property or regularity of the interpolants.
An interpolant~$\objm$ of~$\obj$ is then defined as a solution of
\begin{subequations}
    \label{eq:underdetermined-models}
    \begin{align}
        \min        & \quad \mathcal{F}(Q)\\
        \text{s.t.} & \quad Q(y^i) = \obj(y^i), ~ i \in \set{1, 2, \dots, m}\\
                    & \quad Q \in \qpoly. \nonumber
    \end{align}
\end{subequations}
Two examples of the functional~$\mathcal{F}$ are given in \cref{subsec:least-frobenius-norm-models,subsec:symmetric-broyden-updates}.
If~$m = \dim \qpoly$ and if~$\xpt$ is poised in the sense of \cref{def:poisedness}, then the underdetermined interpolant reduces to the unique model that satisfies the interpolation conditions.

In the remaining of this section, we adapt the definitions and some properties presented in the previous section to special cases of underdetermined interpolation.

\subsection{Least Frobenius norm quadratic models}
\label{subsec:least-frobenius-norm-models}

One of the simplest functional~$\mathcal{F}$ mentioned above is the Frobenius norm of the Hessian matrix of the interpolants.
In this context, a quadratic model~$\objm$ solves
\begin{subequations}
    \label{eq:least-frobenius-norm-models}
    \begin{align}
        \min        & \quad \norm{\nabla^2 Q}_{\mathsf{F}}\\
        \text{s.t.} & \quad Q(y^i) = \obj(y^i), ~ i \in \set{1, 2, \dots, m} \label{eq:least-frobenius-norm-models-eq}\\
                    & \quad Q \in \qpoly, \nonumber
    \end{align}
\end{subequations}
where~$\norm{\cdot}_{\mathsf{F}}$ denotes the Frobenius norm.
This norm is chosen because it can be easily evaluated, unlike the spectral norm for example.
The Frobenius norm also brings computational advantages, because~\cref{eq:least-frobenius-norm-models} is essentially a quadratic programming problem, which can be tackled by solving a linear system.
Examples of \gls{dfo} methods that use least Frobenius norm quadratic models include \gls{mnh}~\cite{Wild_2008} and \gls{dfoalg}~\cite{Conn_Scheinberg_Toint_1997a,Conn_Scheinberg_Toint_1997b,Conn_Scheinberg_Toint_1998}.
% https://www.zhangzk.net/docs/publications/2014snq.pdf
We also assume hereafter that~$m \ge n + 2$ as otherwise, any solution to~\cref{eq:least-frobenius-norm-models} would be linear.

We first present the notion of poisedness for problem~\cref{eq:least-frobenius-norm-models} as follows.

\begin{definition}[Poisedness]
    \label{def:poisedness}
    The set~$\xpt$ is \emph{poised} in the minimum Frobenius norm sense if the solution to problem~\cref{eq:least-frobenius-norm-models} exists and is unique for any real-valued function~$\obj$.
\end{definition}

To get an in-depth view of the poisedness of~$\xpt$, let us investigate the solvability of the problem~\cref{eq:least-frobenius-norm-models}.
Given any fix~$\bar{x} \in \R^n$, we expand a quadratic polynomial~$Q \in \qpoly$ for any~$x \in \R^n$ as
\begin{equation*}
    Q(x) = \alpha + \inner{g, x - \bar{x}} + \frac{1}{2} \inner{x - \bar{x}, H(x - \bar{x})},
\end{equation*}
where~$\alpha \in \R$,~$g \in \R^n$, and~$H \in \R^{n \times n}$ are the coefficients of~$Q$, with~$H$ being symmetric.
Note that the constraint~\cref{eq:least-frobenius-norm-models-eq} is linear with respect to~$(\alpha, g, H)$.
Moreover, the Frobenius norm is convex, so that the problem~\cref{eq:least-frobenius-norm-models} is convex.
In this setting, we can show that the \gls{kkt} conditions~\cref{eq:kkt-introduction} given in \cref{thm:first-order-necessary-conditions} are necessary and sufficient.
Let~$\lambda = [\lambda_1, \lambda_2, \dots, \lambda_m]^{\top}$ be the Lagrange multiplier of~\cref{eq:least-frobenius-norm-models}.
The stationarity condition~\cref{eq:kkt-introduction-stationarity} can be formulated as
\begin{equation*}
    \sum_{i = 1}^m \lambda_i (y^i - \bar{x}) (y^i - \bar{x})^{\T} = H, \quad \sum_{i = 1}^m \lambda_i = 0, \quad \text{and} \quad \sum_{i = 1}^m \lambda_i (y^i - \bar{x}) = 0.
\end{equation*}
When combined with the primal feasibility condition~\cref{eq:kkt-introduction-primal-feasibility-eq}, the coefficients of~$Q$ can be evaluated by solving the system
\begin{equation}
    \label{eq:least-frobenius-kkt-system}
    \begin{bmatrix}
        Y_H         & Y_L\\
        Y_L^{\T}    & 0
    \end{bmatrix}
    \begin{bmatrix}
        \lambda\\
        \alpha\\
        g
    \end{bmatrix}
    =
    \begin{bmatrix}
        \obj(\xpt)\\
        0
    \end{bmatrix},
\end{equation}
where~$\obj(\xpt) \eqdef [\obj(y^1), \obj(y^2), \dots, \obj(y^m)]^{\top}$ and where~$Y_H \in \R^{m \times m}$ and~$Y_L \in \R^{m \times (n + 1)}$ are defined by
\begin{equation}
    \label{eq:least-frobenius-kkt-system-matrice-1}
    Y_H \eqdef \frac{1}{2}
    \begin{bmatrix}
        \norm{y^1 - \bar{x}}^4                  & \inner{y^1 - \bar{x}, y^2 - \bar{x}}^2    & \cdots    & \inner{y^1 - \bar{x}, y^m - \bar{x}}^2\\
        \inner{y^2 - \bar{x}, y^1 - \bar{x}}^2  & \norm{y^2 - \bar{x}}^4                    & \cdots    & \inner{y^2 - \bar{x}, y^m - \bar{x}}^2\\
        \vdots                                  & \vdots                                    & \ddots    & \vdots\\
        \inner{y^m - \bar{x}, y^1 - \bar{x}}^2  & \inner{y^m - \bar{x}, y^2 - \bar{x}}^2    & \cdots    & \norm{y^m - \bar{x}}^4\\
    \end{bmatrix}
\end{equation}
and
\begin{equation}
    \label{eq:least-frobenius-kkt-system-matrice-2}
    Y_L \eqdef 
    \begin{bmatrix}
        1       & y_1^1 - \bar{x}   & y_2^1 - \bar{x}   & \cdots    & y_n^1 - \bar{x}\\
        1       & y_1^2 - \bar{x}   & y_2^2 - \bar{x}   & \cdots    & y_n^2 - \bar{x}\\
        \vdots  & \vdots            & \vdots            & \ddots    & \vdots\\
        1       & y_1^m - \bar{x}   & y_2^m - \bar{x}   & \cdots    & y_n^m - \bar{x}\\
    \end{bmatrix}.
\end{equation}
The Hessian matrix~$H$ can then be retrieved from the value of~$\lambda$.
Thus, the interpolation set~$\xpt$ is poised if and only if the coefficient matrix in~\cref{eq:least-frobenius-kkt-system} is nonsingular~\cite[\S~5.3]{Conn_Scheinberg_Vicente_2009b}.
We can show that this nonsingularity does not depend on the choice of~$\bar{x}$.
In fact,~\cite{Conn_Scheinberg_Vicente_2009b} defines the poisedness of~$\xpt$ as the nonsingularity of the coefficient matrix in~\cref{eq:least-frobenius-kkt-system}.

The uniqueness of the solution to~\cref{eq:least-frobenius-norm-models} implies that the matrix~$Y_L$ has full-column rank.
This is because~$\alpha + \inner{g, x - \bar{x}}$ is a nonzero linear polynomial that interpolates the zero function on~$\xpt$ if~$(\alpha, g^{\top})^{\top} \in \R^{n + 1}$ is a nonzero vector in~$\ker(Y_L)$.
Such a polynomial should not exist because of the uniqueness of the solution in the case~$\obj \equiv 0$, since the zero polynomial should be the only solution.
Therefore, any interpolation set that is poised in the sense of \cref{def:poisedness} must be affinely independent, i.e., it does not lie in a low-dimensional affine subset of~$\R^n$.

\subsubsection{Overview of the minimum Frobenius norm Lagrange polynomials}

Throughout this section, we fix a poised interpolation set~$\xpt \subseteq \R^n$ of~$m$ points in the minimum Frobenius norm sense, with~$n + 2 \le m \le \dim \qpoly$.
We now present the minimum Frobenius norm Lagrange polynomials for the interpolation system~\cref{eq:interpolation-conditions-quadratic}.

\begin{definition}[Minimum Frobenius norm Lagrange polynomials for~\cref{eq:interpolation-conditions-quadratic}]
    \label{def:lagrange-polynomials-minimum-norm}
    For each~$i \in \set{1, 2, \dots, m}$, the~$i$th minimum Frobenius norm Lagrange polynomial~$\lagp[i]$ for the interpolation problem~\cref{eq:interpolation-conditions-quadratic} is the unique quadratic polynomial that solves
    \begin{align*}
        \min        & \quad \norm{\nabla^2 Q}_{\mathsf{F}}\\
        \text{s.t.} & \quad Q(y^i) = 1,\\
                    & \quad Q(y^j) = 0, ~ j \in \set{1, 2, \dots, m} \setminus \set{i},\\
                    & \quad Q \in \qpoly.
    \end{align*}
\end{definition}

As in the case where~$m = \dim \qpoly$, any minimum Frobenius norm interpolant can be expressed as a linear combinaison of minimum Frobenius norm Lagrange polynomials, as described by \cref{thm:lagrange-polynomials-basis-minimum-norm} (see~\cite[\S~3]{Powell_2004b}).

\begin{theorem}
    \label{thm:lagrange-polynomials-basis-minimum-norm}
    The minimum Frobenius norm quadratic interpolant~$\objm$ of~$\obj$ on~$\xpt$ is given for any~$x \in \R^n$ by
    \begin{equation*}
        \objm(x) = \sum_{i = 1}^m \obj(y^i) \lagp[i](x).
    \end{equation*}
\end{theorem}

However, the set of all minimum Frobenius norm Lagrange polynomials is not a basis of~$\qpoly$ if~$m < \dim \qpoly$.
Now that we defined the minimum Frobenius norm Lagrange polynomials, we are equiped to set up the~$\Lambda$-poisedness of the interpolation set~$\xpt$ in the minimum Frobenius norm sense.

\subsubsection{Poisedness of interpolation sets in the minimum Frobenius norm sense}

We now present the notion of~$\Lambda$-poisedness in the minimum Frobenius norm sense, which is a straightforward adaptation of \cref{def:lambda-poisedness}.

\todo[noline]{Put exact references in definitions/theorems/\dots}

\begin{definition}[$\Lambda$-poisedness in the minimum Frobeninus norm sense~\cite{Conn_Scheinberg_Vicente_2009b}] % def.~5.6
    \label{def:lambda-poisedness-minimum-norm}
    A poised interpolation set~$\xpt \subseteq \R^n$ in the minimum Frobenius norm sense is said to be~$\Lambda$-poised in the minimum Frobenius norm sense in a compact set~$\mathcal{C} \subseteq \R^n$, for some~$\Lambda > 0$, if
    \begin{equation*}
        \Lambda \ge \max_{1 \le i \le m} \max_{x \in \mathcal{C}} \abs{\lagp[i](x)}.
    \end{equation*}
\end{definition}

Similarly to what we mentioned before, if~$\xpt$ is~$\Lambda_0$-poised in the minimum Frobenius norm sense in~$\mathcal{C}$, it is obviously~$\Lambda$-poised in the minimum Frobenius norm sense in~$\mathcal{C}$ for all~$\Lambda \ge \Lambda_0$.
Moreover, one can show that the set~$\xpt$ is~$\Lambda$-poised if and only if the condition number of the coefficient matrix in~\cref{eq:least-frobenius-kkt-system} is bounded by some terms proportional to~$\Lambda$~\cite[thm.~5.8]{Conn_Scheinberg_Vicente_2009b}.

\subsection{Quadratic models based on symmetric Broyden updates}
\label{subsec:symmetric-broyden-updates}

In this section, we present another functional~$\mathcal{F}$ that can be used in the variational problem~\cref{eq:underdetermined-models}.
It is introduced by \citeauthor{Powell_2004b}~\cite{Powell_2004b}.
Instead of taking up the freedom bequeathed by the interpolation conditions~\cref{eq:interpolation-conditions-quadratic} by minimizing the Frobenius norm of the Hessian matrix of the interpolant, it minimizes the difference between the Hessian matrix of the interpolant and a prior estimation~$H_{\text{old}} \in \R^{n \times n}$ of this matrix.
More specifically, we build a quadratic model~$\objm$ by solving
\begin{subequations}
    \label{eq:symmetric-broyden-models}
    \begin{align}
        \min        & \quad \norm{\nabla^2 Q - H_{\text{old}}}_{\mathsf{F}}\\
        \text{s.t.} & \quad Q(y^i) = \obj(y^i), ~ i \in \set{1, 2, \dots, m}\\
                    & \quad Q \in \qpoly.
    \end{align}
\end{subequations}
In a \gls{dfo} method, we choose at the~$k$th iteration~$H_{\text{old}} = \nabla^2 \objm[k - 1]$, where~$\objm[k - 1] \in \qpoly$ denotes the model of~$\obj$ at the~$(k - 1)$th iteration, and we define~$\objm[-1] \equiv 0$.
This scheme is referred to as the \emph{least Frobenius norm updating of quadratic models}~\cite{Powell_2004b}.
Examples of \gls{dfo} methods that use this update include \gls{newuoa}~\cite{Powell_2006}, \gls{bobyqa}~\cite{Powell_2009}, and~\gls{lincoa}~\cite{Powell_2015}, presented in \cref{subsec:newuoa-bobyqa-lincoa}.
The new method we introduce in \cref{ch:cobyqa-introduction} also uses this update.

The discussions we had on the minimum Frobenius norm models can be easily adapted to the least Frobenius norm updating.
It is because if we are given a function~$\objm[\text{old}]$ such that~$H_{\text{old}} = \nabla^2 \objm[\text{old}]$, then~$\objm - \objm[\text{old}]$ solves
\begin{align*}
    \min        & \quad \norm{\nabla^2 Q}_{\mathsf{F}}\\
    \text{s.t.} & \quad Q(y^i) = \obj(y^i) - \objm[\text{old}](y^i), ~ i \in \set{1, 2, \dots, m}\\
                & \quad Q \in \qpoly.
\end{align*}
Therefore, the coefficients of the model obtained by least Frobenius norm updates can be evaluated by solving a linear system similar to~\cref{eq:least-frobenius-kkt-system}.

The update~\cref{eq:symmetric-broyden-models} is also referred to as the \emph{\gls{dfo} symmetric Broyden update}~\cite{Powell_2013}.
The symmetric Broyden method~\cite{Powell_1970b} provides a mechanism for updating the Hessian matrices of the models in gradient-based optimization.
In such a method, the~$k$th model~$\objm[k]$ of~$\obj$ is evaluated by solving the variational problem
\begin{align*}
    \min        & \quad \norm{\nabla^2 Q - \nabla^2 \objm[k - 1]}_{\mathsf{F}}\\
    \text{s.t.} & \quad Q(x^k) = \obj(x^k),\\
                & \quad \nabla Q(x^k) = \nabla \obj(x^k),\\
                & \quad \nabla Q(x^{x - 1}) = \nabla \obj(x^{k - 1}),\\
                & \quad Q \in \qpoly,
\end{align*}
where~$\set{x^k}_{k \ge 0}$ denotes the iterates of the method.
The update~\cref{eq:symmetric-broyden-models} can therefore be seen as a \gls{dfo} variation of the symmetric Broyden update.

\subsection{Implementation in \glsfmtlong{dfo} methods}

We assume in this section that we are given a model-based \gls{dfo} methods that generates the iterate~$x^{k + 1} \in \R^n$ at the~$k$th iteration, with~$k \ge 0$.
Further, we assume that it builds the models~$\objm[k] \in \qpoly$ of the objective function~$\obj$ by an underdetermined interpolation scheme on the interpolation set~$\xpt[k] = \set{y^{k, 1}, y^{k, 2}, \dots, y^{k, m}} \subseteq \R^n$, with~$x^k \in \xpt[k]$.
The quadratic model~$\objm[k]$ is
\begin{equation*}
    \objm[k](x) = \alpha^k + \inner{g^k, x - x^k} + \frac{1}{2} \inner{x - x^k, H^k (x - x^k)}, \quad \text{for~$x \in \R^n$}.
\end{equation*}
where~$\alpha^k \in \R$,~$g^k \in \R^n$, and~$H^k \in \R^{n \times n}$ are the coefficients of the model,~$H$ being symmetric.
As we shown before, the coefficients of~$\objm[k]$ can be evaluated by solving
\begin{equation}
    \label{eq:least-frobenius-kkt-system-2}
    \begin{bmatrix}
        Y_H         & Y_L\\
        Y_L^{\T}    & 0
    \end{bmatrix}
    \begin{bmatrix}
        \lambda\\
        \alpha\\
        g
    \end{bmatrix}
    =
    \begin{bmatrix}
        v\\
        0
    \end{bmatrix},
\end{equation}
where~$\alpha \in \R$,~$g \in \R^n$, and~$H \in \R^{n \times n}$ are unknown coefficients,~$H$ being symmetric, and where~$v \in \R^m$ has the value
\begin{equation*}
    v \eqdef
    \begin{bmatrix}
        \obj(y^{k, 1})\\
        \obj(y^{k, 2})\\
        \vdots\\
        \obj(y^{k, m})
    \end{bmatrix}
    \quad \text{or} \quad v \eqdef
    \begin{bmatrix}
        \obj(y^{k, 1}) - \objm[k - 1](y^{k, 1})\\
        \obj(y^{k, 2}) - \objm[k - 1](y^{k, 2})\\
        \vdots\\
        \obj(y^{k, m}) - \objm[k - 1](y^{k, m})
    \end{bmatrix},
\end{equation*}
the former being chosen if we want a least Frobenius norm model and the latter if we want to update the model by a \gls{dfo} symmetric Broyden update, and where the matrices~$Y_H \in \R^{m \times m}$ and~$Y_L \in \R^{m \times (n + 1)}$ are defined by~\cref{eq:least-frobenius-kkt-system-matrice-1,eq:least-frobenius-kkt-system-matrice-2}, respectively.
Further, if we want a least Frobenius norm model, we set~$\alpha^k = \alpha$,~$g^k = g$, and
\begin{equation*}
    H^k = \sum_{i = 1}^m \lambda_i (y^{k, i} - x^k) (y^{k, i} - x^k)^{\T},
\end{equation*}
and if we want a model obtained by the \gls{dfo} symmetric Broyden update, we set
\begin{empheq}[left=\empheqlbrace]{alignat*=1}
    & \alpha^k = \objm[k - 1](x^k) + \alpha,\\
    & g^k = \nabla \objm[k - 1](x^k) + g,\\
    & H^k = \nabla^2 \objm[k - 1] + \sum_{i = 1}^m \lambda_i (y^{k, i} - x^k) (y^{k, i} - x^k)^{\T}.
\end{empheq}

In practice, we often maintain the inverse of the coefficient matrix in~\cref{eq:least-frobenius-kkt-system-2} to evaluates the coefficients of the model~$\objm[k]$.
We present in this section techniques that are used to reduce the computational cost of computing these models and to improve the stability of the methods.

\subsubsection{Storage of the quadratic models}

The intercepts and the gradient vectors of the models can be stored directly, as they are directly evaluated or updated by the linear system~\cref{eq:least-frobenius-kkt-system-2}.
However, solving this system does not provide directly the Hessian matrix, but a decomposition of it.
Therefore, as suggested by \citeauthor{Powell_2004b}~\cite[\S~3]{Powell_2004b}, the Hessian matrix of the model can be stored as
\begin{equation}
    \label{eq:hessian-explicit-implicit}
    \nabla^2 \objm[k] = \Gamma^k + \sum_{i = 1}^m \gamma_i^k (y^{k, i} - x^k) (y^{k, i} - x^k)^{\T},
\end{equation}
where~$\Gamma^k \in \R^{n \times n}$ is referred to as an explicit part of~$\nabla^2 \objm[k]$, and~$\set{\gamma_1^k, \gamma_2^k, \dots, \gamma_m^k} \subseteq \R$ is referred to as an implicit part of~$\nabla^2 \objm[k]$.
In doing so, we avoid computing~$m$ outer products when~$k = 0$, by setting~$\Gamma = \nabla \objm[-1] = 0$ and~$\gamma_i = \lambda_i$ for~$i \in \set{1, 2, \dots, m}$.
Evaluating the product of~$\nabla^2 \objm$ with any vector requires, therefore, to evaluate one matrix-vector product and~$m$ inner products.
This mechanism is used for instance by \gls{newuoa}~\cite{Powell_2006}, \gls{bobyqa}~\cite{Powell_2009}, and~\gls{lincoa}~\cite{Powell_2015}, presented in \cref{subsec:newuoa-bobyqa-lincoa}.
The new method we introduce in \cref{ch:cobyqa-introduction} also uses this storage techniques for the models.

\subsubsection{Storage of the inverse of the coefficient matrix}

As we mentioned earlier, the inverse of the coefficient matrix in~\cref{eq:least-frobenius-kkt-system-2} can be maintained by the \gls{dfo} method we consider.
However, this inverse should not be stored directly for the two following reasons.
\begin{enumerate}
    \item The~$(m + 1)$th row of this matrix is not needed, because it provides only the value of~$\alpha$, which is not needed since~$x^k \in \xpt[k]$ and hence,~$\alpha^k = \obj(x^k)$.
    \item As mentioned by \citeauthor{Powell_2004c}~\cite{Powell_2004c}, the leading~$m \times m$ submatrix of the inverse of the coefficient matrix in~\cref{eq:least-frobenius-kkt-system-2} has rank~$m - n - 1$ and is positive semi-definite.
    He observed, however, during the development of a \gls{dfo} methods for unconstrained optimization (see \cref{subsec:newuoa-bobyqa-lincoa}) that the rank property was lost in practice.
    Therefore, this submatrix is not stored directly, but using the rank factorization~$Z D Z^{\top}$, with~$Z \in \R^{m \times (m - n - 1)}$ and~$D \in \R^{(m - n - 1) \times (m - n - 1)}$, where~$D$ is a diagonal matrix with elements~$\pm 1$ on the diagonal.
    From the theoretical standpoint, the matrix~$D$ should always be identity.
    However, in practice, allowing the leading~$m \times m$ submatrix of the inverse of the coefficient matrix in~\cref{eq:least-frobenius-kkt-system-2} to have negative eigenvalues prevent damages from computer rounding errors.
    Details on this factorization can be found in~\cite{Powell_2004c}.
\end{enumerate}

\subsubsection{Update of the inverse of the coefficient matrix and the models}

The methods we consider in this thesis that employ the inverse of the coefficient matrix in~\cref{eq:least-frobenius-kkt-system-2} build~$\xpt[k + 1]$ from~$\xpt[k]$ by modifying only one point.
Therefore, only one row and one column of the coefficient matrix in~\cref{eq:least-frobenius-kkt-system-2} is modified from an iteration to another.
This modification leads to an at-most rank-$2$ modification of the inverse of this matrix.
Hence, we do not compute explicitely the inverse matrix at each iteration, but we update it using the Sherman-Morrison-Woodbury formula.
Such an update requires only~$\bigo(n^2)$, much less than what would be required to compute the inverse matrix from scratch.
It is also more stable.
Details on this update can be found in~\cite[\S~2]{Powell_2004c}.

An important feature of this update from a computational standpoint is the following.
Assume that only~$y^{k + 1, \ell}$ differs from~$y^{k, \ell}$.
If we are evaluating the minimum Frobenius norm model, we can set
\begin{empheq}[left=\empheqlbrace]{alignat*=2}
    & \Gamma^{k + 1} = 0,               && \\
    & \gamma_{i}^{k + 1} = \lambda_{i}, && \quad \text{for~$i \in \set{1, 2, \dots, m}$},
\end{empheq}
and if we are updating the model using the \gls{dfo} symmetric Broyden update, we can set
\begin{empheq}[left=\empheqlbrace]{alignat*=2}
    & \Gamma^{k + 1} = \Gamma^k + \gamma_{\ell}^k (y^{k, \ell} - x^k) (y^{k, \ell} - x^k)^{\top},   && \\
    & \gamma_{\ell}^{k + 1} = \lambda_{\ell},                                                       && \\
    & \gamma_{i}^{k + 1} = \gamma_{i}^k + \lambda_{i},                                              && \quad \text{for~$i \in \set{1, 2, \dots, m} \setminus \set{\ell}$}.
\end{empheq}
This also motivates the decomposition~\cref{eq:hessian-explicit-implicit}, because updating a quadratic model requires only one outer product computation once the linear system~\cref{eq:least-frobenius-kkt-system-2} is solved.

\section{An optimal number of interpolation points}

We study in this section an interpolation set that we will use in \cref{ch:cobyqa-introduction} of this thesis, where we introduce a new model-based \gls{dfo} method.
This interpolation set is adapted from~\cite{Powell_2001} as follows.
Let~$\delta > 0$ be fixed and for~$j \in \set{1, 2, \dots, 2n + 1}$, let~$z^j \in \R^n$ be
\begin{subequations}
    \label{eq:initial-interpolation-set}
    \begin{empheq}[left={z^j \eqdef \empheqlbrace}]{alignat=2}
        & 0,                        && \quad \text{if~$j = 1$,}\\
        & \delta e_{j - 1},         && \quad \text{if~$2 \le j \le n + 1$,}\\
        & -\delta e_{j - n - 1},    && \quad \text{otherwise,}
    \end{empheq}
\end{subequations}
where~$e_j \in \R^n$ denotes the~$i$th standard coordinate vector.
We then define the interpolation set~$\mathcal{Z}_m \subseteq \R^n$ for each~$m \in \set{n + 2, n + 3, \dots, 2n + 1}$ by
\begin{equation*}
    \mathcal{Z}_m \eqdef \set{z^1, z^2, \dots, z^m}.
\end{equation*}
Further, we denote by~$\ball[p][m]$ the smallest~$\ell_p$-norm ball containing~$\mathcal{Z}_m$, for~$p \in [1, \infty]$.
Note that we allow~$p = \infty$.
By the construction of~$\mathcal{Z}_m$, we observe that
\begin{equation}
    \label{eq:ball-initial}
    \ball[p][m] \equiv \ball[p](\delta) \eqdef \set{x \in \R^n : \norm{x}_p \le \delta}.
\end{equation}

Throughout this section, we denote by~$\lagp[i]$ the ~$i$th minimum Frobenius norm Lagrange polynomial associated with~$\mathcal{Z}_m$ for~$i \in \set{1, 2, \dots, m}$.
According to the~\cref{def:lambda-poisedness-minimum-norm} and the equation~\cref{eq:ball-initial}, the set~$\mathcal{Z}_m$ is~$\Lambda_p$-poised in~$\ball[p][m]$ in the minimum Frobenius norm sense with
\begin{equation*}
    \Lambda_p \eqdef \max_{1 \le i \le m} \max_{x \in \ball[p](\delta)} \abs{\lagp[i](x)}.
\end{equation*}
In what follows, we develop bounds for~$\Lambda_p$ and establish its value in some special cases.

\Cref{lem:lagrange-polynomials-initial} provides explicit formulae for~$\lagp[i]$ for all~$i \in \set{1, 2, \dots, m}$.
These formulae are given in~\cite[\S~3]{Powell_2006}, without a proof.

\begin{lemma}
    \label{lem:lagrange-polynomials-initial}
    For each~$m \in \set{n + 2, n + 3, \dots,  2n + 1}$ and all~$x \in \R^n$, we have
    \begin{empheq}[left={\lagp[i](x) = \empheqlbrace}]{alignat*=2}
        & 1 - \delta^{-2} \sum_{j = 1}^{m - n - 1} x_j^2 - \delta^{-1} \sum_{j = m - n}^n x_j,  && \quad \text{if~$i = 1$,}\\
        & (\sqrt{2} \delta)^{-2} x_{i - 1}^2 + (2 \delta)^{-1} x_{i - 1},                       && \quad \text{if~$2 \le i \le m - n$,}\\
        & \delta^{-1} x_{i - 1},                                                                && \quad \text{if~$m - n + 1 \le i \le n + 1$,}\\
        & (\sqrt{2} \delta)^{-2} x_{i - 1}^2 - (2 \delta)^{-1} x_{i - 1}.                       && \quad \text{otherwise.}
    \end{empheq}
    In the formulation of~$\lagp[1]$, if~$m = 2n + 1$, we define~$\sum_{j = m - n}^n x_j = 0$.
\end{lemma}

\begin{proof}
    Let~$i \in \set{1, 2, \dots, m}$ be fixed and let~$\lagp$ be a quadratic polynomial that satisfies
    \begin{subequations}
        \label{eq:lagrange-polynomials-initial-proof}
        \begin{empheq}[left={\lagp(z^j) = \empheqlbrace}]{alignat=2}
            & 1,    && \quad \text{if~$j = i$,}\\
            & 0,    && \quad \text{otherwise.}
        \end{empheq}
    \end{subequations}
    First, it is straightforward to verify that~$\lagp[i]$ satisfies the interpolation conditions~\cref{eq:lagrange-polynomials-initial-proof}.
    Hence, it suffices to show that the Frobenius norm of its Hessian matrix is least.
    According to the equation~\cref{eq:initial-interpolation-set}, for any~$j \in \set{1, 2, \dots, m - n - 1}$, we have~$z^1 = 0$,~$z^{j + 1} = \delta e_j$, and~$z^{n + j + 1} = - \delta e_j$.
    Therefore,
    \begin{empheq}[left=\empheqlbrace]{alignat*=1}
        & \lagp(z^{j + 1}) = \lagp(z^1) + \delta \inner{\nabla \lagp(z^1), e_j} + \frac{\delta^2}{2} \inner{e_j, (\nabla^2 \lagp) e_j},\\
        & \lagp(z^{n + j + 1}) = \lagp(z^1) - \delta \inner{\nabla \lagp(z^1), e_j} + \frac{\delta^2}{2} \inner{e_j, (\nabla^2 \lagp) e_j},
    \end{empheq}
    and hence,
    \begin{equation*}
        \inner{e_j, (\nabla^2 \lagp) e_j} = \frac{\lagp(z^{j + 1}) + \lagp(z^{n + j + 1}) - 2 \lagp(z^1)}{\delta^2}.
    \end{equation*}
    This fixes the first~$m - n - 1$ diagonal entries of~$\nabla^2 \lagp$, which are exactly those of~$\nabla^2 \lagp[i]$.
    Since all the other entries of~$\nabla^2 \lagp[i]$ are zero, we have
    \begin{equation*}
        \norm{\nabla^2 \lagp[i]}_{\mathsf{F}}^2 \le \norm{\nabla^2 \lagp}_{\mathsf{F}}^2,
    \end{equation*}
    which completes the proof.
\end{proof}

The next lemma simplifies the value of~$\Lambda_p$ for further computations.

\begin{lemma}
    \label{lem:lambda-poisedness-initial-simple}
    For any~$m \in \set{n + 2, n + 3, \dots, 2n + 1}$ and any~$p \in [1, \infty]$, we have
    \begin{equation}
        \label{eq:lambda-poisedness-initial-simple}
        \Lambda_p = \max_{x \in \ball[p](\delta)} \abs{\lagp[1](x)}.
    \end{equation}
\end{lemma}

\begin{proof}
    For each~$i \in \set{2, 3, \dots, n + 1}$, according to~\cref{lem:lagrange-polynomials-initial},~$\lagp[i](x)$ depends only on~$x_{i - 1}$ for all~$x \in \R^n$, and hence
    \begin{equation*}
        \max_{x \in \ball[p](\delta)} \abs{\lagp[i](x)} = \max_{t \in [-\delta, \delta]} \abs{\lagp[i](t e_{i - 1})} = 1.
    \end{equation*}
    Similarly, for each~$i \in \set{n + 2, n + 3, \dots, m}$, since~$\lagp[i](x)$ depends only on~$x_{i - n - 1}$ for all~$x \in \R^n$, we have
    \begin{equation*}
        \max_{x \in \ball[p](\delta)} \abs{\lagp[i](x)} = \max_{t \in [-\delta, \delta]} \abs{\lagp[i](t e_{i - n - 1})} = 1.
    \end{equation*}
    Noting that~$\lagp[1](z^1) = 1$ and~$z^1 \in \ball[p](\delta)$, we thus have
    \begin{equation*}
        \Lambda_p = \max_{x \in \ball[p](\delta)} \abs{\lagp[1](x)}.
    \end{equation*}
\end{proof}

We are now equiped to develop bounds on~$\Lambda_p$ in the general case.

\begin{theorem}
    \label{thm:lambda-poisedness-initial}
    For any~$m \in \set{n + 2, n + 3, \dots, 2n + 1}$ and any~$p \in [1, \infty]$, we have
    \begin{equation*}
        1 + (2n + 1 - m)^{\frac{p - 1}{p}} \le \Lambda_p \le n,
    \end{equation*}
    where we assume that~$0^0 = 0$ and that the lower bound is~$2n + 2 - m$ for~$p = \infty$.
\end{theorem}

\begin{proof}
    We will establish the bounds using the formulation of~$\Lambda_p$ in~\cref{lem:lambda-poisedness-initial-simple}.
    For the lower bound, by considering only the points in~$\R^n$ whose leading~$m - n - 1$ components are zeros and whose remaining~$2n + 1 - m$ components are equal, we have
    \begin{equation*}
        \Lambda_p \ge \max_{t \in \R} \set{1 - \delta^{-1} (2n + 1 - m) t : (2n + 1 - m) \abs{t}^p \le \delta^p} = 1 + (2n + 1 - m)^{\frac{p - 1}{p}}.
    \end{equation*}
    
    We now prove the upper bound.
    Note that for any~$p \ge 1$, we have~$\ball[p](\delta) \subseteq \ball[\infty](\delta)$, so that~$\Lambda_p \le \Lambda_{\infty}$.
    Therefore, we only need to show that~$\Lambda_{\infty} \le n$.
    Considering both~$\lagp[1]$ and~$-\lagp[1]$, we obtain
    \begin{equation*}
        \Lambda_{\infty} = \max_{x \in \ball[\infty](\delta)} \abs{\lagp[1](x)} = \max \set{2n + 2 - m, n - 1} \le n.
    \end{equation*}
\end{proof}

We now express the exact value of~$\Lambda_p$ in the special cases~$p \in \set{1, 2}$.

\begin{proposition}
    For any~$m \in \set{n + 2, n + 3, \dots, 2n + 1}$, we have
    \begin{subequations}
        \label{eq:lambda-poisedness-1}
        \begin{empheq}[left={\Lambda_1 = \empheqlbrace}]{alignat=2}
            & 2,    && \quad \text{if~$n + 2 \le m \le 2n$,}\\
            & 1,    && \quad \text{otherwise.}
        \end{empheq}
    \end{subequations}
\end{proposition}

\begin{proof}
    According to~\cref{thm:lambda-poisedness-initial},~$\Lambda_1$ is lower bounded by the right-hand side of~\cref{eq:lambda-poisedness-1}.
    Therefore, we only need to prove that this right-hand side is also a lower bound for~$\Lambda_1$, using the formulation in \cref{lem:lambda-poisedness-initial-simple}.

    For any~$x \in \ball[1](\delta)$, we have
    \begin{equation*}
        \lagp[1](x) \le 1 - \frac{1}{\delta} \sum_{j = m - n}^n x_j \le 1 + \frac{1}{\delta} \sum_{j = m - n}^n \abs{x_j}.
    \end{equation*}
    Therefore,
    \begin{subequations}
        \label{eq:eq:lambda-poisedness-1-proof-1}
        \begin{empheq}[left={\lagp[1](x) \le \empheqlbrace}]{alignat=2}
            & 2,    && \quad \text{if~$n + 2 \le m \le 2n$,}\\
            & 1,    && \quad \text{otherwise.}
        \end{empheq}
    \end{subequations}
    On the other hand,
    \begin{subequations}
        \label{eq:eq:lambda-poisedness-1-proof-2}
        \begin{align}
            \lagp[1](x) &= 1 - \sum_{j = 1}^{m - n - 1} \frac{x_j^2}{\delta^2} - \sum_{j = m - n}^n \frac{x_j}{\delta}\\
                        &\ge 1 - \sum_{j = 1}^{m - n - 1} \frac{\abs{x_j}}{\delta} - \sum_{j = m - n}^n \frac{\abs{x_j}}{\delta}\\
                        &\ge 1 - \frac{\norm{x}_1}{\delta} \ge 0.
        \end{align}
    \end{subequations}
    We conclude the proof by combining~\cref{eq:eq:lambda-poisedness-1-proof-1,eq:eq:lambda-poisedness-1-proof-2} with~\cref{lem:lambda-poisedness-initial-simple}.
\end{proof}

\begin{proposition}
    For any~$m \in \set{n + 2, n + 3, \dots, 2n + 1}$, we have
    \begin{equation}
        \label{eq:lambda-poisedness-2}
        \Lambda_2 = 1 + \sqrt{2n + 1 - m}.
    \end{equation}
\end{proposition}

\begin{proof}
    If~$m = 2n + 1$, \cref{lem:lagrange-polynomials-initial} tells us that~$\lagp[1](x) = 1 - \delta^{-2} \norm{x}_2^2$ for~$x \in \ball[2](\delta)$.
    Therefore, \cref{lem:lambda-poisedness-initial-simple} directly provides the desired result~$\Lambda_2 = 1$.
    We now focus on the case with~$n + 2 \le m < 2n + 1$.

    According to~\cref{thm:lambda-poisedness-initial},~$\Lambda_2$ is lower bounded by the right-hand side of~\cref{eq:lambda-poisedness-2}.
    Therefore, we only need to prove that this right-hand side is also an upper bound for~$\Lambda_2$, using the formulation in \cref{lem:lambda-poisedness-initial-simple}.
    
    For any~$x \in \ball[2](\delta)$, we have
    \begin{subequations}
        \begin{align}
            \lagp[1](x) &= 1 - \frac{1}{\delta^2} \sum_{j = 1}^{m - n - 1} x_j^2 - \frac{1}{\delta} \sum_{j = m - n}^n x_j\\
                        &\ge 1 - \frac{1}{\delta^2} \bigg( \delta^2 - \sum_{j = m - n}^n x_j^2 \bigg) - \frac{1}{\delta} \sum_{j = m - n}^n x_j = \sum_{j = m - n}^n \frac{x_j}{\delta} \bigg( 1 - \frac{x_j}{\delta} \bigg)\\
                        &\ge \min_{y \in \ball[2](1)} \sum_{j = m - n}^n y_j (1 - y_j). \label{eq:lambda-poisedness-initial-2-proof}
        \end{align}
    \end{subequations}
    Let~$y^{\ast} \in \ball[2](1)$ be a minimizer in~\cref{eq:lambda-poisedness-initial-2-proof}.
    \Cref{thm:first-order-necessary-conditions} ensures that there exists a Lagrange multiplier~$\lambda^{\ast} \ge 0$ such that~$1 - 2 y_j^{\ast} + 2 \lambda^{\ast} y_j^{\ast} = 0$ for all~$j \in \set{m - n, \dots, n}$.
    Therefore, the last~$(2n + 1 - m)$ components of~$y^{\ast}$ are equal, and hence,
    \begin{align*}
        \lagp[1](x) \ge \min_{y \in \ball[2](1)} \sum_{j = m - n}^n y_j (1 - y_j)   &= \min_{t \in \R} \set{(2n + 1 - m) t (1 - t) : (2n + 1 - m) t^2 \le 1}\\
                                                                                    &= -1 - \sqrt{2n + 1 - m}.
    \end{align*}

    Furthermore,
    \begin{equation*}
        \lagp[1](x) \le 1 + \sum_{j = m - n}^n \frac{\abs{x_j}}{\delta} \le 1 + \sqrt{2n + 1 - m} \sum_{j = m - n}^n \frac{x_j^2}{\delta^2} \le 1 + \sqrt{2n + 1 - m}.
    \end{equation*}
    Therefore,~$\abs{\lagp[1](x)} \le 1 + \sqrt{2n + 1 - m}$ and hence, according to~\cref{lem:lambda-poisedness-initial-simple,thm:lambda-poisedness-initial}, we have
    \begin{equation*}
        \Lambda_2 = \max_{x \in \ball[2](\delta)} \abs{\lagp[1](x)} = 1 + \sqrt{2n + 1 - m},
    \end{equation*}
    which concludes the proof.
\end{proof}

Before developing the last special value of~$\Lambda_p$ we introduce the following lemma for sake of clarity.

\begin{lemma}
    \label{lem:max-norm-pq}
    For any~$p \ge 1$ and~$q \ge 1$, we have
    \begin{subequations}
        \label{eq:max-norm-pq}
        \begin{empheq}[left={\max\limits_{x \in \ball[q](1)} \norm{x}_p = \empheqlbrace}]{alignat=2}
            & 1,                    && \quad \text{if~$p \ge q$,} \label{eq:max-norm-pq-1}\\
            & n^{\frac{q - p}{pq}}, && \quad \text{otherwise.} \label{eq:max-norm-pq-2}
        \end{empheq}
    \end{subequations}
\end{lemma}

\begin{proof}
    Let us first consider the case where~$p \ge q$.
    For~$x \in \ball[q](1)$, we have~$\norm{x}_p \le 1$, and this bound is attained at the first coordinate vector~$e_1 \in \ball[q](1)$, so that~\cref{eq:max-norm-pq-1} holds.

    We now consider the case where~$p < q$.
    Let~$e \in \R^n$ be the all-one vector,~$r = q/p$, and~$s = r / (r - 1) = q / (q - p)$.
    For~$x \in \ball[q](1)$, define~$y = (\abs{x_1}^p, \abs{x_2}^p, \dots, \abs{x_n}^p)$.
    According to the H{\"{o}}lder inequality, we have
    \begin{equation*}
        \norm{x}_p  = \inner{e, y}^{\frac{1}{p}} \le (\norm{e}_s \norm{y}_r)^{\frac{1}{p}} = n^{\frac{q - p}{pq}} \norm{x}_q \le n^{\frac{q - p}{pq}}.
    \end{equation*}
    Moreover, this bound is attained by~$x = n^{-\frac{1}{q}}e$, which proves~\cref{eq:max-norm-pq-2}.
\end{proof}

\begin{proposition}
    \label{prop:lambda-poisedness-initial-optimal}
    For any~$p \ge 1$, if~$m = 2n + 1$, then
    \begin{equation*}
        \Lambda_p = \max \set[\big]{1, n^{\frac{p - 2}{p}} - 1}.
    \end{equation*}
\end{proposition}

\begin{proof}
    It is clear that
    \begin{equation}
        \label{eq:lambda-poisedness-infty-proof-1}
        \max_{x \in \ball[p](\delta)} \lagp[1](x) = \max_{x \in \ball[p](\delta)} \bigg( 1 - \frac{\norm{x}_2^2}{\delta^2} \bigg) = 1.
    \end{equation}
    Moreover, according to \cref{lem:max-norm-pq}, we have
    \begin{subequations}
        \label{eq:lambda-poisedness-infty-proof-2}
        \begin{empheq}[left={\max\limits_{x \in \ball[p](\delta)} -\lagp[1](x) = \max\limits_{x \in \ball[p](\delta)} \dfrac{\norm{x}_2^2}{\delta^2} - 1 = \empheqlbrace}]{alignat=2}
            & 0,                        && \quad \text{if~$p \le 2$,}\\
            & n^{\frac{p - 2}{p}} - 1,  && \quad \text{otherwise.}
        \end{empheq}
    \end{subequations}
    The desired result is obtained by combining~\cref{eq:lambda-poisedness-infty-proof-1,eq:lambda-poisedness-infty-proof-2} with \cref{lem:lambda-poisedness-initial-simple}.
\end{proof}

It is important to note that the last three proposition overlap in the case~$m = 2n + 1$.
The three of them show that in such a case,~$\Lambda_1 = \Lambda_2 = 1$.
In fact, note that~\cref{prop:lambda-poisedness-initial-optimal} show that~$\Lambda_p = 1$ for all~$p \in [1, 2]$.
Therefore, in this sense, the value~$m = 2n + 1$ is optimal for this interpolation set.
We conjecture that this property remain true for all~$\ell_p$-norms, with~$p \ge 1$, but we do not provide any proof of such statement.
Moreover,~\cite{Powell_2001} propose an extension of~$\mathcal{Z}_m$ in the case~$2n + 1 \le m \le (n + 1)(n + 2) / 2$.
Numerical experiments made by the author showed that the value of~$\Lambda_p$ seems to the least also for~$m = 2n + 1$.
This conjecture is an interesting future research direction.
