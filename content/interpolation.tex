%% contents/interpolation.tex
%% Copyright 2021-2022 Tom M. Ragonneau
%
% This work may be distributed and/or modified under the
% conditions of the LaTeX Project Public License, either version 1.3
% of this license or (at your option) any later version.
% The latest version of this license is in
%   http://www.latex-project.org/lppl.txt
% and version 1.3 or later is part of all distributions of LaTeX
% version 2005/12/01 or later.
%
% This work has the LPPL maintenance status `maintained'.
%
% The Current Maintainer of this work is Tom M. Ragonneau.
\chapter{Interpolation models for \glsfmtlong{dfo}}

We study in this thesis model-based \gls{dfo} methods (see \cref{subsec:model-based-methods}).
These methods necessitate approximating locally the functions involved in the optimization problem by some models, using only function evaluations.
Thoughout this section, we denote by~$\obj$ the real-valued function on~$\R^n$ to be approximated and~$\mathcal{Y} \subseteq \R^n$ the set of points at which this function is to be evaluated.
We also denote~$\mathcal{F}_{n, m}$ the space of real-valued polynomial functions on~$\R^n$ of degree at most~$m \ge 0$.
In this chapter, we aim at finding a function~$\objm \in \mathcal{F}_{n, m}$ for some fixed~$m$ that approximates~$\obj$ on the set~$\mathcal{Y}$.

These models are generally obtained by interpolation or regression.

\begin{equation*}
    \objm(y) = \obj(y), \quad y \in \mathcal{Y}
\end{equation*}

\section{Elementary concepts of multivariate interpolation}
\label{sec:multivariate-interpolation}

\begin{itemize}
    \item Linear models, quadratic models, RBF models, \dots
    \item What is a simplex gradient? (Needed in the introduction)
\end{itemize}

% Perhaps the simplest models we encounter are linear models.
% These models have~$n + 1$ parameters and they may be obtained by interpolation.
% Several model-based methods use linear models, including \gls{cobyla} (see \cref{subsec:cobyla}).

\section{Overview of the Lagrange polynomials}

\section{Poisedness of interpolation sets}

\subsection{Measuring the well poisedness of interpolation sets}

\subsection{Relationship with the conditioning of the interpolation system}

\section{Underdetermined interpolation systems}
\label{sec:underdetermined-interpolation}

\subsection{Least Frobenius norm quadratic models}

\begin{itemize}
    \item Factorization of the symmetric Broyden matrix.
    \item Updating the symmetric Broyden matrix.
    \item Decompositions of the Hessian matrices of the quadratic models.
\end{itemize}

\subsection{Quadratic models based on symmetric Broyden updates}

\begin{itemize}
    \item Explain the name.
    \item $\Lambda$-poisedness.     
    \item Condition number of the interpolation system.
    \item Bound using Lagrange polynomial.
\end{itemize}
