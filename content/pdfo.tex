%% contents/pdfo.tex
%% Copyright 2021-2022 Tom M. Ragonneau
%
% This work may be distributed and/or modified under the
% conditions of the LaTeX Project Public License, either version 1.3
% of this license or (at your option) any later version.
% The latest version of this license is in
%   http://www.latex-project.org/lppl.txt
% and version 1.3 or later is part of all distributions of LaTeX
% version 2005/12/01 or later.
%
% This work has the LPPL maintenance status `maintained'.
%
% The Current Maintainer of this work is Tom M. Ragonneau.
\chapter{Development of the \glsfmttext{pdfo} package}
\label{ch:pdfo}

Powell designed five model-based \gls{dfo} solvers to tackle unconstrained and constrained problems, namely~\gls{cobyla}~\cite{Powell_1994}, \gls{uobyqa}~\cite{Powell_2002}, \gls{newuoa}~\cite{Powell_2006}, \gls{bobyqa}~\cite{Powell_2009}, and \gls{lincoa}.
These solvers were implemented by Powell, with particular attention paid to their numerical stability and algebraic complexity.
Renowned for their robustness and efficiency, these solvers are very appealing to practitioners and widely used in applications, including aeronautical engineering~\cite{Gallard_Etal_2018}, astronomy~\cite{Biviano_Etal_2013,Mamon_Biviano_Boue_2013}, computer vision~\cite{Izadinia_Shan_Seitz_2017}, robotics~\cite{Mombaur_Truong_Laumond_2010}, and statistics~\cite{Bates_Etal_2015}.
However, Powell implemented them in Fortran 77, an old-fashion language that damps the enthusiasm of many users to exploit these solvers in their projects.

In this chapter, we present \gls{pdfo}, a MATLAB and Python package we developed to interface Powell's solvers, which has been downloaded more than 30,000 times\footnote{See \url{https://www.pdfo.net/}.} as of August 2022.
\Cref{sec:pdfo-introduction} presents the motivations for the development of \gls{pdfo}.
We then provide an overview of Powell's five solvers in \cref{sec:powell}.
\Cref{sec:pdfo-core-features} introduces the core features of \gls{pdfo} and its implementation.
Finally, we present in \cref{sec:pdfo-experiments} some numerical experiments on Powell's solvers using \gls{pdfo}, showing the advantage of having all five solvers in one package.
One of these experiments will reveal an interesting behavior of the solvers, namely that \gls{cobyla} performs quite well compared to other solvers on noisy unconstrained problems (see \cref{fig:ppun-50}).
This behavior is intriguing, because \gls{cobyla} is the oldest of these methods and uses the simplest models.

\section{Introduction and motivation}
\label{sec:pdfo-introduction}

There has been considerable demand from both researchers and practitioners for the availability of the five Powell's solvers in more user-friendly languages such as Python, MATLAB, and Julia.
Our aim is to wrap Powell's Fortran code into a package named \gls{pdfo}, which enables users of such languages to call Powell's solvers without any need to deal with the Fortran code.
For each supported language, \gls{pdfo} provides a simple subroutine that can invoke one of Powell's solvers according to the user's request (if any) or according to the type of the problem to solve.
The current release (version 1.2) of \gls{pdfo} supports Python and MATLAB, with more languages to be covered in the future.
The signature of the Python subroutine is consistent with the \texttt{minimize} function of the SciPy optimization library~\cite{Virtanen_Etal_2020};
the signature of the MATLAB subroutine is consistent with the \texttt{fmincon} function of the MATLAB Optimization Toolbox.
The package is cross-platform, available on Linux, macOS, and Windows at once.

\Gls{pdfo} is not the first attempt to facilitate the usage of Powell's solvers in languages other than Fortran.
Various efforts have been made in this direction in response to the continual demands from both researchers and practitioners: Py-BOBYQA~\cite{Cartis_Etal_2019} provides a Python implementation of \gls{bobyqa}; NLopt~\cite{Johnson_2019} includes multi-language interfaces for \gls{cobyla}, \gls{newuoa}, and \gls{bobyqa}; minqa~\cite{Bates_Etal_2014} wraps \gls{uobyqa}, \gls{newuoa}, and \gls{bobyqa} in R; SciPy~\cite{Virtanen_Etal_2020} makes \gls{cobyla} available in Python under its optimization library.
Nevertheless, \gls{pdfo} has several features that distinguish it from others.

\paragraph{Comprehensiveness}

To the best of our knowledge, \gls{pdfo} is the only package that provides all of \gls{cobyla}, \gls{uobyqa}, \gls{newuoa}, \gls{bobyqa}, and \gls{lincoa} with a uniform interface.
In addition to homogenizing the usage, such an interface eases the comparison between these solvers in case multiple of them can tackle a given problem.
Doing so, we may gain insights that cannot be obtained otherwise into the behavior of the solvers, as will be illustrated in~\cref{sec:pdfo-experiments}.

\paragraph{Solver selection}

When using \gls{pdfo}, the user can specifically invoke one of Powell's solvers; nevertheless, if the user does not specify any solver, \gls{pdfo} will automatically select a solver according to the given problem.
The selection takes into consideration the performance of the solvers on the CUTEst problem set~\cite{Gould_Orban_Toint_2015}.
Interestingly, it turns out that the solver with the best performance may not be the most intuitive one.
For example, \gls{newuoa} is not always the best choice for solving an unconstrained problem.
This will be elaborated on in~\cref{subsec:solver-selection}.

\paragraph{Code patching}

During the development of \gls{pdfo}, we spotted in the original Fortran code some bugs, which led, for example, to infinite cycling or segmentation faults on some ill-conditioned problems.
The bugs have been patched in \gls{pdfo}.
Nevertheless, we provide an option that can enforce the package to use the original code of Powell without the patches, which is not recommended except for research.
In addition, \gls{pdfo} provides \gls{cobyla} in double precision, whereas Powell used single precision when he implemented it in the 1990s.
See~\cref{subsec:bug-corrections} for details.

\paragraph{Fault tolerance}

\Gls{pdfo} takes care of failures in the evaluations of the objective or constraint functions when NaN or infinite values are returned.
In case of such failures, \gls{pdfo} will not exit but try to progress.
Moreover, \gls{pdfo} ensures that the returned solution is not a point where the evaluation fails, while the original code of Powell may return a point whose objective function value is numerically NaN.

\paragraph{Problem preprocessing}

\gls{pdfo} preprocesses the inputs to simplify the problem and reformulate it to meet the requirements of Powell's solvers.
For instance, if the problem has linear constraints~$A_{\scriptscriptstyle\ieq} \iter = b_{\scriptscriptstyle\ieq}$, \gls{pdfo} can rewrite it into a problem on the null space of~$A_{\scriptscriptstyle\ieq}$, eliminating such constraints and reducing the dimension.
Another example is that the starting point of a linearly-constrained problem is projected onto the feasible region because \gls{lincoa} needs a feasible starting point to work properly.

\paragraph{Additional options}

\gls{pdfo} includes options for the user to control the solvers in some manners that are useful in practice.
For example, the user can request \gls{pdfo} to scale the problem according to the bounds of the variables before solving it.

\section{Overview of Powell's \glsentrylong{dfo} methods}
\label{sec:powell}

We consider the optimization problem
\begin{subequations}
    \label{eq:problem-pdfo}
    \begin{align}
        \min        & \quad \obj(\iter)\\
        \text{s.t.} & \quad \con{i}(\iter) \le 0, ~ i \in \iub,\\
                    & \quad \iter \in \R^n, \nonumber
    \end{align}
\end{subequations}
where the objective and constraint functions~$\obj$ and~$\con{i}$, with~$i \in \iub$, are real-valued functions on~$\R^n$, and where the set of indices~$\iub$ is finite (perhaps empty).
Note that this problem is exactly the problem~\cref{eq:problem-introduction} studied in \cref{ch:introduction} by setting~$\ieq = \emptyset$.
In general, any problem of the form~\cref{eq:problem-introduction} can be reformulated into~\cref{eq:problem-pdfo} by considering the equalities as two inequalities.

We assume that derivatives of~$\obj$ and~$\con{i}$, for~$i \in \iub$, are unavailable.
Powell published his first \gls{dfo} algorithm based on conjugate directions in \citeyear{Powell_1964}\footnote{According to Google Scholar, this is Powell's second published paper and also the second most cited work. The earliest and meanwhile most cited one is his paper on the \gls{dfp} method~\cite{Fletcher_Powell_1963}, co-authored with Fletcher and published in 1963.}~\cite{Powell_1964}.
His code for this algorithm is contained in the HSL Mathematical Software Library~\cite{HSL} as the subroutine \texttt{VA24}.
It is not included in \gls{pdfo} because the code is not in the public domain, although open-source implementations are available~(see~\cite[Fn.~4]{Conn_Scheinberg_Toint_1997b}).

From the 1990s to the final days of his career, Powell developed five model-based \gls{dfo} algorithms to solve the problem~\cref{eq:problem-pdfo} in different cases, namely \gls{cobyla}~\cite{Powell_1994} for nonlinearly-constrained problems, \gls{uobyqa}~\cite{Powell_2002} and \gls{newuoa}~\cite{Powell_2006} for unconstrained problems, \gls{bobyqa}~\cite{Powell_2009} for bound constrained problems, and~\gls{lincoa} for linearly constrained problems.
In addition, Powell implemented these algorithms into Fortran solvers and made the code publicly available.
The solvers constitute the cornerstones of \gls{pdfo}.

\subsection{A sketch of the algorithms}

Powell's model-based \gls{dfo} algorithms are trust-region methods.
At the~$k$th iteration, the algorithms construct a linear or quadratic trust-region model~$\objm[k]$ for the objective function~$\obj$ by underdetermined interpolation on a finite set~$\xpt[k] \subseteq \R^n$ updated along the iterations, based on the symmetric Broydem update (see \cref{subsec:symmetric-broyden-updates}).
\Gls{cobyla} models the constraints by interpolants on~$\xpt[k]$ as well, which we denote~$\conm[k]{i}$\nomenclature[Fg]{$\conm{i}$}{Quadratic model of~$\con{i}$, with~$i \in \iub \cup \ieq$}, for~$i \in \iub$.
Instead of repeating Powell's description of these algorithms, we provide a sketch of them in the sequel.

In all the five algorithms, the~$k$th iteration places the trust-region center~$\iter[k]$ at the \enquote{best} point where the objective function and constraints have been evaluated so far.
Such a point is selected according to the objective function or a merit function that takes the constraints into account.
After choosing the trust-region center~$\iter[k]$, with the trust-region model~$\objm[k]$ constructed, a trial point is then obtained by solving approximately the trust-region subproblem
\begin{subequations}
    \label{eq:powell-trust-region}
    \begin{align}
        \min        & \quad \objm[k](\iter)\\
        \text{s.t.} & \quad \con{i}(\iter) \le 0, ~ i \in \iub, \label{eq:powell-trust-region-ub}\\
                    & \quad \norm{\iter - \iter[k]} \le \rad[k],\\
                    & \quad \iter \in \R^n. \nonumber
    \end{align}
\end{subequations}
where~$\rad[k]$ is the trust-region radius.
Only \gls{cobyqa} differs from this scheme, and replaces~\cref{eq:powell-trust-region-ub} by
\begin{equation*}
    \conm[k]{i}(\iter) \le 0, ~ i \in \iub.
\end{equation*}
This is because for all other solvers, either~$\iub = \emptyset$ for \gls{uobyqa} and \gls{newuoa}, the constraints~$\set{\con{i}}_{i \in \ieq}$ are bound constraints for \gls{bobyqa}, or linear constraints for \gls{lincoa}.
Such simple constraints do not necessitate to be modeled.

\subsection{The \glsfmttext{cobyla} method}
\label{subsec:cobyla}

\index{COBYLA@\glsfmttext{cobyla}|(}Powell released the \gls{cobyla} solver in May 1992 and published the corresponding paper~\cite{Powell_1994} in \citedate{Powell_1994}.
The solver is named after \emph{\glsdesc{cobyla}}.
It aims to solve the optimization problem~\cref{eq:problem-pdfo} whenever the constraint functions~$\con{i}$, for~$i \in \iub$, are nonlinear functions whose derivatives are unknown.
In other words, only function values of the constraint functions are accessible.

At the~$k$th iteration, \gls{cobyla} models the objective and the constraint functions with linear interpolants on the interpolation set~$\xpt[k] \subseteq \R^n$, which consists of~$n + 1$ points that are updated along the iterations.
In this context, the interpolation set~$\xpt[k]$ is poised if and only if the volume of the~$n$-simplex engendered by~$\xpt[k]$ is nonzero (see \cref{subsec:poisedness} for details).

Once the linear models~$\conm[k]{i}$ of the constraint functions~$\con{i}$, for~$i \in \iub$, are built, the trust-region subproblem
\begin{subequations}
    \label{eq:cobyla-subproblem}
    \begin{align}
        \min        & \quad \objm[k](\iter)\\
        \text{s.t.} & \quad \conm[k]{i}(\iter) \le 0, ~ i \in \iub, \label{eq:cobyla-subproblem-ub}\\
                    & \quad \norm{\iter - \iter[k]} \le \rad[k],\\
                    & \quad \iter \in \R^n, \nonumber
    \end{align}
\end{subequations}
is handled in the following way.
Solving the problems~\cref{eq:cobyla-subproblem} sequentially by replacing the trust-region radius~$\rad[k]$ with a constant continuously increasing from zero to~$\rad[k]$ would generate a piecewise linear path from~$\iter[k]$ to the solution of the subproblem.
The strategy of \gls{cobyla} is to compute this path by updating the active sets of the constraint models.
However, the trust-region constraint~$\norm{\iter - \iter[k]} \le \rad[k]$ and the region~\cref{eq:cobyla-subproblem-ub} may contradict each others, in which case the trial point is chosen to solve approximately
\begin{align*}
    \min        & \quad \max_{i \in \iub} @@ \posp{\conm[k]{i}(\iter)}\\
    \text{s.t.} & \quad \norm{\iter - \iter[k]} \le \rad[k],\\
                & \quad \iter \in \R^n,
\end{align*}
\nomenclature[Oa]{$\posp{\cdot}$}{Positive-part operator}%
where~$\posp{\cdot}$ takes the positive-part of a given number.
In doing so, the method attempts to reduce the~$\ell_{\infty}$-violation of the constraint models while ensuring that the trial point lies in the trust region.\index{COBYLA@\glsfmttext{cobyla}|)}

\subsection{The \glsfmttext{uobyqa} method}
\label{subsec:uobyqa}

\index{UOBYQA@\glsfmttext{uobyqa}|(}Later on, in \citedate{Powell_2002}, Powell developed \gls{uobyqa}~\cite{Powell_2002}, named after \emph{\glsdesc{uobyqa}}.
It aims at solving the nonlinear optimization problem~\cref{eq:problem-pdfo} in the unconstrained case, i.e., when~$\iub = \emptyset$.
To do so, at each iteration, it models the objective function with a quadratic obtained by fully-determined interpolation on the set~$\xpt[k] \subseteq \R^n$ containing~$N = (n + 1)(n + 2) / 2$ points.
The set~$\xpt[k + 1]$ differs from~$\xpt[k]$ of at most one point.
During a classical iteration, a trial point, i.e., an approximate solution of the trust-region subproblem~\cref{eq:powell-trust-region} replaces an interpolation of~$\xpt[k]$.
As we already mentioned, it is essential to maintain some geometrical properties of~$\xpt[k]$ to ensure the existence and uniqueness of the models from a computational viewpoint.
Hence, \gls{uobyqa} may undertake geometry-improving steps whenever the models seem not to be accurate, led by a remarkable result pointed out in~\cite{Powell_2001}.
It states that given a point~$y \in \xpt[k]$ to remove from the interpolation set, its most suitable substitute to build~$\xpt[k + 1]$ solves the subproblem
\begin{subequations}
    \label{eq:biglag}
    \begin{align}
        \max        & \quad \abs{\lagp[y](x)}\\
        \text{s.t.} & \quad \norm{x - \iter} \le \rad[k],\\
                    & \quad x \in \R^n,
    \end{align}
\end{subequations}
where~$\lagp[y] \colon \R^n \to \R$ denotes the Lagrange function associated with~$y$ (see \cref{sec:lagrange-polynomials} for details).
Since \gls{uobyqa} requires only a rough solution of the geometry-improving subproblem~\cref{eq:biglag}, Powell developed an algorithm that requires only~$\bigo(n^2)$ operations, based on an estimation of~$\abs{\lagp[y](\cdot)}$.
Once such a point is calculated, the solver continues with a classical trust-region step, and the subproblem~\cref{eq:powell-trust-region} is solved with the Mor{\'{e}}-Sorensen algorithm~\cite{More_Sorensen_1983}, as it controls the accuracy of the solution.\index{UOBYQA@\glsfmttext{uobyqa}|)}

\subsection{The \glsfmttext{newuoa}, \glsfmttext{bobyqa}, and \glsfmttext{lincoa} methods}
\label{subsec:newuoa-bobyqa-lincoa}

\index{NEWUOA@\glsfmttext{newuoa}|(}\index{BOBYQA@\glsfmttext{bobyqa}|(}\index{LINCOA@\glsfmttext{lincoa}|(}The major flaw of \gls{uobyqa} is the number of required interpolation points, which can become prohibitively huge when~$n$ increases.
Therefore, Powell designed a mechanism to define quadratic models that require fewer interpolation points~\cite{Powell_2004a}.
These models, based on the symmetric Broyden update, are presented in \cref{subsec:symmetric-broyden-updates}.
They only require the number of points in~$\xpt[k]$ to be at least~$n + 2$ and at most~$(n + 1)(n + 2) / 2$.
As in the fully-determined case, the geometry of the interpolation set plays a crucial role, as it influences the accuracy of the quadratic models and the geometry-improving steps~\cref{eq:biglag} should also be entertained.

The last three \gls{dfo} solvers of Powell are \gls{newuoa}~\cite{Powell_2006, Powell_2008}, \gls{bobyqa}~\cite{Powell_2009}, and~\gls{lincoa}.
\Gls{bobyqa} and \gls{lincoa} are named respectively after \emph{\glsdesc{bobyqa}} and \emph{\glsdesc{lincoa}}, but the meaning of the acronym \gls{newuoa} is not known (even though most people agree on \emph{\glsdesc{newuoa}}).
Powell never published a paper introducing \gls{lincoa}, and~\cite{Powell_2015} discusses only the resolution of its trust-region subproblem.
As their names suggest, they aim to solve unconstrained, bound-constrained, and linearly-constrained problems, using quadratic models of the objective function.
All three use the underdetermined interpolation technique described above to build the quadratic models so that \gls{newuoa} is more suitable than \gls{uobyqa} for solving problems with a relatively high dimension.
The feasible region~$\set{\iter \in \R^n : \con{i}(\iter) \le 0, ~ i \in \iub}$ corresponds to the whole space for \gls{newuoa}, a box for \gls{bobyqa}, and a polyhedron for \gls{lincoa}.
A subtlety of \gls{bobyqa} is that the constraints are always respected, for each iterate and each point in~$\xpt[k]$.
Therefore, the geometry-improving subproblem~\cref{eq:biglag} should be adapted to incorporate the bound constraints, which makes its resolution more difficult.
Moreover, as we already mentioned, at most one point of the interpolation set is altered at each iteration, which leads to an at-most rank-$2$ update of the coefficient matrix of the \gls{kkt} system of interpolation.
This remark suggests that it can be much more efficient to update such a matrix instead of computing it from scratch at each iteration.
Powell derived an updating formula in~\cite{Powell_2004b} that requires only~$\bigo(N^2)$ operations instead of~$\mathcal{O}(N^3)$ if the computation was made from scratch with no loss of accuracy, where~$N$ denotes the number of interpolation points.
Details are given in \cref{subsec:implementation-symmetric-broyden-update}.
When it comes to solving the trust-region subproblems~\cref{eq:powell-trust-region}, \gls{newuoa} employs the Steihaug-Toint \gls{tcg} algorithm~\cite{Steihaug_1983,Toint_1981}, and \gls{bobyqa} and \gls{lincoa} use an active-set variation of it.
However, the trust-region subproblem solver of \gls{bobyqa} always respects the bounds, while the solver of \gls{lincoa} may visit points lying outside of the polyhedron of the constraints and may even return points that are infeasible.\index{LINCOA@\glsfmttext{lincoa}|)}\index{BOBYQA@\glsfmttext{bobyqa}|)}\index{NEWUOA@\glsfmttext{newuoa}|)}

\section{Core features of the \glsfmttext{pdfo} package}
\label{sec:pdfo-core-features}

\index{PDFO@\glsfmttext{pdfo}|(}Powell implemented all five methods in Fortran in a very robust and efficient manner.
However, fewer and fewer people are using Fortran in our present-day world, and Fortran has quite old standards.
The authors developed therefore \gls{pdfo}\footnote{Available at \url{https://www.pdfo.net/}.}, named after \emph{\glsdesc{pdfo}}.
It is a cross-platform package providing MATLAB and Python interfaces for using all five Powell's \gls{dfo} solvers, available for Linux, macOS, and Windows.
\Gls{pdfo} does not reimplement Powell's solvers but rather links the modern languages MATLAB and Python with the Fortran source code.
At a low level, it uses F2PY~\cite{Peterson_2009} to interface Python with Fortran and MEX to interface it with MATLAB.

\subsection{Main structure of the package}

The philosophy of \gls{pdfo} is simple: providing to users a single function to solve a \gls{dfo} problem.
It takes for input an optimization problem of the form
\begin{subequations}
    \label{eq:pdfo}
    \begin{align}
        \min        & \quad \obj(\iter)\\
        \text{s.t.} & \quad \xl \le \iter \le \xu, \label{eq:pdfo-bds}\\
                    & \quad A_{\scriptscriptstyle\iub} \iter \le b_{\scriptscriptstyle\iub}, \label{eq:pdfo-lub}\\
                    & \quad A_{\scriptscriptstyle\ieq} \iter = b_{\scriptscriptstyle\ieq}, \label{eq:pdfo-leq}\\
                    & \quad \con{i}(\iter) \le 0, ~ i \in \iub, \label{eq:pdfo-nlub}\\
                    & \quad \con{i}(\iter) = 0, ~ i \in \ieq, \label{eq:pdfo-nleq}\\
                    & \quad \iter \in \R^n, \nonumber
    \end{align}
\end{subequations}
where~$\xl, \xu \in (\R \cup \set{\pm \infty})^n$,~$A_{\scriptscriptstyle\iub}$ and~$A_{\scriptscriptstyle\ieq}$ are real matrices,~$b _{\iub}$ and~$b _{\ieq}$ are real vectors, and~$\con{i}$, for~$i \in \iub \cup \ieq$, is a real-valued function.
The problem~\cref{eq:pdfo} covers every possible case, since all matrices, vectors and functions can be set to zero from a mathematical viewpoint and \gls{pdfo} does not require the constraints~\cref{eq:pdfo-bds,eq:pdfo-lub,eq:pdfo-leq,eq:pdfo-nlub,eq:pdfo-nleq} to be provided.
A broad example of use is shown in MATLAB in \cref{lst:minimum-example-matlab} and Python in \cref{lst:minimum-example-python}, where variable names have clear correspondences with the problem~\cref{eq:pdfo}.
For both MATLAB and Python, \gls{pdfo} returns the best point calculated (evaluated via a merit function) and (optionally) the corresponding optimal value, together with different fields describing the backend calculations and their behaviors.

\begin{lstmatlab}[%
    caption=An elementary example of \gls{pdfo} in MATLAB,
    label=lst:minimum-example-matlab,
]
    x = pdfo(@fun, x0, A, b, Aeq, beq, lb, ub, @nonlcon);

    function fx = fun(x)
    ...
    return
    end

    function [c, ceq] = nonlcon(x)
    ...
    return
    end
\end{lstmatlab}

\begin{lstpython}[%
    caption=An elementary example of \gls{pdfo} in Python,
    label=lst:minimum-example-python,
]
    import numpy as np
    from pdfo import *

    def fun(x):
        return ...
    
    def cub(x):
        return ...
    
    def ceq(x):
        return ...
    
    bounds = Bounds(lb, ub)
    constraints = [
        LinearConstraint(A, -np.inf, b),
        LinearConstraint(Aeq, beq, beq),
        NonlinearConstraint(cub, -np.inf, 0.0),
        NonlinearConstraint(ceq, 0.0, 0.0),
    ]

    res = pdfo(fun, x0, bounds=bounds, constraints=constraints)
\end{lstpython}

\subsection{Problem preprocessing}

The package \gls{pdfo} preprocesses the arguments provided by the user, detects the type of the problem, and then invokes the Powell's solver that best matches the given problem.
The initial preprocessing of the arguments include handling their internal types together with some programming-related procedures to allow as much freedom in the problem definition as possible, in order to make the use of \gls{pdfo} as easy as possible.
More importantly, \gls{pdfo} preprocesses the constraints provided to generate a problem as simple as possible.
For instance, all linear constraints in~\cref{eq:pdfo-lub,eq:pdfo-leq} that are satisfied for every point in~$\R^n$ are removed and obvious infeasibility in the constraints~\cref{eq:pdfo-bds,eq:pdfo-lub,eq:pdfo-leq} are detected.
Another noticeable preprocessing of the constraints made by \gls{pdfo} is the treatment of the linear equality constraints~\cref{eq:pdfo-leq}.
As long as these constraints are consistent, they define a subspace of~$\R^n$ of lower dimension, and \gls{pdfo} takes into account this property to generate a new~$(n - \rank A_{\scriptscriptstyle\ieq})$\nomenclature[Om]{$\rank$}{Rank operator}-dimensional problem that is exactly equivalent to~\cref{eq:pdfo}, using the QR factorization of~$A_{\scriptscriptstyle\ieq}$.

A crucial point of \gls{bobyqa} and \gls{lincoa} is that they require the initial guess to be feasible so that \gls{pdfo} attempts to project the provided initial guess onto the feasible set (\gls{lincoa} would otherwise increase the coefficients of the right-hand side of the linear constraints to make the initial guess feasible).

\subsection{Automatic selection of the solver}
\label{subsec:solver-selection}

Another main feature of \gls{pdfo} is its solver selection mechanism.
When a constrained problem is received, the selected solver is the one corresponding to the most general constraint provided.
For example, when \gls{pdfo} receives a problem that admits both bound constraints~\cref{eq:pdfo-bds} and linear constraints~\cref{eq:pdfo-lub,eq:pdfo-leq}, \gls{lincoa} will be chosen.
It is possible on some examples that \gls{lincoa} gives better results than \gls{bobyqa} on bound-constrained problems.
This is likely because \gls{bobyqa} is a feasible method, while \gls{lincoa} may visit infeasible points (but on an engineering problem, these points may be unassessable).
At last, when \gls{pdfo} receives an unconstrained problem, it will attempt to solve it with \gls{uobyqa} when its size is reasonable ($2 \le n \le 8$, say), and with \gls{newuoa} otherwise.
We note that \gls{uobyqa} cannot handle problems with univariate objective functions.

\subsection{Bug corrections in the Fortran source files}
\label{subsec:bug-corrections}

The authors wanted to keep the source code of all solvers in its original states but introduced some minor revisions and corrections.
A new parameter has been added to allow each solver to exit whenever a target value has been reached, and a flag of termination has been included in the returned values.
A revision has also been made to the source code of \gls{cobyla}.
In the original version, trial points may be discarded prematurely before the update of the penalty coefficient of the merit function.
This has been revised in \gls{pdfo}.
The authors also detected some minor bugs on failure exits, for which some returned values might not have been updated.
Although very rarely, \gls{lincoa} might encounter infinite cycling, even on well-conditioned problems.
These bugs have been patched.
\Gls{pdfo} also handle more carefully ill-conditioned problems, for which NaN values (resulting, e.g., from a division by zero) may occur.
These values might cause infinite cycling or segmentation faults in the original code.
In the early stage of \gls{pdfo}, such errors occurred on the CUTEst problems~\cite{Gould_Orban_Toint_2015} DANWOODLS or GAUSS1LS, for example.
NaN values detected in the objective or constraint functions are managed using extreme barriers, but NaN values encountered in the variables internal to the Fortran code result in early exits.
Besides, when interfacing Fortran with MATLAB and Python, rounding errors occured in the problem's variables, leading to failures in extreme cases.
For example, when using \gls{pdfo}, the user may provided initial and final trust-region radii~$\rad[0]$ and~$\rad[\infty]$ (respectively set to~$1$ and~$10^{-6}$ by default).
If these values are chosen to be very close, although the condition~$\rad[0] \le \rad[\infty]$ is satisfied in the MATLAB or Python code, the Fortran code may receive perturbed values with~$\rad[0] > \rad[\infty]$, leading to failure exit in the original code.
Therefore, the conditions required by Powell are ensured in the Fortran code directly.\index{PDFO@\glsfmttext{pdfo}|)}

\section{Numerical experiments}
\label{sec:pdfo-experiments}

\subsection{Comparison on the CUTEst library}

We make a comparison of the \gls{pdfo}'s solvers on different problems from the CUTEst library~\cite{Gould_Orban_Toint_2015}.
Performance profiles~\cite{Dolan_More_2002,More_Wild_2009} on unconstrained problem of dimensions at most~$10$ and~$50$ are provided respectively in \cref{fig:ppu-10,fig:ppu-50}.
For details on performance profiles, see \cref{sec:benchmarking-tools}.
We can observe an expected behavior; \gls{uobyqa} performs better on small problems than all the others, as it is based on quadratic models obtained by fully-determined interpolation, and the performances of \gls{cobyla} decrease with the dimension rise, as it uses only linear models to approximate the problem locally.

\begin{figure}[ht]
    \begin{subfigure}[b]{0.49\textwidth}
        \centering
        \drawperformanceprofiles{{"NEWUOA","BOBYQA","LINCOA","COBYLA","UOBYQA"}}{plain-1-10-perf-newuoa-bobyqa-lincoa-cobyla-uobyqa-u.csv}{4}
        \caption{Dimension at most~$10$}
        \label{fig:ppu-10}
    \end{subfigure}
    \hfill
    \begin{subfigure}[b]{0.49\textwidth}
        \centering
        \drawperformanceprofiles{{"NEWUOA","BOBYQA","LINCOA","COBYLA"}}{plain-1-50-perf-newuoa-bobyqa-lincoa-cobyla-uobyqa-u.csv}{4}
        \caption{Dimension at most~$50$}
        \label{fig:ppu-50}
    \end{subfigure}
    \caption{Performance profiles on unconstrained problems with~$\tau = 10^{-4}$}
\end{figure}

Consider however the following experiment.
Given a smooth objective function~$\obj$, assume that the value received by the solvers is
\begin{equation*}
    \tilde{\obj}(\iter) \eqdef [1 + \epsilon(\iter)] \obj(\iter),
\end{equation*}
where~$\epsilon(x)$ is a random variable that follows a standard normal distribution~$N(0, \sigma^2)$, and~$\sigma \ge 0$ is a given noise level.
\Cref{fig:ppun-50} presents the performance profiles on the same unconstrained problems of dimension at most~$50$ from the CUTEst library as the previous experiment by randomizing the objective functions as~$\tilde{\obj}$ with~$\sigma = 10^{-2}$.
Because of the stochastic behavior of the experiment, the convergence test~\cref{eq:convergence-test-profiles-unconstrained} needs to be adapted.
Each problem is solved~$10$ times by each solver, the objective value considered at each iteration is the average value for all runs, and the optimal value of a given problem is decided as follows.
It is the least value reached by the solvers for every run on the noised variation of the problem and by all the solvers on the noise-free original problem.
In doing so, one should expect a decrease in the proportion of problems solved when compared with the previous experiment.

\begin{figure}[ht]
    \centering
    \begin{subfigure}[b]{0.49\textwidth}
        \centering
        \drawperformanceprofiles{{"NEWUOA","BOBYQA","LINCOA","COBYLA"}}{noisy-1-50-2-perf-newuoa-bobyqa-lincoa-cobyla-u.csv}{1}
        \caption{Tolerance~$\tau = 10^{-1}$}
    \end{subfigure}
    \hfill
    \begin{subfigure}[b]{0.49\textwidth}
        \centering
        \drawperformanceprofiles{{"NEWUOA","BOBYQA","LINCOA","COBYLA"}}{noisy-1-50-2-perf-newuoa-bobyqa-lincoa-cobyla-u.csv}{2}
        \caption{Tolerance~$\tau = 10^{-2}$}
    \end{subfigure}
    \caption{Performance profiles on noisy problems}
    \label{fig:ppun-50}
\end{figure}

We observe, however, a interesting behavior of \gls{cobyla} in this experiment, as it defeats all other solvers on unconstrained problems even though it is not defined for such kind of problem and uses the simplest models.
It seems that the linear models of \gls{cobyla} are, in some sense, less sensitive to Gaussian noise, but the authors did not derive a theory for this behavior as of today.

\subsection{An example of hyperparameter tuning problem}

We now consider the more practical problem of the hyperparameter tuning of a \gls{svm}.
The model we consider is a~$C$-SVM~\cite{Chang_Lin_2011} for binary classification problems with an \gls{rbf} kernel, admitting two hyperparameters: a regularization parameter~$C > 0$ and a kernel parameter~$\gamma > 0$.
We want to compare the performance of \gls{pdfo} with a prominent Bayesian optimization method and a \gls{rs} method.
To this end, we use the Python package hyperopt~\cite{Bergstra_Yamins_Cox_2013} for solving the optimization problems, which provides both the \gls{tpe} and the \gls{rs} methods.
Our experiments are based on binary classification problems from the LIBSVM datasets\footnote{Available at \url{https://www.csie.ntu.edu.tw/~cjlin/libsvmtools/datasets/}.}.
A description of the datasets employed is provided in \cref{tab:htdata}.

\begin{table}[ht]
    \caption{Considered LIBSVM datasets description}
    \label{tab:htdata}
    \centering
    \begin{tabular}{ccS[table-parse-only]S[table-parse-only]}
        \toprule
        Dataset~$\mathcal{D}$   & Attribute characteristic  & {Dimension~$d$}   & {Dataset size}\\
        \midrule
        splice                  & $[-1, 1]$, scaled         & 60                & 1000\\
        svmguide1               & $[-1, 1]$, scaled         & 4                 & 3088\\
        svmguide3               & $[-1, 1]$, scaled         & 21                & 1242\\
        ijcnn1                  & $[-1, 1]$                 & 22                & 49990\\
        \bottomrule
    \end{tabular}
\end{table}
\nomenclature[Sb]{$[a, b]$}{Closed set~$\set{\iter \in \R : a \le \iter \le b}$ with~$a \le b$}%

The problem we consider is as follows.
A dataset~$\mathcal{D} \subseteq [-1, 1]^d$ from \cref{tab:htdata} is randomly split into a training dataset~$\mathcal{T}_1$, admitting approximately~$70\%$ of the data, and a testing dataset~$\mathcal{T}_2$, with~$\mathcal{D} = \mathcal{T}_1 \cup \mathcal{T}_2$.
We want to maximize the~$5$-fold \gls{auc} validation score of the \gls{svm} trained on~$\mathcal{T}_1$ with respect to the hyperparameters~$C$ and~$\gamma$.
The \gls{auc} score, a real number in~$[0, 1]$, measures the area underneath the \gls{roc} curve, a graph representing the performance of a binary classification model.
This curve plots the true positive classification rate with respect to the false positive classification rate at different classification thresholds.
For details on the~$5$-fold cross-validation of an \gls{auc}, see \cref{subsec:machine-learning}.
Such an experiment lies in the \gls{dfo} context.
The numerical results for this experiment are provided in \cref{tab:splice,tab:svmguide1,tab:svmguide3,tab:ijcnn1}.

\begin{table}[!ht]
    \caption{Hyperparameter tuning problem on the dataset \enquote{splice}}
    \label{tab:splice}
    \centering
    \begin{tabular}{cS[table-parse-only]SSS}
        \toprule
        solver      & {No.\ eval.}  & {AUC Score ($10^{-1}$)}   & {Accuracy ($10^{-1}$)}    & {Execution time (\si{\second})}\\
        \midrule
        \gls{pdfo}  & 65            & 9.568                     & 9.933                     & 3.697\\
        \gls{rs}    & 100           & 6.409                     & 5.300                     & 4.635\\
        \gls{rs}    & 200           & 7.880                     & 5.300                     & 9.244\\
        \gls{rs}    & 300           & 7.880                     & 5.300                     & 13.763\\
        \gls{tpe}   & 100           & 5.000                     & 5.033                     & 4.889\\
        \gls{tpe}   & 300           & 7.736                     & 5.300                     & 15.726\\
        \bottomrule
    \end{tabular}
\end{table}

\begin{table}[!ht]
    \caption{Hyperparameter tuning problem on the dataset \enquote{svmguide1}}
    \label{tab:svmguide1}
    \centering
    \begin{tabular}{cS[table-parse-only]SSS}
        \toprule
        Solver      & {No.\ eval.}  & {AUC Score ($10^{-1}$)}   & {Accuracy ($10^{-1}$)}    & {Execution time (\si{\second})}\\
        \midrule
        \gls{pdfo}  & 68            & 9.966                     & 9.730                     & 4.906\\
        \gls{rs}    & 100           & 9.966                     & 9.676                     & 16.178\\
        \gls{rs}    & 200           & 9.966                     & 9.676                     & 32.914\\
        \gls{rs}    & 300           & 9.966                     & 9.676                     & 48.404\\
        \gls{tpe}   & 100           & 9.966                     & 9.720                     & 13.057\\
        \gls{tpe}   & 300           & 9.966                     & 9.720                     & 33.392\\
        \bottomrule
    \end{tabular}
\end{table}

\begin{table}[!ht]
    \caption{Hyperparameter tuning problem on the dataset \enquote{svmguide3}}
    \label{tab:svmguide3}
    \centering
    \begin{tabular}{cS[table-parse-only]SSS}
        \toprule
        Solver      & {No.\ eval.}  & {AUC Score ($10^{-1}$)}   & {Accuracy ($10^{-1}$)}    & {Execution time (\si{\second})}\\
        \midrule
        \gls{pdfo}  & 68            & 8.241                     & 8.016                     & 2.793\\
        \gls{rs}    & 100           & 8.025                     & 7.882                     & 4.233\\
        \gls{rs}    & 200           & 8.141                     & 7.775                     & 8.308\\
        \gls{rs}    & 300           & 8.141                     & 7.775                     & 12.414\\
        \gls{tpe}   & 100           & 7.774                     & 7.453                     & 4.197\\
        \gls{tpe}   & 300           & 8.106                     & 7.989                     & 12.912\\
        \bottomrule
    \end{tabular}
\end{table}

\begin{table}[!ht]
    \caption{Hyperparameter tuning problem on the dataset \enquote{ijcnn1}}
    \label{tab:ijcnn1}
    \centering
    \begin{tabular}{cS[table-parse-only]SSS}
        \toprule
        Solver      & {No.\ eval.}  & {AUC Score ($10^{-1}$)}   & {Accuracy ($10^{-1}$)}    & {Execution time (\SI{}[10^3]{\second})}\\
        \midrule
        \gls{pdfo}  & 59            & 9.940                     & 9.819                     & 1.892\\
        \gls{rs}    & 100           & 9.886                     & 9.773                     & 4.435\\
        \gls{rs}    & 200           & 9.886                     & 9.773                     & 9.146\\
        \gls{rs}    & 300           & 9.886                     & 9.773                     & 13.251\\
        \gls{tpe}   & 100           & 9.891                     & 9.791                     & 4.426\\
        \gls{tpe}   & 300           & 9.896                     & 9.786                     & 12.552\\
        \bottomrule
    \end{tabular}
\end{table}

The \gls{auc} scores and accuracies presented in the tables correspond to the ones computed on~$\mathcal{T}_2$ with an \gls{svm} trained on~$\mathcal{T}_1$, with the tuned parameters~$C$ and~$\gamma$.
In a nutshell, it globally shows that the numerical performances of \gls{pdfo} against the two classical approaches are very similar, but the computations require much fewer \gls{auc} evaluations, and hence, much less computation time.
This behavior is particularly visible on the dataset \enquote{ijcnn1} in \cref{tab:ijcnn1}, as the size of this dataset is much larger than the others.
Thus, we can conclude that \gls{pdfo} performed better than the package hyperopt on these problems, even though the final numerical results are mostly similar.

\section{Summary and remarks}

\todo[noline]{Write the conclusion for \cref{ch:pdfo}}
