%% contents/conclusion.tex
%% Copyright 2021-2022 Tom M. Ragonneau
%
% This work may be distributed and/or modified under the
% conditions of the LaTeX Project Public License, either version 1.3
% of this license or (at your option) any later version.
% The latest version of this license is in
%   http://www.latex-project.org/lppl.txt
% and version 1.3 or later is part of all distributions of LaTeX
% version 2005/12/01 or later.
%
% This work has the LPPL maintenance status `maintained'.
%
% The Current Maintainer of this work is Tom M. Ragonneau.
\chapter{Conclusion and future research directions}
\label{ch:conclusion}
% \addcontentsline{toc}{chapter}{\nameref*{ch:conclusion}}
% \markboth{\nameref*{ch:conclusion}}{\nameref*{ch:conclusion}}

\section{Conclusion}

In this thesis, we investigated model-based methods for \gls{dfo}, covering both theoretical and practical aspects.

On the theoretical side, we first studied properties of an interpolation set for underdetermined interpolation employed by Powell in \gls{newuoa}.
We showed that this set is optimal when composed of~$2n + 1$ points,~$n$ being the dimension of the problem.
This supports the recommendation made by Powell, and it also guides us to devise the initial interpolation set of our new \gls{dfo} method.

We also examined several theoretical aspects of the \gls{sqp} method.
In particular, we highlighted a new interpretation of the \gls{sqp} subproblem, regarding its objective function as a quadratic approximation of the original objective function in the tangent space of a surface.
Moreover, we established a relation between the Vardi and the Byrd-Omojokun approaches in the equality-constrained case, interpreting the former as an approximation of the latter.
We also briefly studied the Lagrangian and augmented Lagrangian functions of the \gls{sqp} subproblem, establishing connections between the \gls{sqp} method and certain augmented Lagrangian methods.

Writing a thesis on model-based \gls{dfo} methods, we allocated a major part of our efforts to practices and implementations of these methods, which are essential in \gls{dfo}.
In particular, we developed in \cref{ch:pdfo} a cross-platform package named \gls{pdfo}, providing user-friendly MATLAB and Python interfaces for employing late Prof.\ M. J. D. Powell's \gls{dfo} solvers.
The package also patches several bugs contained in the Fortran 77 code.
As of August 2022, this package has been downloaded more than \num{30000} times, and has been included in GEMSEO, an engine for \gls{mdo}.

The aforementioned theoretical and practical study lays a solid groundwork for our development of a new derivative-free method for nonlinearly constrained optimization.
Named \gls{cobyqa}, it is a trust-region \gls{sqp} method that models the objective and constraint functions using quadratic models based on the derivative-free symmetric Broyden update.
The development of this method is fully documented in \cref{ch:cobyqa-introduction,ch:cobyqa-subproblems,ch:cobyqa-implementation}, covering the mathematical basis, the detailed descriptions of the subproblem solvers, and the Python implementation.
In this development, we documented, adapted, and extended various tools for the practices of \gls{sqp} and \gls{dfo}.
For example, we proposed a new extension of the Byrd-Omojokun approach for the inequality-constrained case, which worked remarkably well in our tests.
Finally, extensive numerical experiments demonstrated the excellent performance of \gls{cobyqa}, confirming that we have achieved our initial goal of developing a general-purpose \gls{dfo} solver, as a successor to \gls{cobyla}.

\section{Future research directions}

The study of this thesis opens many interesting directions for further investigations.
We briefly mention four of them.

\begin{enumerate}
    \item When developing \gls{cobyqa}, we followed Powell's philosophy reflected in his work on \gls{dfo}, which is to first produce an efficient and robust solver that performs well in practice, instead of setting up a conceptual framework without a practical implementation.
    Now that \gls{cobyqa} is successful in practice, we will proceed to analyze its convergence properties, including global convergence and worst-case complexity.
    We expect that such an analysis can be done based on the existing techniques for analyzing \gls{dfo} methods in the unconstrained case, and the \gls{sqp} method in the derivative-based case.
    \item We will further improve the implementation of \gls{cobyqa} to enhance its performance, particularly in the linearly constrained case, recalling that the current version does not perform as well as \gls{lincoa} on linearly constrained problems with violable bounds.
    In addition, we plan to implement \gls{cobyqa} in modern Fortran (F2018 or above), because the current Python version can be slow for high-dimensional problems.
    We will also make \gls{cobyqa} available in other languages, such as MATLAB, Julia, and R.
    The implementation of \gls{pdfo} in Julia and R are also under consideration.
    \item Our new interpretation of the \gls{sqp} subproblem sheds lights on the approximation of nonlinear functions on manifolds.
    It will be interesting to investigate the implication of our interpretation on manifold optimization.
    \item The connections between the \gls{sqp} and the augmented Lagrangian methods that we mentioned in \cref{subsec:lagrangian-augmented-lagrangian} also deserve further investigations.
    We will study the implications of these connections on the theory and practice of the \gls{sqp} and the augmented Lagrangian methods, trying to develop theories that bridge these two methodologies and algorithms that combine their strenghts.
    % \item We showed in \cref{sec:optimal-interpolation-set} that an interpolation set for quadratic underdetermined interpolation introduced by \citeauthor{Powell_2006}~\cite{Powell_2006} is optimal to some extend when the number of interpolation points is~$2n + 1$.
    % However, we assume in this study that this number of points was at most~$2n + 1$, although the original interpolation set is also define for more.
    % Therefore, we will later extend the properties shown to the general case.
    % \item We presented our software \gls{pdfo} for MATLAB and Python.
    % However, Julia~\cite{Bezanson_Etal_2017} is gaining more and more interest for researchers and practitioners.
    % Therefore, the author and Dr.\ Zaikun Zhang plan to implement an Julia version of \gls{pdfo}.
    % \item We established in \cref{thm:sqp-path} a new result on the subproblem of the \gls{sqp} framework.
    % More specifically, it shows that the objective function of the \gls{sqp} subproblem is a natural approximation of the original objective function in the tangent space of the contour of the constraints at the current iterate.
    % However, we established this result in the case where no inequality constraint is provided.
    % Therefore, we plan to extend this result in the future to the general case.
    % \item We showed in \cref{prop:vardi-byrd-omojokun} an intriguing relation between the Vardi approach and the Byrd-Omojokun approach, presented in \cref{subsec:composite-step-equality}.
    % More specifically, we showed that the normal step of the Vardi approach was a truncated Newton step for normal subproblem the Byrd-Omojokun approach.
    % We established however this result only in the equality-constrained case, and we will extend it in the future to the general case.
    % \item We plan to study in the near future the convergence properties of \gls{cobyqa}.
    % As we discussed, numerical experiments indicate that the method works in practice, but mathematical proofs are yet to be constructed.
    % \item Finally, we plan to implement \gls{cobyqa} in Fortran, as the current Python version can be slow for high-dimensional problem, because Python is an interpreted language.
    % Moreover, we will fine-tune the implementation to improve the performance, particularly in the linearly and nonlinearly constrained cases.
\end{enumerate}

\todo[noline]{Complete the index}
