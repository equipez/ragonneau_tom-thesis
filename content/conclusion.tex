%% contents/conclusion.tex
%% Copyright 2021-2022 Tom M. Ragonneau
%
% This work may be distributed and/or modified under the
% conditions of the LaTeX Project Public License, either version 1.3
% of this license or (at your option) any later version.
% The latest version of this license is in
%   http://www.latex-project.org/lppl.txt
% and version 1.3 or later is part of all distributions of LaTeX
% version 2005/12/01 or later.
%
% This work has the LPPL maintenance status `maintained'.
%
% The Current Maintainer of this work is Tom M. Ragonneau.
\chapter{Conclusion and future work}
\label{ch:conclusion}
% \addcontentsline{toc}{chapter}{\nameref*{ch:conclusion}}
% \markboth{\nameref*{ch:conclusion}}{\nameref*{ch:conclusion}}

\section{Conclusion}

In this thesis, we presented and analyzed trust-region methods for \gls{dfo}, including some theoretical analyses and some practical implementation.
In particular, we introduced in \cref{ch:pdfo} \gls{pdfo}, a cross-platform package providing MATLAB and Python interfaces for using late Prof.\ M. J. D. Powell's \gls{dfo} solvers, developed in a joint work with Dr.\ Zaikun Zhang.
As of August 2022, this package has been downloaded more than \num{30000} times, and is included in the Python package GEMSEO~\cite{Gallard_Etal_2018}, an engine for \gls{mdo}.

Further, we presented our new method \gls{cobyqa}, for nonlinear-constrained \gls{dfo} problems.
It is based on the \gls{sqp} framework, which we carefully introduced in \cref{ch:sqp}.
In particular, we interpreted its subproblem from different perspectives, one of which led to establishing a new result, namely \cref{thm:sqp-path}.
It interprets the objective function of the \gls{sqp} subproblem as a quadratic approximation of the original objective function in the tangent space of a surface.
After introducing a classical trust-region \gls{sqp} framework for derivative-based optimization, we adapted it in \cref{ch:cobyqa-introduction} to \gls{dfo} settings.
Finally, in \cref{ch:cobyqa-subproblems,ch:cobyqa-implementation}, we provided more details on the implementation of the method.
The implementation of such a \gls{dfo} method is a highly challenging work, due to the nature of \gls{dfo} solvers.
The numerical experiments in \cref{sec:cobyqa-experiments} are promising, and encourage us to continue working on such an algorithm.

\section{Future research directions}

To conclude this thesis, we summarize in this section the future research directions we discussed in the thesis, chronologically.

\begin{enumerate}
    \item We showed in \cref{sec:optimal-interpolation-set} that an interpolation set for quadratic underdetermined interpolation introduced by \citeauthor{Powell_2006}~\cite{Powell_2006} is optimal to some extend when the number of interpolation points is~$2n + 1$.
    However, we assume in this study that this number of points was at most~$2n + 1$, although the original interpolation set is also define for more.
    Therefore, we will later extend the properties shown to the general case.
    \item We presented our software \gls{pdfo} for MATLAB and Python.
    However, Julia~\cite{Bezanson_Etal_2017} is gaining more and more interest for researchers and practitioners.
    Therefore, the author and Dr.\ Zaikun Zhang plan to implement an Julia version of \gls{pdfo}.
    \item We established in \cref{thm:sqp-path} a new result on the subproblem of the \gls{sqp} framework.
    More specifically, it shows that the objective function of the \gls{sqp} subproblem is a natural approximation of the original objective function in the tangent space of the contour of the constraints at the current iterate.
    However, we established this result in the case where no inequality constraint is provided.
    Therefore, we plan to extend this result in the future to the general case.
    \item We showed in \cref{prop:vardi-byrd-omojokun} an interesting relation between the Vardi approach and the Byrd-Omojokun approach, presented in \cref{subsec:composite-step-equality}.
    \item More specifically, we showed that the normal step of the Vardi approach was a truncated Newton step for normal subproblem the Byrd-Omojokun approach.
    We established however this result only in the equality-constrained case, and we will extend it in the future to the general case.
    \item We plan to study in the near future the convergence properties of \gls{cobyqa}.
    As we discussed, numerical experiments indicate that the method works in practice, but mathematical proofs are yet to be constructed.
    \item Finally, we plan to implement \gls{cobyqa} in Fortran, as the current Python version can be slow for high-dimensional problem, because Python is an interpreted language.
    Moreover, we will fine-tune the implementation to improve the performance, particularly in the linearly- and nonlinearly-constrained cases.
\end{enumerate}

\todo[noline]{Complete the index}
