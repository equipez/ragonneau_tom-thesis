%% contents/acknowledgments.tex
%% Copyright 2021-2022 Tom M. Ragonneau
%
% This work may be distributed and/or modified under the
% conditions of the LaTeX Project Public License, either version 1.3
% of this license or (at your option) any later version.
% The latest version of this license is in
%   http://www.latex-project.org/lppl.txt
% and version 1.3 or later is part of all distributions of LaTeX
% version 2005/12/01 or later.
%
% This work has the LPPL maintenance status `maintained'.
%
% The Current Maintainer of this work is Tom M. Ragonneau.
\chapter*{Acknowledgments}
\label{ch:acknowledgments}
\addcontentsline{toc}{chapter}{\nameref*{ch:acknowledgments}}
\markboth{\nameref*{ch:acknowledgments}}{\nameref*{ch:acknowledgments}}

First, I would like to express my deepest gratitude to Prof.\ Xiaojun Chen and Dr.\ Zaikun Zhang from the Department of Applied Mathematics of The Hong Kong Polytechnic University for supervising the research study.
Their patience and motivation helped me to overcome the challenges I faced during my studies.

The support and encouragement provided by Prof.\ Serge Gratton from the Department of Applied Mathematics of Toulouse INP/ENSEEIHT and Prof.\ Ya-xiang Yuan from the Institute of Computational Mathematics and Scientific/Engineering Computing of the Chinese Academy of Sciences were greatly appreciated.
They shared beneficial comments on the research projects and helped me through my research studies.

Words cannot express my gratitude to my parents and my sister for their wholesome counsel, wise guidance, and sympathetic ears during the research study, and my friends, for always being by my side, providing invigorating discussions along with resourceful and necessary distractions.

Finally, I would like to extend my sincere thanks to the Research Grants Council of Hong Kong, under the aegis of the University Grants Committee of Hong Kong, for providing financial support under the Hong Kong Ph.D. Fellowship Scheme.
I also thank The Hong Kong Polytechnic University for supporting the research study and, more specifically, the Department of Applied Mathematics for providing the necessary equipment and research facilities.
