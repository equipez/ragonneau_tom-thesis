%% contents/cobyqa-subproblems.tex
%% Copyright 2021-2022 Tom M. Ragonneau
%
% This work may be distributed and/or modified under the
% conditions of the LaTeX Project Public License, either version 1.3
% of this license or (at your option) any later version.
% The latest version of this license is in
%   http://www.latex-project.org/lppl.txt
% and version 1.3 or later is part of all distributions of LaTeX
% version 2005/12/01 or later.
%
% This work has the LPPL maintenance status `maintained'.
%
% The Current Maintainer of this work is Tom M. Ragonneau.
\chapter{\glsfmttext{cobyqa} \textemdash\ solving the subproblems}
\label{ch:cobyqa-subproblems}

This chapter introduces the methods employed by \gls{cobyqa} to solve its various subproblems.
We first present the classical Steihaug-Toint \glsxtrfull{tcg} method in \cref{sec:tcg}.
We then introduce in \cref{sec:cobyqa-tangential,sec:cobyqa-normal} active-set variations of the \gls{tcg} method to solve the tangential subproblem~\cref{eq:cobyqa-tangential} and the normal subproblem~\cref{eq:cobyqa-normal}, respectively.
\Cref{sec:cobyqa-lagrange-multipliers} adapts the \gls{nnls} algorithm~\cite[Alg.~23.10]{Lawson_Hanson_1987} to solve~\cref{eq:least-squares-lagrange-multipliers-cobyqa} for estimating the Lagrange multiplier in \gls{cobyqa}.
Finally, \cref{sec:cobyqa-geometry-improving} details the procedure that \gls{cobyqa} employs to solve the geometry-improving subproblem~\cref{eq:geometry-subproblem}.

\section{The \glsfmtlong{tcg} (\glsfmtshort{tcg}) method}
\label{sec:tcg}

The classical trust-region subproblem for unconstrained optimization is of the form
\begin{subequations}
    \label{eq:problem-tcg}
    \begin{align}
        \min_{\sstep \in \R^n}  & \quad Q(\sstep)\\
        \text{s.t.}             & \quad \norm{\sstep} \le \rad,
    \end{align}
\end{subequations}
with~$Q \in \qpoly$ being the trust-region model and~$\rad > 0$ being the trust-region radius.
The usual convergence results for trust-region methods do not require us to solve~\cref{eq:problem-tcg} exactly.
Rather, we only need to find a inexact solution~$\sstep[\ast]$ that satisfies
\begin{equation*}
    Q(0) - Q(\sstep[\ast]) \ge c \norm{\nabla Q(0)} \min @@ \set[\bigg]{\frac{\norm{\nabla Q(0)}}{\norm{\nabla^2 Q}}, \rad},
\end{equation*}
for some~$c > 0$, where we assume that~$\norm{\nabla Q(0)} / \norm{\nabla^2 Q} = \infty$ if~$\nabla^2 Q = 0$.
This is achieved for example by the Cauchy step, i.e., the step that minimizes~$Q$ along~$-\nabla Q(0)$, subject to the trust-region constraint~\cite[Thm.~4]{Powell_1975b}.

% Therefore, in trust-region method, we solve~\cref{eq:problem-tcg} approximately.
\index{TCG@\glsfmtshort{tcg}|(}A well-known method for finding such an inexact solution to~\cref{eq:problem-tcg} is the Steihaug-Toint \gls{tcg} method~\cite{Steihaug_1983,Toint_1981}, presented in \cref{alg:tcg}.
Briefly speaking, this algorithm entertains conjugate gradients iterations, and stops the computations if a step reaches the boundary of the trust region.

\begin{algorithm}
    \caption{Steihaug-Toint \glsfmtshort{tcg} method}
    \label{alg:tcg}
    \DontPrintSemicolon
    \onehalfspacing
    \KwData{Quadratic function~$Q$ and trust-region radius~$\rad > 0$.}
    \KwResult{An approximate solution to~\cref{eq:problem-tcg}.}
    Set~$\sstep[0] \gets 0$\;
    Set the search direction~$\pstep[0] \gets -\nabla Q(\sstep[0])$\;
    \For{$k = 0, 1, \dots$ until~$\norm{\pstep[k]} = 0$}{
        Set
        \begin{algoempheq}[left={\alpha_Q^k \gets \empheqlbrace}]{alignat*=2}
            & \frac{- \nabla Q(\sstep[k])^{\T} \pstep[k]}{(\pstep[k])^{\T} (\nabla^2 Q) \pstep[k]}  && \quad \text{if~$(\pstep[k])^{\T} (\nabla^2 Q) \pstep[k] > 0$,}\\
            & \infty                                                                                && \quad \text{otherwise}
        \end{algoempheq}
        Compute~$\alpha_{\rad}^k \gets \argmax @@ \set{\alpha \ge 0 : \norm{\sstep[k] + \alpha \pstep[k]} \le \rad}$\nomenclature[Op]{$\argmax$}{Global maximizer operator}\;
        Set the steplength~$\alpha^k \gets \min @@ \set{\alpha_Q^k, \alpha_{\rad}^k}$\;
        Update the iterate~$\sstep[k + 1] \gets \sstep[k] + \alpha^k \pstep[k]$\;
        \eIf{$\alpha^k = \alpha_{\rad}^k$}{
            Break\;
        }{
            Set~$\beta^k \gets \norm{\nabla Q(\sstep[k + 1])}^2 / \norm{\nabla Q(\sstep[k])}^2$\;
            Update~$\pstep[k + 1] \gets -\nabla Q(\sstep[k]) + \beta^k \pstep[k]$\;
        }
    }
\end{algorithm}

This algorithm enjoys many good properties.
In particular, \citeauthor{Yuan_2000}~\cite{Yuan_2000} showed that the \gls{tcg} step provides at least half the reduction provided by the exact solution to~\cref{eq:problem-tcg} if~$\nabla^2 Q$ is positive definite.
The tangential and normal subproblem solvers employed by \gls{cobyqa} are constrained variations of the \gls{tcg} method, as presented in \cref{sec:cobyqa-tangential,sec:cobyqa-normal}.\index{TCG@\glsfmtshort{tcg}|)}

\section{Solving the tangential subproblem}
\label{sec:cobyqa-tangential}

We now present the constrained variations of the \gls{tcg} method we use in \gls{cobyqa} to solve the tangential subproblem~\cref{eq:cobyqa-tangential}.
We introduce two variations, one for the bound constrained case (when~$\iub \cup \ieq = \emptyset$), and the other for the linearly-constrained case (when~$\iub \cup \ieq \neq \emptyset$).

\subsection{Bound-constrained case}

\index{TCG@\glsfmtshort{tcg}!bound-constrained|(}We present here the bound-constrained variation of the \gls{tcg} method designed by \citeauthor{Powell_2009} for his solver \gls{bobyqa}~\cite{Powell_2009}.
This is the method employed by \gls{cobyqa} to solve the tangential subproblem~\cref{eq:cobyqa-tangential} when only bound constraints exist.
In such a case,~\cref{eq:cobyqa-tangential} is of the form
\begin{subequations}
    \label{eq:problem-bvtcg}
    \begin{align}
        \min_{\sstep \in \R^n}  & \quad Q(\sstep) \label{eq:problem-bvtcg-obj}\\
        \text{s.t.}             & \quad \xl \le \sstep \le \xu,\\
                                & \quad \norm{\sstep} \le \rad,
    \end{align}
\end{subequations}
where the objective function~$Q$ is quadratic, the bounds~$\xl \in (\R \cup \set{-\infty})^n$ and~$\xu \in (\R \cup \set{\infty})^n$ satisfy~$\xl \le 0$,~$\xu \ge 0$, and~$\xl < \xu$.
Note that the lower bound~$\xl$ in~\cref{eq:problem-bvtcg} indeed corresponds to the bound~$\xl - \iter[k]$ in~\cref{eq:cobyqa-tangential}, but we abuse the notation~$\xl$ for simplicity.
The situation for~$\xu$ is similar.

As an active-set variation of the \gls{tcg} method presented in \cref{alg:tcg}, the method designed by \citeauthor{Powell_2009} for~\cref{eq:problem-bvtcg} is as follows.
The method maintains a working set of active bounds.
At the~$k$th iteration, if a bound constraint is in the working set, then the corresponding coordinate of~$\sstep[k]$ is fixed at this bound, and it will not be changed in the subsequent iterations.
In the subspace of the coordinates that are not fixed, a \gls{tcg} step is taken.
If a new bound is hit by the \gls{tcg} step, the bound is added to the working set, and the procedure is restarted.
The working set is only enlarged through the iterations, and hence, the restarting happens only finitely many times.

The method is detailled in \cref{alg:bvtcg}.
The initial working set is a subset of the active bounds at the starting point~$\sstep[0] = 0$, which is the trust-region center.
It does not include an active bound if a step along~$-\nabla Q(\sstep[0])$ would depart from the bound.
More specifically, the initial working set is
\begin{equation}
    \label{eq:bounds-initial-working-set}
    \mathcal{W}^0 \eqdef \set[\bigg]{i \in \set{1, \dots, n} : \xl_i = \sstep[0]_i ~ \text{and} ~ \frac{\partial Q}{\partial \sstep_i}(\sstep[0]) \ge 0, ~ \text{or} ~ \xu_i = \sstep[0]_i ~ \text{and} ~ \frac{\partial Q}{\partial \sstep_i}(\sstep[0]) \le 0}.
\end{equation}
Given a vector~$z \in \R^n$, the algorithm uses~$\Pi^k(z) \in \R^n$ to denote the vector whose~$i$th component is~$z_i$ if~$i \notin \mathcal{W}^k$, and zero otherwise.

\begin{algorithm}
    \caption[Bound-constrained \glsfmtshort{tcg} method]{Bound-constrained \glsfmtshort{tcg} method~\cite[\S~3]{Powell_2009}}
    \label{alg:bvtcg}
    \DontPrintSemicolon
    \onehalfspacing
    \KwData{Quadratic function~$Q$, bounds~$\xl < \xu$, and trust-region radius~$\rad > 0$.}
    \KwResult{An approximate solution to~\cref{eq:problem-bvtcg}.}
    Set~$\sstep[0] \gets 0$ and~$\mathcal{W}^0$ according to~\cref{eq:bounds-initial-working-set}\;
    Set the search direction~$\pstep[0] \gets -\Pi^0(\nabla Q(\sstep[0]))$\;
    \For{$k = 0, 1, \dots$ until~$\norm{\pstep[k]} = 0$}{
        Set
        \begin{algoempheq}[left={\alpha_Q^k \gets \empheqlbrace}]{alignat*=2}
            & \frac{- \nabla Q(\sstep[k])^{\T} \pstep[k]}{(\pstep[k])^{\T} (\nabla^2 Q) \pstep[k]}  && \quad \text{if~$(\pstep[k])^{\T} (\nabla^2 Q) \pstep[k] > 0$,}\\
            & \infty                                                                                && \quad \text{otherwise}
        \end{algoempheq}
        Compute~$\alpha_{\rad}^k \gets \argmax @@ \set{\alpha \ge 0 : \norm{\sstep[k] + \alpha \pstep[k]} \le \rad}$\;
        Compute~$\alpha_B^k \gets \argmax @@ \set{\alpha \ge 0 : \xl \le \sstep[k] + \alpha \pstep[k] \le \xu}$\;
        Set the steplength~$\alpha^k \gets \min @@ \set{\alpha_Q^k, \alpha_{\rad}^k, \alpha_B^k}$\;
        Update the iterate~$\sstep[k + 1] \gets \sstep[k] + \alpha^k \pstep[k]$\;
        \uIf{$\alpha^k = \alpha_{\rad}^k$}{
            Break\;
        }
        \uElseIf{$\alpha^k = \alpha_B^k$}{
            Add the first bound hit by~$\sstep[k + 1]$ to~$\mathcal{W}^k$ to obtain~$\mathcal{W}^{k + 1}$\;
            Restart the \gls{tcg} iterations by setting~$\pstep[k + 1] \gets -\Pi^{k + 1}(\nabla Q(\sstep[k + 1]))$\;
        }
        \Else{
            Set~$\mathcal{W}^{k + 1} \gets \mathcal{W}^k$\;
            Set~$\beta^k \gets \norm{\Pi^{k + 1}(\nabla Q(\sstep[k + 1]))}^2 / \norm{\Pi^{k + 1}(\nabla Q(\sstep[k]))}^2$\;
            Update~$\pstep[k + 1] \gets -\Pi^{k + 1}(\nabla Q(\sstep[k])) + \beta^k \pstep[k]$\;
        }
    }
\end{algorithm}

\Gls{cobyqa} employs an adapted version of \cref{alg:bvtcg} for solving its tangential subproblem~\cref{eq:cobyqa-tangential} if~$\iub \cup \ieq = \emptyset$.
The adaptation, also proposed by \citeauthor{Powell_2009}~\cite[\S~3]{Powell_2009} and implemented in \gls{bobyqa}, is based on the following observation.
Let~$\sstep[\ast]$ be the point returned by \cref{alg:bvtcg}.
If~$\sstep[\ast]$ satisfies~$\norm{\sstep[\ast]} = \rad$, it is likely that the objective function in~\cref{eq:problem-bvtcg-obj} can be further decreased by moving this point round the trust-region boundary.
% In fact, the global solution of~\cref{eq:problem-bvtcg} is on the trust-region boundary in such a case.
The method employed by \gls{cobyqa} then attempts to further reduce the objective function by refining~$\sstep[\ast]$ with a step that approximately solves
\begin{subequations}
    \label{eq:cobyqa-tangential-bounds-alternative}
    \begin{align}
        \min_{\sstep \in \R^n}  & \quad Q(\sstep[\ast] + \sstep)\\
        \text{s.t.}             & \quad \xl \le \sstep[\ast] + \sstep \le \xu,\\
                                & \quad \norm{\sstep[\ast] + \sstep} = \rad,\\
                                & \quad \sstep \in \vspan @@ \set{\Pi^{\ast}(\sstep[\ast]), \Pi^{\ast}(\nabla Q(\sstep[\ast]))},
    \end{align}
\end{subequations}
where~$\Pi^{\ast}$ corresponds initially to the operator~$\Pi^k$ at the last iteration of \cref{alg:bvtcg}.
Note that a substantial reduction in~$Q$ is unlikely if either~$\Pi^{\ast}(\nabla Q(\sstep[\ast]))$ or the angle between~$\Pi^{\ast}(\sstep[\ast])$ and~$-\Pi^{\ast}(\nabla Q(\sstep[\ast]))$ is tiny.
Therefore, the method designed by \citeauthor{Powell_2009} attempts to refine~$\sstep[\ast]$ round the trust-region boundary only if
\begin{equation*}
    \norm{\Pi^{\ast}(\sstep[\ast])}^2 \norm{\Pi^{\ast}(\nabla Q(\sstep[\ast]))}^2 - [\Pi^{\ast}(\sstep[\ast])^{\T} \Pi^{\ast}(\nabla Q(\sstep[\ast]))]^2 > \xi [Q(\sstep[0]) - Q(\sstep[\ast])]^2,
\end{equation*}
for some~$\xi \in (0, 1)$.
In \gls{cobyqa}, we choose~$\xi = 10^{-4}$.
When this refinement is entertained, the method builds an orthogonal basis~$\set{\Pi^{\ast}(\sstep[\ast]), w}$ of~$\vspan \set{\Pi^{\ast}(\sstep[\ast]), \Pi^{\ast}(\nabla Q(\sstep[\ast]))}$ by computing the vector~$w \in \R^n$ that satisfies
\begin{equation*}
    w^{\T} \Pi^{\ast}(\sstep[\ast]) = 0, \quad w^{\T} \Pi^{\ast}(\nabla Q(\sstep[\ast])) < 0, \quad \text{and} \quad \norm{w} = \norm{\Pi^{\ast}(\sstep[\ast])}.
\end{equation*}
Further, the method considers the function
\begin{equation*}
    \sstep(\theta) \eqdef [\cos(\theta) - 1] \Pi^{\ast}(\sstep[\ast]) + \sin(\theta) w, \quad \text{for~$0 \le \theta \le \pi / 2$,}
\end{equation*}
finds an approximate solution~$\theta^{\ast}$ to the univariate problem
\begin{subequations}
    \begin{align}
        \min_{\theta \in \R}    & \quad Q(\sstep[\ast] + \sstep(\theta))\\
        \text{s.t.}             & \quad \xl \le \sstep[\ast] + \sstep(\theta) \le \xu, \label{eq:cobyqa-tangential-bounds-alternative-2-bds}\\
                                & \quad 0 \le \theta \le \pi / 2,
    \end{align}
\end{subequations}
and uses~$\sstep(\theta^{\ast})$ as an approximate solution to~\cref{eq:cobyqa-tangential-bounds-alternative}.
Note that the choice of~$w$ ensures that~$\norm{\sstep[\ast] + \sstep(\theta^{\ast})} = \norm{\sstep[\ast]} = \rad$.

% To solve approximately such a problem, the solver seeks for the greatest reduction in the objective function for a range of equally spaced values of~$\theta$, chosen to ensure the feasibility of the iterates.
If the value of~$\theta^{\ast}$ is restricted by~\cref{eq:cobyqa-tangential-bounds-alternative-2-bds}, then the first active bound is added to the working set~$\mathcal{W}^{\ast}$, and the refining procedure specified above is invoked again after~$\sstep[\ast]$ is updated to~$\sstep[\ast] + \sstep(\theta^{\ast})$, and~$\Pi^{\ast}$ is updated according to~$\mathcal{W}^{\ast}$.
Since the working set is only enlarged, this procedure is invoked at most~$n - \card(\mathcal{W}^{\ast})$ times, where~$\mathcal{W}^{\ast}$ denotes the working set when \cref{alg:bvtcg} terminates.

% The bound-constrained \gls{tcg} procedure together with the improving mechanism is implemented in \gls{cobyqa} under the function \texttt{cobyqa.linalg.bvtcg}.\index{TCG@\glsfmtshort{tcg}!bound-constrained|)}

\subsection{Linearly-constrained case}
\label{subsec:lctcg}

\index{TCG@\glsfmtshort{tcg}!linearly-constrained|(}We now present the linearly-constrained variation of the \gls{tcg} method~\cite{Powell_2015} designed by \citeauthor{Powell_2015} for his solver \gls{lincoa}.
The original method is devised for linear inequality constraints only, but is adapted here to handle linear equality constraints as well.

This method is employed by \gls{cobyqa} when~$\iub \cup \ieq \neq \emptyset$.
In such a case, the tangential subproblem~\cref{eq:cobyqa-tangential} is of the form
\begin{subequations}
    \label{eq:problem-lctcg}
    \begin{align}
        \min_{\sstep \in \R^n}  & \quad Q(\sstep)\\
        \text{s.t.}             & \quad A \sstep \le b, \label{eq:problem-lctcg-ub}\\
                                & \quad C \sstep = 0,\\
                                & \quad \norm{\sstep} \le \rad,
    \end{align}
\end{subequations}
where the objective function~$Q$ is quadratic,~$A \in \R^{m_1 \times n}$,~$C \in \R^{m_2 \times n}$, and~$b \in \R^{m_1}$ with~$b \ge 0$.
In this form, we include the bound constraints in~\cref{eq:problem-lctcg-ub}.
Since the method is a feasible method, the bounds will be respected.

The method uses an active-set strategy.
It maintains a working set and the QR factorization of a matrix whose columns form a basis of the space spanned by the gradients of the constraints in this working set.
As in \cref{alg:bvtcg}, if the working set is modified, then the \gls{tcg} procedure is restarted.
In contrast to \cref{alg:bvtcg} however, this method allows constraints to be removed from the working set.

\subsubsection{Description of the working set}

The working set is not directly the active set at the current iterate, for the following reason.
Assume that~$b_j$ is positive and tiny for a certain~$j$, so that the~$j$th constraint is almost active at~$\sstep[0] = 0$.
If~$j$ does not belong to the initial working set and if we have~$e_j^{\T} A \nabla Q(\sstep[0]) < 0$, then it is likely that the first step~$\sstep[1]$ generated by the \gls{tcg} method has a small norm, because a tiny step along the search direction~$-\nabla Q(\sstep[0])$ would quickly reach the boundary of the feasible set, leading to a change of the working set.
Moreover, as observed by \citeauthor{Powell_2015}, small steps that cause changes in the working set are numerically expensive, and hence, should be avoided~\cite[\S~3]{Powell_2015}.
Therefore, we must consider constraints with small residuals when defining the working set.

The method defines simultaneously the initial search direction and the initial working set.
More precisely, for any feasible~$\sstep \in \R^n$, we define
\begin{equation*}
    \mathcal{J}(\sstep) \eqdef \set[\big]{j \in \set{1, \dots, m_1} : b_j - e_j^{\T} A \sstep \le \sigma \rad \norm{e_j^{\T} A}},
\end{equation*}
for some~$\sigma \in (0, 1)$, set to~$\sigma = 0.2$ in \gls{cobyqa}.
The method defines the initial search direction~$\pstep[0]$ as the closest direction from~$-\nabla Q(\sstep[0])$ that do not tighten the proximity of~$\sstep[0]$ to the constraints in~$\mathcal{J}(\sstep[0])$.
In other words,~$\pstep[0]$ solves
\begin{subequations}
    \label{eq:tcg-linear-working-set}
    \begin{align}
        \min_{\sstep \in \R^n}  & \quad \frac{1}{2} \norm{\sstep + \nabla Q(\sstep[k])}^2\\
        \text{s.t.}             & \quad e_j^{\T} A \sstep \le 0, ~ j \in \mathcal{J}(\sstep[k]),\\
                                & \quad C \sstep = 0,
    \end{align}
\end{subequations}
with~$k = 0$.
Moreover, if~$e_j^{\T} A \pstep[0] < 0$ for some~$j \in \mathcal{J}(\sstep[0])$, then the point~$\sstep[1]$ will be further from this constraint than~$\sstep[0]$.
Hence, the initial working set is chosen to be
\begin{equation}
    \label{eq:tcg-linear-working-set-initial}
    \mathcal{W}^0 = \set{j \in \mathcal{J}(\sstep[0]) : e_j^{\T} A \pstep[0] = 0}.
\end{equation}

The method then entertains \gls{tcg} iterations in the null space of
\begin{equation*}
    \set[\big]{e_j^{\T} A : j \in \mathcal{W}^0} \cup \set[\big]{e_j^{\T} C : 1 \le j \le m_2}.
\end{equation*}
If a new constraint is hit by at a \gls{tcg} iteration that is not close from the trust region, then the method is restarted, changing the next search direction~$\pstep[k + 1]$ to the unique solution to~\cref{eq:tcg-linear-working-set} after incrementing~$k$.
Further, the working set is modified to be
\begin{equation}
    \label{eq:tcg-linear-working-set-restart}
    \mathcal{W}^{k + 1} = \set{j \in \mathcal{J}(\sstep[k + 1]) : e_j^{\T} A \pstep[k + 1] = 0}.
\end{equation}
We denote for convenience by~$\Pi^k(z) \in \R^n$ the unique solution to
\begin{align*}
    \min_{\sstep \in \R^n}  & \quad \frac{1}{2} \norm{\sstep - z}^2\\
    \text{s.t.}             & \quad e_j^{\T} A \sstep = 0, ~ j \in \mathcal{W}^k,\\
                            & \quad C \sstep = 0,
\end{align*}
for~$z \in \R^n$, and we remark that~$\pstep[0] = \Pi^0(-\nabla Q(\sstep[0]))$ and that~$\pstep[k + 1] = \Pi^{k + 1}(-\nabla Q(\sstep[k + 1]))$ when the method is restarted.

The solution to~\cref{eq:tcg-linear-working-set} is calculated using the \citeauthor{Goldfarb_Idnani_1983} method for quadratic programming~\cite{Goldfarb_Idnani_1983}.
It is an active-set method designed for minimizing positive definite quadratic function subject to linear constraints.
Using QR factorizations, it builds a linearly independent subset of the working set.
Therefore, after executing the \citeauthor{Goldfarb_Idnani_1983} method, we have a solution to~\cref{eq:tcg-linear-working-set} together with a linearly independent subset of the working set.
In fact, denote~$\hat{Q}R$ the QR factorization of the matrix whose columns are the gradients of the active constraints at the solution, and let~$\check{Q}$ be a matrix such that~$[\hat{Q}, \check{Q}]$ is orthogonal.
It turns out that the solution to~\cref{eq:tcg-linear-working-set} is~$\check{Q} \check{Q}^{\T} v$ (see~\cite[Eq.~(3.7)]{Powell_2015}).
Therefore, we have in particular~$\Pi^k(-z) = -\Pi^k(z)$ for all~$z \in \R^n$.

\subsubsection{Description of the \glsfmtshort{tcg} method}

We are now equiped to present the linearly-constrained \gls{tcg} method designed by \citeauthor{Powell_2015}.
The complete framework is described in \cref{alg:lctcg}.

\begin{algorithm}
    \caption[Linearly-constrained \glsfmtshort{tcg} method]{Linearly-constrained \glsfmtshort{tcg} method~\cite[\S~3]{Powell_2015}}
    \label{alg:lctcg}
    \DontPrintSemicolon
    \onehalfspacing
    \KwData{Quadratic function~$Q$, matrices~$A$ and~$C$, right-hand side~$b \ge 0$, trust-region radius~$\rad > 0$, and constant~$\sigma \in (0, 1)$.}
    \KwResult{An approximate solution to~\cref{eq:problem-lctcg}.}
    Set~$\sstep[0] \gets 0$\;
    Set~$\pstep[0]$ to the solution to~\cref{eq:tcg-linear-working-set} with~$k = 0$\;
    Set~$\mathcal{W}^0$ as described in~\cref{eq:tcg-linear-working-set-initial}\; \nllabel{alg:lctcg-sd-init}
    \For{$k = 0, 1, \dots$ until~$\norm{\pstep[k]} = 0$}{
        Set
        \begin{algoempheq}[left={\alpha_Q^k \gets \empheqlbrace}]{alignat*=2}
            & \frac{- \nabla Q(\sstep[k])^{\T} \pstep[k]}{(\pstep[k])^{\T} (\nabla^2 Q) \pstep[k]}  && \quad \text{if~$(\pstep[k])^{\T} (\nabla^2 Q) \pstep[k] > 0$,}\\
            & \infty                                                                                && \quad \text{otherwise}
        \end{algoempheq}
        Compute~$\alpha_{\rad}^k \gets \argmax @@ \set{\alpha \ge 0 : \norm{\sstep[k] + \alpha \pstep[k]} \le \rad}$\;
        Compute~$\alpha_L^k \gets \argmax @@ \set{\alpha \ge 0 : A(\sstep[k] + \alpha \pstep[k]) \le b}$\;
        Set the steplength~$\alpha^k \gets \min @@ \set{\alpha_Q^k, \alpha_{\rad}^k, \alpha_L^k}$\;
        Update the iterate~$\sstep[k + 1] \gets \sstep[k] + \alpha^k \pstep[k]$\;
        \uIf{either $\alpha^k = \alpha_L^k$ and~$\norm{\sstep[k + 1]} > (1 - \sigma) \rad$ or~$\alpha^k = \alpha_{\rad}^k$}{
            Break\;
        }
        \uElseIf{$\alpha^k = \alpha_L^k$ and~$\norm{\sstep[k + 1]} \le (1 - \sigma) \rad$}{
            Set~$\pstep[k + 1]$ to the solution to~\cref{eq:tcg-linear-working-set} after incrementing~$k$\; \nllabel{alg:lctcg-sd-restart}
            Restart the \gls{tcg} iterations by setting~$\mathcal{W}^{k + 1}$ as described in~\cref{eq:tcg-linear-working-set-restart}\;
        }
        \Else{
            Set~$\mathcal{W}^{k + 1} \gets \mathcal{W}^k$\;
            Set
            \begin{algomathdisplay}
                \beta^k \gets \frac{\Pi^{k + 1} \big( \nabla Q(\sstep[k + 1]) \big)^{\T} (\nabla^2 Q) \pstep[k]}{(\pstep[k])^{\T} (\nabla^2 Q) \pstep[k]}
            \end{algomathdisplay}
            Update~$\pstep[k + 1] \gets -\Pi^{k + 1} \big( \nabla Q(\sstep[k]) \big) + \beta^k \pstep[k]$\;
        }
    }
\end{algorithm}

\Citeauthor{Powell_2015}~\cite{Powell_2015} proposed the following modification, which is employed in \gls{cobyqa}.
After calculating the search direction at \cref{alg:lctcg-sd-init}, we might have
\begin{equation*}
    b_j - e_j^{\T} A(\sstep[0] + \pstep[0]) \ge \xi \rad \norm{e_j^{\T} A}
\end{equation*}
for some~$\xi \in (0, \sigma)$ and some~$j$ in the working set.
In such a case, the method first moves~$\sstep[0]$ towards the boundaries of the constraints in the working set.
This is because the constraints in the working set are presumed active at the solution,
In \gls{cobyqa}, we choose~$\xi = 10^{-4}$.
The same applies to \cref{alg:lctcg-sd-restart}.

\subsubsection{Additional stopping criteria}

\Citeauthor{Powell_2015} also proposed two additional stopping criteria, to avoid doing unworthy computations.
Note that these two stopping critera are also crucial from a theoretical standpoint, because the termination of the method, proven by \citeauthor{Powell_2015}~\cite[\S~5]{Powell_2015}, relies on them.
First of all, the computations are stopped if~$\nabla Q(\sstep[k])$ is small along the current search direction, namely if
\begin{equation*}
    \alpha_{\rad}^k \abs[\big]{\nabla Q(\sstep[k])^{\T} \pstep[k]} \le \nu [Q(\sstep[0]) - Q(\sstep[k])],
\end{equation*}
for some constant~$\nu > 0$, set to~$\nu = 0.01$ in \gls{cobyqa}.
Moreover, the computations are also stopped if the reduction provided by updated iterate is tiny compared to the reduction so far, that is if
\begin{equation*}
    Q(\sstep[k]) - Q(\sstep[k + 1]) \le \nu [Q(\sstep[0]) - Q(\sstep[k + 1])].
\end{equation*}\index{TCG@\glsfmtshort{tcg}!linearly-constrained|)}

% The linearly-constrained \gls{tcg} procedure with these additional stopping criteria is implemented in \gls{cobyqa} under the function \texttt{cobyqa.linalg.lctcg}.

\section{Solving the normal subproblem}
\label{sec:cobyqa-normal}

In this section, we present the method employed by \gls{cobyqa} to approximately solve its normal subproblem~\cref{eq:cobyqa-normal}.
Unlike for the tangential subproblem, the objective function of the normal subproblem is not a quadratic.
It is rather piecewise quadratic.
More specifically, it is of the form
\begin{subequations}
    \label{eq:problem-cpqp}
    \begin{align}
        \min_{\sstep \in \R^n}  & \quad \frac{1}{2} \big[ \norm{\posp{A \sstep - b}}^2 + \norm{C \sstep - d}^2 \big]\\
        \text{s.t.}             & \quad \xl \le \sstep \le \xu,\\
                                & \quad \norm{\sstep} \le \rad,
    \end{align}
\end{subequations}
where~$A \in \R^{m_1 \times n}$,~$C \in \R^{m_2 \times n}$,~$b \in \R^{m_1}$,~$d \in \R^{m_2}$, and the lower bound~$\xl \in (\R \cup \set{-\infty})^n$ and the upper bound~$\xu \in (\R \cup \set{\infty})^n$ satisfy~$\xl \le 0$,~$\xu \ge 0$, and~$\xl < \xu$.
Note that these bounds are not the same as in~\cref{eq:problem-cobyqa}.
In~\cref{eq:problem-cpqp},~$\posp{\cdot}$ takes the positive part of a number, and~$\norm{\cdot}$ denotes the~$\ell_2$-norm.

\subsubsection{Reformulation of the problem}

The main difficulty in solving~\cref{eq:problem-cpqp}, even approximately, is the piecewise quadratic term~$\norm{\posp{A \sstep - b}}^2$ in its objective function.
However, it can clearly be reformulated by introducing a variable~$v \in \R^{m_1}$ as
\begin{subequations}
    \label{eq:problem-cpqp-reformulated}
    \begin{align}
        \min_{(\sstep, v) \in \R^n \times \R^{m_1}} & \quad Q(\sstep, v) \eqdef \frac{1}{2} \big[ \norm{v}^2 + \norm{C \sstep - d}^2 \big]\\
        \text{s.t.}                                 & \quad E \sstep + F v \le h,\\
                                                    & \quad \norm{\sstep} \le \rad,
    \end{align}
\end{subequations}
where
\begin{equation*}
    E \eqdef
    \begin{bmatrix}
        A\\
        I_n\\
        -I_n\\
        0
    \end{bmatrix}, \quad
    F \eqdef
    \begin{bmatrix}
        -I_{m_1}\\
        0\\
        0\\
        I_{m_1}
    \end{bmatrix}, \quad \text{and} \quad
    h \eqdef
    \begin{bmatrix}
        b\\
        \xu\\
        -\xl\\
        0
    \end{bmatrix}.
\end{equation*}
When compared to the original problem~\cref{eq:problem-cpqp}, the advantage of the reformulated problem~\cref{eq:problem-cpqp-reformulated} is that its objective function is quadratic, although
\begin{enumerate}
    \item the dimension of the reformulated problem is higher, and
    \item linear inequality constraints appear in the reformulated problem.
\end{enumerate}

We observe nonetheless that problem~\cref{eq:problem-cpqp-reformulated} is almost of the right form to be solved approximately by \cref{alg:lctcg}.
Only the trust-region constraint differs, and is not applied to all variables of the reformulated problem.
From a geometrical point of view, the feasible set engendered by the only nonlinear constraint is a ball for the problems tackled by \cref{alg:lctcg}, while it is a cylinder in~\cref{eq:problem-cpqp-reformulated}.
To solve problem~\cref{eq:problem-cpqp-reformulated}, and hence~\cref{eq:problem-cpqp}, we design below a variation of \cref{alg:lctcg}.

\subsubsection{Description of the working set}

Just as in \cref{subsec:lctcg}, we define
\begin{equation*}
    \mathcal{J}(\sstep, v) \eqdef \set[\Big]{j \le 2n + 2m_1 : h_j - e_j^{\T}(E \sstep + F v) \le \sigma \rad \sqrt{\norm{e_j^{\T} E}^2 + \norm{e_j^{T} F}^2}},
\end{equation*}
for some~$\sigma \in (0, 1)$, set to~$\sigma = 0.2$ in \gls{cobyqa}.
Further, the initial working set is a subset of~$\mathcal{J}(\sstep[0], v^0)$, with~$\sstep[0] = 0$ and~$v^0 = \posp{-b}$.
To define the initial search directions, let~$(\Pi_{\sstep}^k(x, z), \Pi_v^k(x, z))$ be the unique solution to
\begin{align*}
    \min_{(\sstep, v) \in \R^n \times \R^{m_1}} & \quad \frac{1}{2} [ \norm{\sstep - x}^2 + \norm{v - z}^2 ]\\
    \text{s.t.}                                 & \quad e_j^{\T} (E \sstep + F y), ~ j \in \mathcal{J}(\sstep[k], v^k) \le 0.
\end{align*}
The initial search directions are
\begin{empheq}[left=\empheqlbrace]{alignat*=1}
    & \pstep[0]_{\sstep} = -\Pi_{\sstep}^0 \big( \nabla_{\sstep} Q(\sstep[0], v^0), \nabla_v Q(\sstep[0], v^0) \big),\\
    & \pstep[0]_v = -\Pi_v^0 \big( \nabla_{\sstep} Q(\sstep[0], v^0), \nabla_v Q(\sstep[0], v^0) \big).
\end{empheq}
This problem is solved numerically using the \citeauthor{Goldfarb_Idnani_1983} method~\cite{Goldfarb_Idnani_1983}, briefly described in \cref{subsec:lctcg}.

\subsubsection{Description of the \glsfmtshort{tcg}-like method}

We are now equiped to present the \gls{tcg}-like method for solving the problem~\cref{eq:problem-cpqp-reformulated}, and hence~\cref{eq:problem-cpqp}.
The complete framework is described in \cref{alg:cpqp}.

\begin{algorithm}
    \caption{\Glsfmtshort{tcg}-like method for convex piecewise quadratic programming}
    \label{alg:cpqp}
    \DontPrintSemicolon
    \onehalfspacing
    \KwData{Matrices~$C$,~$E$, and~$F$, vectors~$d$ and~$h$, trust-region radius~$\rad > 0$, and constant~$\sigma \in (0, 1)$.}
    \KwResult{An approximate solution to~\cref{eq:problem-lctcg}.}
    Set~$\sstep[0] \gets 0$,~$v^0 \gets \posp{-b}$,~$g^0 \gets -C^{\T} C d$, and~$\hat{k} \gets 0$\;
    Set the search directions~$\pstep[0]_{\sstep} \gets -\Pi_{\sstep}^{\hat{k}}(g^0, v^0)$ and~$\pstep[0]_v \gets -\Pi_v^{\hat{k}}(g^0, v^0)$\; \nllabel{alg:cpqp-sd-init}
    \For{$k = 0, 1, \dots$ until~$\norm{\pstep[k]_{\sstep}}^2 + \norm{\pstep[k]_v}^2 = 0$}{
        Set
        \begin{algoempheq}[left={\alpha_Q^k \gets \empheqlbrace}]{alignat*=2}
            & \frac{- (g^k)^{\T} \pstep[k]_{\sstep} - (v^k)^{\T} \pstep[k]_v}{\norm{C \pstep[k]_{\sstep}}^2 + \norm{\pstep[k]_v}^2} && \quad \text{if~$\norm{C \pstep[k]_{\sstep}}^2 + \norm{\pstep[k]_v}^2 > 0$,}\\
            & \infty                                                                                                                && \quad \text{otherwise}
        \end{algoempheq}
        Compute~$\alpha_{\rad}^k \gets \argmax @@ \set{\alpha \ge 0 : \norm{\sstep[k] + \alpha \pstep[k]_{\sstep}} \le \rad}$\;
        Compute~$\alpha_L^k \gets \argmax @@ \set{\alpha \ge 0 : E (\sstep[k] + \alpha \pstep[k]_{\sstep}) + F (v^k + \alpha \pstep[k]_v) \le h}$\;
        Set the steplength~$\alpha^k \gets \min @@ \set{\alpha_Q^k, \alpha_{\rad}^k, \alpha_L^k}$\;
        Update the iterates~$\sstep[k + 1] \gets \sstep[k] + \alpha^k \pstep[k]_{\sstep}$ and~$v^{k + 1} \gets v^k + \alpha^k \pstep[k]_v$\;
        Update the gradient~$g^{k + 1} \gets g^k + \alpha^k C^{\T} C \pstep[k]_{\sstep}$\;
        \uIf{$\alpha^k = \alpha_{\rad}^k$}{
            Break\;
        }
        \uElseIf{$\alpha^k = \alpha_L^k$}{
            Update~$\hat{k} = k + 1$\;
            Set~$\pstep[k + 1]_{\sstep} \gets -\Pi_{\sstep}^{\hat{k}}(g^{k + 1}, v^{k + 1})$ and~$\pstep[k + 1]_v \gets -\Pi_v^{\hat{k}}(g^{k + 1}, v^{k + 1})$\; \nllabel{alg:cpqp-sd-restart}
        }
        \Else{
            Set
            \begin{algomathdisplay}
                \beta^k \gets \frac{\Pi_{\sstep}^{\hat{k}}(g^{k + 1}, v^{k + 1})^{\T} C^{\T} C \pstep[k]_{\sstep} + \Pi_v^{\hat{k}}(g^{k + 1}, v^{k + 1})^{\T} \pstep[k]_v}{\norm{C \pstep[k]_{\sstep}}^2 + \norm{\pstep[k]_v}^2}
            \end{algomathdisplay}
            Update~$\pstep[k + 1]_{\sstep} \gets -\Pi_{\sstep}^{\hat{k}}(g^k, v^k) + \beta^k \pstep[k]_{\sstep}$ and~$\pstep[k + 1]_v \gets -\Pi_v^{\hat{k}}(g^k, v^k) + \beta^k \pstep[k]_v$\;
        }
    }
\end{algorithm}

Since the operators~$\Pi_{\sstep}^{\hat{k}}$ and~$\Pi_v^{\hat{k}}$ are evaluated using the \citeauthor{Goldfarb_Idnani_1983} method, \cref{alg:cpqp} shares a large part of the implementation with \cref{alg:lctcg}.

Moreover, similarly to \cref{alg:lctcg}, after calculating the search directions at \cref{alg:cpqp-sd-init,alg:cpqp-sd-restart}, the error term
\begin{equation*}
    h_j - e_j^{\T} E (\sstep[k] + \pstep[k]_{\sstep}) - e_j^{\T} F (v^k + \pstep[k]_v)
\end{equation*}
may be substantial for some~$j$ in the working set.
In such a case, the method first moves~$(\sstep[k], v^k)$ towards the boundaries of the constraints in the working set.

\subsubsection{Additional stopping criteria}

As in \cref{alg:lctcg}, we also implement the following two stopping criteria.
First of all, the computations are stopped if the decrease in the objective function is small along the current search direction, namely if
\begin{equation*}
    \alpha_{\rad}^k \abs[\big]{(g^k)^{\T} \pstep[k]_{\sstep} + (v^k)^{\T} \pstep[k]_v} \le \nu [Q(\sstep[0], v^0) - Q(\sstep[k], v^k)],
\end{equation*}
for some constant~$\nu > 0$, set to~$\nu = 0.01$ in \gls{cobyqa}.
Moreover, the computations are also stopped if the reduction provided by the current search direction is tiny compared to the reduction so far, that is if
\begin{equation*}
    Q(\sstep[k], v^k) - Q(\sstep[k + 1], v^{k + 1}) \le \nu [Q(\sstep[0], v^0) - Q(\sstep[k + 1], v^{k + 1})].
\end{equation*}
These two stopping criteria are once again crucial to the termination of the algorithm.
Indeed, remark that the trust-region constraint appears only as a stopping criteria in \cref{alg:cpqp}.
Therefore, the termination proof made by \citeauthor{Powell_2009}~\cite[\S~3]{Powell_2009} for \cref{alg:lctcg} also holds for \cref{alg:cpqp}.

% The \gls{tcg}-like procedure for convex piecewise quadratic programming we presented is implemented in \gls{cobyqa} under the function \texttt{cobyqa.linalg.cpqp}.

\section{Evaluating the least-squares Lagrange multiplier}
\label{sec:cobyqa-lagrange-multipliers}

We present in this section the method employed by \gls{cobyqa} to solve the least-squares problem~\cref{eq:least-squares-lagrange-multipliers-cobyqa}.
It is adapted from the \gls{nnls} algorithm~\cite[Alg.~23.10]{Lawson_Hanson_1987} as follows.
The update mechanism of the estimated Lagrange multipliers in \gls{cobyqa} is based on constrained least-squares problems, where some variables must remain nonnegative in order to satisfy some complementary slackness conditions.
The problem it solves are of the form
\begin{subequations}
    \label{eq:problem-nnls}
    \begin{align}
        \min_{\sstep \in \R^n}  & \quad \mu(\sstep) \eqdef \frac{1}{2} \norm{A \sstep - b}^2\\
        \text{s.t.}             & \quad \sstep_i \ge 0, ~ i = 1, 2, \dots, n_0,
    \end{align}
\end{subequations}
where~$A \in \R^{m \times n}$,~$b \in \R^m$,~$n_0$ is a nonnegative integer with~$n_0 \le n$, and~$\norm{\cdot}$ denotes the Euclidean norm.
We observe that if~$n_0 = 0$, then~\cref{eq:problem-nnls} is a simple unconstrained least-squares problem, which can be solved using traditional methods.

\subsubsection{Description of the method}

In order to solve problem~\cref{eq:problem-nnls} when~$n_0 \ge 1$, we construct an active-set method based on~\cite[Alg.~23.10]{Lawson_Hanson_1987}, referred to as \gls{nnls}.
The framework of the method is described in \cref{alg:nnls}.
It uses the notation~$\Pi(\mathcal{W})$, define as follows.
Given a working set~$\mathcal{W}$, we let~$\Pi(\mathcal{W})$ be the least-norm solution to
\begin{equation*}
    \min_{\sstep \in \R^n} \frac{1}{2} \norm{\hat{A} \sstep - b}^2
\end{equation*}
where~$\hat{A}$ is the matrix whose~$i$th column is that of~$A$ if~$i \notin \mathcal{W}$, and zero otherwise.


\begin{algorithm}
    \caption{\Glsdesc{nnls} method}
    \label{alg:nnls}
    \DontPrintSemicolon
    \onehalfspacing
    \KwData{Matrix~$A$, vector~$b$, and number of positive variables~$n_0 \ge 1$.}
    \KwResult{A solution to~\cref{eq:problem-nnls}.}
    Set~$\sstep[0] \gets 0$,~$\mathcal{W}^{-1} \gets \set{1, 2, \dots, n_0}$, and~$k \gets 0$\;
    \While{the \gls{kkt} conditions do not hold at~$\sstep[k]$}{ \nllabel{alg:nnls-kkt}
        Build~$\mathcal{W}^k$ by removing from~$\mathcal{W}^{k - 1}$ an index yielding
        \begin{algomathdisplay}
            \min_{i \in \mathcal{W}^k} @@ \frac{\partial \mu}{\partial \sstep_i}(\sstep[k])
        \end{algomathdisplay} \nllabel{alg:nnls-remove}
        Set the trial point~$\pstep[k] \gets \Pi(\mathcal{W}^k)$\; \nllabel{alg:nnls-trial-1}
        \While{$\pstep[k]_i \le 0$ for some~$i \notin \mathcal{W}^k$ with~$i \le n_0$}{  \nllabel{alg:nnls-inner}
            Set the steplength
            \begin{algomathdisplay}
                \alpha^k \gets \min @@ \set[\bigg]{\frac{\sstep[k]_i}{\sstep[k]_i - \pstep[k]_i} : \pstep[k]_i \le 0, ~ i \notin \mathcal{W}^k, ~ i \le n_0}
            \end{algomathdisplay}
            Update~$\sstep[k + 1] \gets \sstep[k] + \alpha^k (\pstep[k] - \sstep[k])$ and increment~$k$\; 
            Expand the working set~$\mathcal{W}^k \gets \mathcal{W}^{k - 1} \cup \set{i \le n_0 : \sstep[k]_i = 0}$\;
            Set the trial point~$\pstep[k] \gets \Pi(\mathcal{W}^k)$\; \nllabel{alg:nnls-trial-2}
        }
        Update~$\sstep[k + 1] \gets \pstep[k]$ and increment~$k$\;
    }
\end{algorithm}

The \gls{kkt} conditions at \cref{alg:nnls-kkt} are easy to verify in practice.
A point~$\sstep[k] \in \R^n$ satisfies the \gls{kkt} conditions for~\cref{eq:problem-nnls} if and only if~$\nabla \mu(\sstep[k]) \ge 0$ and
\begin{equation*}
    \frac{\partial \mu}{\partial \sstep_i} (\sstep[k]) = 0 \quad \text{if~$i > n_0$ or~$\sstep[k]_i > 0$.}
\end{equation*}

% This modification of the \gls{nnls} algorithm is implemented in \gls{cobyqa} under the function \texttt{cobyqa.linalg.nnls}.

\subsubsection{Convergence of the method}

We sketch below the convergence properties of \cref{alg:nnls}.
Since the termination criteria of the outer loop are the \gls{kkt} conditions for problem~\cref{eq:problem-nnls}, the termination of the algorithm implies its convergence because the problem is convex and the constraints are linear.
% The termination proof of the algorithm necessitates the following result, whose proof is established in~\cite[Lem.~23.17]{Lawson_Hanson_1987}.

% \begin{lemma}
%     \label{lem:nnls}
%     Assume that the matrix~$A \in \R^{m \times n}$ is full column rank and that the vector~$b \in \R^m$ satisfies~$b^{\T} A e_i = 0$ for all~$i \in \set{1, 2, \dots, n} \setminus \set{j}$ and~$b^{\T} A e_j > 0$ for some~$j \in \set{1, 2, \dots, n}$.
%     Then the least-squares solution~$\sstep[\ast]$ to~$A \sstep - b$ satisfies~$\sstep[\ast]_j > 0$.
% \end{lemma}

% Assume that the \gls{kkt} conditions for problem~\cref{eq:problem-nnls} do not hold at the origin.
% At the first iteration, the algorithm selects the index ($j$, say) of the most negative component of~$\nabla \rho(\sstep[0]) = -A^{\T} b$ and remove it from the working set.
% According to \cref{lem:nnls}, the solution to the least-squares problem at \cref{alg:nnls-trial-1} satisfies~$\pstep[0]_j > 0$.
% It is then easy to see that at each iteration, the solutions to the least-squares problem at \cref{alg:nnls-trial-1,alg:nnls-trial-2} satisfy~$\pstep[k]_j > 0$, where~$j$ is the index selected at \cref{alg:nnls-remove}.
% Such a solution is then modified by the inner loop to ensure that~$\sstep[k]_i \ge 0$ for all~$i \in \set{1, 2, \dots, n_0}$.
% To do so, it will select the closest point to~$\pstep[k]$ on the line joining~$\sstep[k]$ to~$\pstep[k]$ that is feasible.

\paragraph{Termination of the inner loop}

At each inner loop iteration, the size of the working set~$\mathcal{W}^k$ is enlarged.
If~$\mathcal{W}^k$ is maximal, that is~$\mathcal{W}^k = \set{1, 2, \dots, n_0}$, then the condition at \cref{alg:nnls-inner} is always false, and the loop terminates.

\paragraph{Termination of the outer loop}

Termination of the outer loop can be directly inferred by showing that the value of~$\mu$ strictly decreases at each outer loop iteration.
Such a condition ensures that the working set~$\mathcal{W}^k$ at a given outer loop iteration is different from all its previous instances, as each iterate of the outer loop is feasible (see~\cite[Lem.~23.17]{Lawson_Hanson_1987} and the discussion below).

When an outer loop iteration finishes, the value of~$\sstep[k + 1]$ solves
\begin{align*}
    \min_{\sstep \in \R^n}  & \quad \mu(\sstep)\\
    \text{s.t.}             & \quad \sstep_i \ge 0, ~ i \in \set{1, 2, \dots, n_0},\\
                            & \quad \sstep_i = 0, ~ i \in \mathcal{W}^k,
\end{align*}
If the \gls{kkt} conditions for problem~\cref{eq:problem-nnls} hold at~$\sstep[k + 1]$, then termination occurs.
Otherwise, after incrementing~$k$, the index yielding the least value of~$\nabla \mu(\sstep[k])$ is removed from the working set~$\mathcal{W}^k$, and the tentative solution vector~$\pstep[k]$ clearly provides~$\mu(\pstep[k]) < \mu(\sstep[k])$.
The result is then proven if no inner loop is entertained (that is, if condition at \cref{alg:nnls-inner} does not hold).
Otherwise, at the end of each inner loop, we have
\begin{align*}
    \norm{A \sstep[k + 1] - b}  & = \norm[\big]{A (\sstep[k] + \alpha^k (\pstep[k] - \sstep[k])) - b}\\
                                & = \norm[\big]{(1 - \alpha^k) (A \sstep[k] - b) + \alpha^k (A \pstep[k] - b)}\\
                                & < \norm{A \sstep[k] - b},
\end{align*}
since~$\alpha^k \in (0, 1)$.
The termination of the outer loop is then proven, as well as the convergence of the method.

% \begin{itemize}
%     \item How is it related to the original \gls{nnls} by considering the positive and negative parts?
% \end{itemize}

\section{Solving the geometry-improving subproblem}
\label{sec:cobyqa-geometry-improving}

We now details the way \gls{cobyqa} approximately solves the geometry-improving subproblem~\cref{eq:geometry-subproblem}.
Such a subproblem is of the form
\begin{subequations}
    \label{eq:geometry-simple}
    \begin{align}
        \min_{\sstep \in \R^n}  & \quad \abs{Q(\sstep)}\\
        \text{s.t.}             & \quad \xl \le \sstep \le \xu,\\
                                & \quad \norm{\sstep} \le \rad,
    \end{align}
\end{subequations}
where~$Q$ a is quadratic function, the lower bounds~$\xl \in (\R \cup \set{-\infty})^n$, and the upper bounds~$\xu \in (\R \cup \set{\infty})^n$ satisfy~$\xl \le 0$,~$\xu \ge 0$, and~$\xl < \xu$.
Note that these bounds are not the same as in~\cref{eq:problem-cobyqa}.

The method employed by \gls{cobyqa} is that of \gls{bobyqa}~\cite{Powell_2009}.
It computes two alternative approximations, and select the one that provides the largest value of the denominator of the updating formula in absolute value (see~\cref{subsec:geometry-improvement}).

The first alternative is a truncated Cauchy-like step.
More specifically, the method evaluate two Cauchy-like steps, one for the minimization of~$Q(\sstep)$, one for the minimization of~$-Q(\sstep)$, and selects the one that provides the largest value of~$\abs{Q(\sstep)}$.
The Cauchy-like step for the first calculation is evaluated as follows, the second one being similar.
Let~$\sstep[c]$ be a solution to
\begin{align*}
    \min_{\sstep \in \R^n}  & \quad Q(0) + \nabla Q(0)^{\T} \sstep\\
    \text{s.t.}             & \quad \xl \le \sstep \le \xu,\\
                            & \quad \norm{\sstep} \le \rad,
\end{align*}
The considered Cauchy-like step is then set to a multiple of~$\sstep[c]$ that minimizes~$Q$.

The second alternative is as follows.
Assume that the current interpolation set is~$\xpt$, and that~$0 \in \xpt$ is the best point so far (otherwise, shift the calculations).
The method minimizes~$\abs{Q(\sstep)}$ along the lines that join~$0$ to the other interpolation points.
In other words, it computes
\begin{align*}
    \min_{\sstep \in \R^n}  & \quad \abs{Q(\sstep)}\\
    \text{s.t.}             & \quad \xl \le \sstep \le \xu,\\
                            & \quad \norm{\sstep} \le \rad,\\
                            & \quad \sstep \in \set{\alpha y : \alpha \in (0, 1), ~ y \in \xpt}.
\end{align*}
This subproblem can easily be reformulated as~$\card(\xpt) - 1$ minimizations in the variable~$\alpha \in (0, 1)$, with can be directly computed.
This mechanism is described in~\cite{Powell_2008} for the unconstrained case, stating that adding this new constraint not only simplifies the computations, but also reduces the number of function evaluations in several experiments.

Another method that the author tried to approximately solve~\cref{eq:geometry-simple} is to employ \cref{alg:lctcg} on both~$Q$ and~$-Q$, taking the solution with the largest absolute value.
However, numerical experiments concurred with the observations of \citeauthor{Powell_2008}~\cite{Powell_2008}, as the performance of \gls{cobyqa} drastically decreased when compared to the method presented in this section.
This concludes our discussion of the methods employed by \gls{cobyqa} to solve its subproblem.

\section{Summary and remarks}

We presented in this chapter the detailed methods employed by \gls{cobyqa} to solve its various subproblems.
To solve the tangential subproblem~\cref{eq:cobyqa-tangential} and the normal subproblem~\cref{eq:cobyqa-normal}, we employ active-set variations of the Steihaug-Toint \gls{tcg} algorithm.
In particular, the tangential subproblem~\cref{eq:cobyqa-tangential} is either solved using the subproblem solver of \gls{bobyqa}~\cite{Powell_2009} or that of \gls{lincoa}~\cite{Powell_2015}, depending on whether it admits linear constraints or not.
Moreover, we established an elementary modification of the linearly-constrained \gls{tcg} method for solving approximately the normal subproblem.

We then established a method for solving the least-squares problem~\cref{eq:least-squares-lagrange-multipliers-cobyqa} for evaluating the least-squares Lagrange multiplier.
This solver is a straightforward modification of the \gls{nnls} algorithm~\cite[Alg.~23.10]{Lawson_Hanson_1987} for nonnegative least-squares problems.
The solution returned by this algorithm is exact, and the convergence of the method has been shown.

Finally, we introduced the method employed by \gls{cobyqa} for solving the geometry-improving subproblem~\cref{eq:geometry-subproblem}.
This method is the one designed by \citeauthor{Powell_2009} for solving the geometry-improving subproblem of \gls{bobyqa}~\cite{Powell_2009}.
Two approximate solutions are calculated, the best of which (according to some criterion) being returned.
