%% contents/cobyqa-subproblems.tex
%% Copyright 2021-2022 Tom M. Ragonneau
%
% This work may be distributed and/or modified under the
% conditions of the LaTeX Project Public License, either version 1.3
% of this license or (at your option) any later version.
% The latest version of this license is in
%   http://www.latex-project.org/lppl.txt
% and version 1.3 or later is part of all distributions of LaTeX
% version 2005/12/01 or later.
%
% This work has the LPPL maintenance status `maintained'.
%
% The Current Maintainer of this work is Tom M. Ragonneau.
\chapter{\glsfmttext{cobyqa} \textemdash\ solving the subproblems}
\label{ch:cobyqa-subproblems}

In this chapter, we present the methods employed by \gls{cobyqa} to solve its various subproblems, including the tangential subproblem~\cref{eq:cobyqa-tangential} (with a modified trust-region radius), the normal subproblem~\cref{eq:cobyqa-normal}, the least-squares problem~\cref{eq:least-squares-lagrange-multipliers-cobyqa} for estimating the Lagrange multiplier, and the geometry-improving subproblem~\cref{eq:geometry-subproblem}.

\section{The \glsfmtlong{tcg} method}

\section{Solving the tangential subproblem}
\label{sec:cobyqa-tangential}

\section{Solving the normal subproblem}
\label{sec:cobyqa-normal}

\section{Evaluating the least-squares Lagrange multiplier}
\label{sec:cobyqa-lagrange-multipliers}

\begin{itemize}
    \item How is it related to the original \gls{nnls} by considering the positive and negative parts?
\end{itemize}

\section{Solving the geometry-improving subproblem}
\label{sec:cobyqa-geometry-improving}
