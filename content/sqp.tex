%% contents/sqp.tex
%% Copyright 2021-2022 Tom M. Ragonneau
%
% This work may be distributed and/or modified under the
% conditions of the LaTeX Project Public License, either version 1.3
% of this license or (at your option) any later version.
% The latest version of this license is in
%   http://www.latex-project.org/lppl.txt
% and version 1.3 or later is part of all distributions of LaTeX
% version 2005/12/01 or later.
%
% This work has the LPPL maintenance status `maintained'.
%
% The Current Maintainer of this work is Tom M. Ragonneau.
\chapter{The \glsfmtlong{sqp} (\glsfmtshort{sqp}) method}

\section{The method}

This chapter presents the basic idea behind the \gls{sqp} method.
For this discussion, we assume that the derivatives of the objective and constraint functions are available, as is the case in the classical \gls{sqp} method.
% However, we will extend the method to the derivative-free context in \cref{subsec:derivative-free-sqp}.

% Note that the problem~\cref{eq:problem-cobyqa} formulates the bound constraints~\cref{eq:problem-cobyqa-bd} explicitely.
% This is important in computation, because it may not be reasonable to handle the bounds in the same way as the others due to their different natures (see \cref{subsec:bound-constraints} for details).
% However, in the theoretical development of this section and the next one (\cref{sec:trust-region}), it is not necessary to distinguish bound constraints from the others.
Throughout this chapter, we consider a problem of the form
\begin{subequations}
    \label{eq:problem-cobyqa-sqp}
    \begin{align}
        \min        & \quad \obj(\iter) \label{eq:problem-cobyqa-sqp-obj}\\
        \text{s.t.} & \quad \con{i}(\iter) \le 0, ~ i \in \iub, \label{eq:problem-cobyqa-sqp-ub}\\
                    & \quad \con{i}(\iter) = 0, ~ i \in \ieq, \label{eq:problem-cobyqa-sqp-eq}\\
                    & \quad \iter \in \R^n. \nonumber
    \end{align}
\end{subequations}
% where~$\iub$ contains the bound constraints, if any.

% \begin{remark}
%     Comparing~\cref{eq:problem-cobyqa} with~\cref{eq:problem-cobyqa-sqp}, we are abusing the notations, because~$\iub$ does not represent the same set of constraints in both.
%     However, this will not generate any confusion in our discussions.
% \end{remark}

Note that the problem~\cref{eq:problem-cobyqa-sqp} is precisely the problem~\cref{eq:problem-introduction} discussed in \cref{ch:introduction}; hence, all the theory mentioned is applicable.
For our later discussion, recall in particular that the Lagrangian of this problem is defined by
\begin{equation*}
    \lag(\iter, \lm) \eqdef \obj(\iter) + \sum_{\mathclap{i \in \iub \cup \ieq}} \lm_i \con{i}(\iter), \quad \text{for~$\iter \in \R^n$ and~$\lm_i \in \R$, with~$i \in \iub \cup \ieq$,}
\end{equation*}
where~$\lm = [\lm_i]_{i \in \iub \cup \ieq}^{\T}$ is the dual variable of the considered problem.

\subsection{Overview of the method}

The \gls{sqp} method is known to be one of the most powerful methods for solving the problem~\cref{eq:problem-cobyqa-sqp} when derivatives of~$\obj$ and~$\con{i}$, with~$i \in \iub \cup \ieq$, are available.
The classical \gls{sqp} framework is presented in \cref{alg:sqp}.

\begin{algorithm}
    \caption{Classical \glsfmtshort{sqp} method}
    \label{alg:sqp}
    \DontPrintSemicolon
    \KwData{Initial guess~$\iter[0] \in \R^n$ and estimated Lagrange multiplier~$\lm[0] = [\lm[0]_i]_{i \in \iub \cup \ieq}^{\T}$.}
    \For{$k = 0, 1, \dots$}{
        Define~$H^k \approx \nabla_{x, x}^2 \lag(\iter[k], \lm[k])$\;
        Generate a step~$\step[k] \in \R^n$ by solving approximately
        \begin{subequations}
            \label{eq:sqp-subproblem}
            \begin{algomathalign}
                \min        & \quad \nabla \obj(\iter[k])^{\T} \step + \frac{1}{2} \step^{\T} H^k \step \label{eq:sqp-subproblem-obj}\\
                \text{s.t.} & \quad \con{i}(\iter[k]) + \nabla \con{i}(\iter[k])^{\T} \step \le 0, ~ i \in \iub,\\
                            & \quad \con{i}(\iter[k]) + \nabla \con{i}(\iter[k])^{\T} \step = 0, ~ i \in \ieq,\\
                            & \quad \step \in \R^n, \nonumber
            \end{algomathalign}
        \end{subequations}
        Update the iterate~$\iter[k + 1] \gets \iter[k] + \step[k]$\;
        Estimate the Lagrange multiplier~$\lm[k + 1] = [\lm[k + 1]_i]_{i \in \iub \cup \ieq}^{\T}$
    }
\end{algorithm}

The earliest reference to such a method appeared in the Ph.D. thesis of \citeauthor{Wilson_1963}~\cite{Wilson_1963}, with~$H^k = \nabla_{x, x}^2 \lag(\iter[k], \lm[k])$.
\Citeauthor{Robinson_1974}~\cite{Robinson_1974} showed the local R-quadratic convergence rate of this method.
% Check for Q-quadratic convergence.
Later, \citeauthor{Garcia-Palomares_Mangasarian_1976}~\cite{Garcia-Palomares_1973,Garcia-Palomares_Mangasarian_1976} modified it using a quasi-Newton update for calculating~$H^k$ and established a local R-superlinear convergence rate for such an algorithm.
A similar method was introduced by \citeauthor{Han_1976}~\cite{Han_1976,Han_1977}, but he only approximated~$\nabla_{x, x}^2 \lag(\iter[k], \lm[k])$, while \citeauthor{Garcia-Palomares_Mangasarian_1976} applied quasi-Newton approximations to the whole matrix~$\nabla^2 \lag(\iter[k], \lm[k])$.
In addition, \citeauthor{Han_1976} introduced a line-search strategy to guarantee the global convergence and local Q-superlinear convergence rate, requiring that~$\nabla_{x, x}^2 \lag(\iter[\ast], \lm[\ast])$ is positive definite at the solution~$(\iter[\ast], \lm[\ast])$.
\Citeauthor{Powell_1978a}~\cite{Powell_1978b,Powell_1978a,Powell_1978c} studied the method in the same direction.
In particular, he proposed to apply the damped BFGS quasi-Newton formula~\cite[Eqs.~(5.8),~(5.9), and~(5.10)]{Powell_1978b} to update~$H^k$.
This formula guarantees the positive definiteness of such a matrix, which is beneficial in practice and theory (see the comments towards the end of~\cite[\S~2]{Powell_1978a}).
Moreover, he introduced a practical line-search technique based on a merit function suggested by \citeauthor{Han_1976}~\cite{Han_1976}.
Furthermore, \citeauthor{Powell_1978c} established the global convergence and the local R-superlinear convergence rate for his method without requiring the positive definiteness of~$\nabla_{x, x}^2 \lag(\iter[\ast], \lm[\ast])$ as \citeauthor{Han_1976} did.
Recognizing the contributions of \citeauthor{Wilson_1963}, \Citeauthor{Han_1976}, and \citeauthor{Powell_1978a}, the \gls{sqp} method is also referred to as the Wilson-Han-Powell method~\cite{Schittkowski_1981,Burke_1992}.
See~\cite{Boggs_Tolle_1995} for a more detailed review of the history, theory, and practice of the \gls{sqp} method.

\subsection{A simple example}
\label{subsec:sqp-simple-example}

In \cref{alg:sqp}, it is crucial that~$H^k$ approximates~$\nabla_{x, x}^2 \lag(\iter[k], \lm[k])$.
It may be tempting to set~$H^k \approx \nabla \obj(\iter[k])$, because the objective function of the \gls{sqp} subproblem would then be a local quadratic approximation of~$\obj$ at~$\iter[k]$.
However, such a naive idea does not work, as illustrated by the following~$2$-dimensional example inspired by \citeauthor{Boggs_Tolle_1995}~\cite[\S~2.2]{Boggs_Tolle_1995}.

We consider
\begin{align*}
    \min        & \quad -\iter_1 - \frac{(\iter_2)^2}{4}\\
    \text{s.t.} & \quad \norm{\iter}^2 - 1 = 0,\\
                & \quad \iter \in \R^2,
\end{align*}
whose solution is~$\iter[\ast] = [1, 0]^{\T}$ with the associated Lagrange multiplier~$\lm[\ast] = 1/2$.
Suppose that we have an iterate~$\iter[k] = [t, 0]^{\T}$ with~$t \approx 1$, so it is already close to the solution.
If~$H^k = \nabla^2 \obj(\iter[k])$, then the \gls{sqp} subproblem would become
\begin{subequations}
    \begin{align}
        \min        & \quad -\step_1 - \frac{(\step_2)^2}{4} \label{eq:boggs-tolle-sp-obj}\\
        \text{s.t.} & \quad \step_1 = \frac{1 - t^2}{2 t}, \label{eq:boggs-tolle-sp-eq}\\
                    & \quad \step \in \R^2. \nonumber
    \end{align}
\end{subequations}
This subproblem is unbounded from below, regardless of the value of~$t$.
In addition, the more~$\step[k]$ reduces~\cref{eq:boggs-tolle-sp-obj}, the larger~$\norm{\iter[k] + \step[k] - \iter[\ast]}$ is.
Moreover, if~$t = 1$, we have~$\iter[k] = \iter[\ast]$, but any feasible point~$\step[k]$ for~\cref{eq:boggs-tolle-sp-eq} will push~$\iter[k] + \step[k]$ away from~$\iter[\ast]$, unless~$\step[k]$ is the global maximizer of~\cref{eq:boggs-tolle-sp-obj} subject to~\cref{eq:boggs-tolle-sp-eq}.

Let us now consider the \gls{sqp} subproblem~\cref{eq:sqp-subproblem} with~$H^k = \nabla_{x, x}^2 \lag(\iter[k], \lm[k])$ for a dual variable~$\lm[k] \approx \lm[\ast] = 1/2$.
It is
\begin{align*}
    \min        & \quad -\step_1 + \lm[k] (\step_1)^2 + \bigg( \lm[k] - \frac{1}{4} \bigg) (\step_2)^2\\
    \text{s.t.} & \quad \step_1 = \frac{1 - t^2}{2 t},\\
                & \quad \step \in \R^2. \nonumber
\end{align*}
When~$\lm[k] > 1/4$, the solution to this subproblem is
\begin{equation*}
    \step[k] =
    \begin{bmatrix}
        \dfrac{1 - t^2}{2 t}    & 0
    \end{bmatrix}^{\T}.
\end{equation*}
We thus have
\begin{equation*}
    \iter[k] + \step[k] = 
    \begin{bmatrix}
        \dfrac{t^2 + 1}{2 t}  & 0
    \end{bmatrix}^{\T}.
\end{equation*}
If we set~$\iter[k + 1] = \iter[k] + \step[k]$ and continue to iterate in this way, we will obtain a sequence of iterates that converges quadratically to~$\iter[\ast]$, because
\begin{equation*}
    \norm{\iter[k] + \step[k] - \iter[\ast]} = \frac{(1 - t)^2}{2 \abs{t}} = \bigo(\norm{\iter[k] - \iter[\ast]}^2).
\end{equation*}
This is not surprising, since \citeauthor{Robinson_1974}~\cite{Robinson_1974} showed the local R-quadratic convergence rate of the \gls{sqp} method when~$H^k$ is the exact Hessian matrix of the Lagrangian with respect to~$x$.

To summarize, as indicated by this example, choosing~$H^k \approx \nabla_{x, x}^2 \lag(\iter[k], \lm[k])$ instead of~$H^k \approx \nabla \obj(\iter[k])$ in~\cref{eq:sqp-subproblem} is crucial.

\subsection{Interpretation of the \glsfmtshort{sqp} subproblem}
\label{subsec:sqp-interpretation}

To get some insight into the origin of the \gls{sqp} method, we interpret the \gls{sqp} subproblem~\cref{eq:sqp-subproblem}.
In what follows, we focus on only one iteration of \cref{alg:sqp} and hence,~$k$ is fixed.
We will explain why it is reasonable to update~$\iter[k]$ by a solution to~\cref{eq:sqp-subproblem}.

\subsubsection{Bilinear approximation of the \glsfmtlong{kkt} conditions}

This is the most classical interpretation of the \gls{sqp} subproblem.
According to \cref{thm:first-order-necessary-conditions}, if~$\iter[\ast] \in \R^n$ is a local solution to the problem~\cref{eq:problem-cobyqa-sqp}, under some mild assumptions, there exists a Lagrange multiplier~$\lm[\ast] = [\lm[\ast]_i]_{i \in \iub \cup \ieq}^{\T}$ with~$\lm[\ast]_i \in \R$ for all~$i \in \iub \cup \ieq$ such that
\begin{subequations}
    \label{eq:sqp-kkt}
    \begin{empheq}[left=\empheqlbrace]{alignat=2}
        & \nabla_x \lag(\iter[\ast], \lm[\ast]) = 0,    && \\
        & \con{i}(\iter[\ast]) \le 0,                   && \quad \text{if~$i \in \iub$,}\\
        & \con{i}(\iter[\ast]) = 0,                     && \quad \text{if~$i \in \ieq$,}\\
        & \lm[\ast]_i \con{i}(\iter[\ast]) = 0,         && \quad \text{if~$i \in \iub$,} \label{eq:sqp-kkt-complementary-slackness}\\
        & \lm[\ast]_i \ge 0,                            && \quad \text{if~$i \in \iub$.}
    \end{empheq}
\end{subequations}
Regard~\cref{eq:sqp-kkt} as a nonlinear system of inequalities and equalities, and~$(\iter[k], \lm[k])$ as an approximation of~$(\iter[\ast], \lm[\ast])$.
If we want to solve this system by the Newton-Raphson method\footnote{Discussions are needed on how to apply the Newton-Raphson method to systems of nonlinear inequalities and equalities. We will not go further in this direction but refer to \cite{Pshenichnyi_1970a,Pshenichnyi_1970b,Robinson_1972b,Daniel_1973} for fundamental works on this topic.} starting from~$(\iter[k], \lm[k])$, we would seek a step~$(\step, \mu)$ that satisfies the system
\begin{subequations}
    \label{eq:sqp-kkt-linearization}
    \begin{empheq}[left=\empheqlbrace]{alignat=2}
        & \nabla_x \lag(\iter[k], \lm[k] + \mu) + \nabla_{x, x}^2 \lag(\iter[k], \lm[k]) \step = 0,         && \\
        & \con{i}(\iter[k]) + \nabla \con{i}(\iter[k])^{\T} \step \le 0,                                    && \quad \text{if~$i \in \iub$,}\\
        & \con{i}(\iter[k]) + \nabla \con{i}(\iter[k])^{\T} \step = 0,                                      && \quad \text{if~$i \in \ieq$,}\\
        & \lm[k]_i [\con{i}(\iter[k]) + \nabla \con{i}(\iter[k])^{\T} \step] + \mu_i \con{i}(\iter[k]) = 0, && \quad \text{if~$i \in \iub$,} \label{eq:sqp-kkt-linearization-complementary-slackness}\\
        & \lm[k]_i + \mu_i \ge 0,                                                                           && \quad \text{if~$i \in \iub$,}
    \end{empheq}
\end{subequations}
which is a linear approximation of~\cref{eq:sqp-kkt} at~$(\iter[k], \lm[k])$.
However, as pointed out by \citeauthor{Robinson_1972a}~\cite[Rem.~3]{Robinson_1972a}, an objection to such a method is that it would not solve even a linear program in one iteration.
To cope with this deffect, we let~$(d, \mu)$ solve instead the following bilinear approximation of~\cref{eq:sqp-kkt},
\begin{subequations}
    \label{eq:sqp-subproblem-kkt}
    \begin{empheq}[left=\empheqlbrace]{alignat=2}
        & \nabla_x \lag(\iter[k], \lm[k] + \mu) + \nabla_{x, x}^2 \lag(\iter[k], \lm[k]) \step = 0, && \\
        & \con{i}(\iter[k]) + \nabla \con{i}(\iter[k])^{\T} \step \le 0,                            && \quad \text{if~$i \in \iub$,}\\
        & \con{i}(\iter[k]) + \nabla \con{i}(\iter[k])^{\T} \step = 0,                              && \quad \text{if~$i \in \ieq$,}\\
        & (\lm[k]_i + \mu_i) [\con{i}(\iter[k]) + \nabla \con{i}(\iter[k])^{\T} \step] = 0,         && \quad \text{if~$i \in \iub$,} \label{eq:sqp-subproblem-kkt-complementary-slackness}\\
        & \lm[k]_i + \mu_i \ge 0,                                                                   && \quad \text{if~$i \in \iub$.}
    \end{empheq}
\end{subequations}
Its only difference from the system~\cref{eq:sqp-kkt-linearization} lies in the condition~\cref{eq:sqp-subproblem-kkt-complementary-slackness}, which includes the bilinear term~$\mu_i \nabla \con{i}(\iter[k])^{\T} \step$.
If the problem~\cref{eq:problem-cobyqa-sqp} is a linear program, then~\cref{eq:sqp-subproblem-kkt} is precisely its \gls{kkt} system, while~\cref{eq:sqp-kkt-linearization} is only an approximation.
Observe that the bilinear system~\cref{eq:sqp-subproblem-kkt} is nothing but the \gls{kkt} conditions of the \gls{sqp} subproblem~\cref{eq:sqp-subproblem}, with~$\lm[k] + \mu$ being the Lagrange multiplier.
Therefore, a \gls{kkt} pair for the \gls{sqp} subproblem~\cref{eq:sqp-subproblem} is similar to a Newton-Raphson step for the \gls{kkt} system of the problem~\cref{eq:problem-cobyqa-sqp}, and it is even better in the sense that the resulting method solves a linear program in one iteration.

Note that discrepancy between the systems~\cref{eq:sqp-kkt-linearization,eq:sqp-subproblem-kkt} disappears if~$\iub = \emptyset$ in the problem~\cref{eq:problem-cobyqa-sqp} and hence, a \gls{kkt} pair for the \gls{sqp} subproblem~\cref{eq:sqp-subproblem} is exactly a Newton-Raphson step for the \gls{kkt} system of the problem~\cref{eq:problem-cobyqa-sqp} in such a situation.

\subsubsection{Approximation of a modified Lagrangian}

This interpretation is due to \citeauthor{Robinson_1972a}~\cite[Rem.~4]{Robinson_1972a}.
Let~$\widetilde{\lag}$ be the function
\begin{equation*}
    \widetilde{\lag}(\iter, \lm) \eqdef \obj(\iter) + \sum_{\mathclap{i \in \iub \cup \ieq}} \lm_i \delta_i(\iter), \quad \text{for~$\iter \in \R^n$ and~$\lm_i \in \R$, with~$i \in \iub \cup \ieq$},
\end{equation*}
where~$\delta_i$, for~$i \in \iub \cup \ieq$, is defined by
\begin{equation*}
    \delta_i(\iter) \eqdef \con{i}(\iter) - \con{i}(\iter[k]) - \nabla \con{i}(\iter[k])^{\T} (\iter - \iter[k]), \quad \text{for~$\iter \in \R^n$.}
\end{equation*}
The function~$\delta_i$ is referred to as the departure from linearity\footnote{When~$\con{i}$ is strictly convex,~$\delta_i$ defines the Bregman distance~\cite{Bregman_1967} associated with~$\con{i}$.} for~$\con{i}$ at the point~$\iter[k]$~\cite[\S~2]{Gill_Wong_2011}.
The \gls{sqp} subproblem~\cref{eq:sqp-subproblem} with~$H^k = \nabla_{x, x}^2 \lag(\iter[k], \lm[k])$ can then be seen as the minimization of the second-order Taylor approximation of~$\widetilde{\lag}$ subject to the linear approximations of the constraints~\cref{eq:problem-cobyqa-sqp-ub,eq:problem-cobyqa-sqp-eq} at~$(\iter[k], \lm[k])$, i.e.,
\begin{align*}
    \min        & \quad \nabla_x \widetilde{\lag}(\iter[k], \lm[k])^{\T} \step + \frac{1}{2} \step^{\T} \nabla_{x, x}^2 \widetilde{\lag}(\iter[k], \lm[k]) \step\\
    \text{s.t.} & \quad \con{i}(\iter[k]) + \nabla \con{i}(\iter[k])^{\T} \step \le 0, ~ i \in \iub,\\
                & \quad \con{i}(\iter[k]) + \nabla \con{i}(\iter[k])^{\T} \step = 0, ~ i \in \ieq,\\
                & \quad \step \in \R^n.
\end{align*}

By expressing its subproblem in this form, we observe that the \gls{sqp} method is a special case of \citeauthor{Robinson_1972a}'s method~\cite{Robinson_1972a}, known to have a local R-quadratic convergence rate.

\subsubsection{Approximation of the objective function in the tangent space of the feasible set}

Inspired by an observation in~\cite[\S~2]{Gill_Wong_2011}, we can also interpret the \gls{sqp} subproblem as minimizing an approximation of the objective function in the tangent space of the feasible set.
As will be shown in \cref{thm:sqp-path}, when approximating~$\obj$ in such a space, we will naturally get the Hessian matrix of the Lagrangian in the second-order term.

For this interpretation, we consider the problem
\begin{subequations}
    \label{eq:problem-cobyqa-auglag}
    \begin{align}
        \min        & \quad \obj(\iter)\\
        \text{s.t.} & \quad h(\iter) = 0,\\
                    & \quad \iter \ge 0, ~ \iter \in \R^n, \nonumber
    \end{align}
\end{subequations}
with~$h : \R^n \to \R^m$.
The problem~\cref{eq:problem-cobyqa-sqp} can be reformulated in this form\footnote{In the reformulation, the dimension and the meaning of~$\iter$ may be altered, but we do not change the notations since it does not lead to confusion}.
Recall that the Lagrangian of~\cref{eq:problem-cobyqa-auglag} is
\begin{equation*}
    \lag(\iter, \lm) \eqdef \obj(\iter) + \lm^{\T} h(\iter), \quad \text{for~$\iter \ge 0$ and~$\lm \in \R^m$.}
\end{equation*}

Let~$\bar{\iter} \in \R^n$,~$\bar{\lm} \in \R^m$ be given, and define
\begin{equation*}
    Q(\step) \eqdef \obj(\bar{\iter}) + \nabla \obj(\bar{\iter})^{\T} \step + \frac{1}{2} \step^{\T} \nabla_{x, x}^2 \lag(\bar{\iter}, \bar{\lm}) \step.
\end{equation*}
If~$\bar{\iter}$ and~$\bar{\lm}$ represent the current iterate and approximate Lagrange multiplier, then~$Q$ is the objective function of the \gls{sqp} subproblem with the exact second-order term.
Therefore, the \gls{sqp} subproblem approximates~$\obj$ by~$Q$ and the feasible set by its tangent space.
Such an approximation is reasonable only if~$Q$ approximates~$\obj$ in this tangent space, which turns out to be true, as detailed by \cref{thm:sqp-path}.

\begin{theorem}
    \label{thm:sqp-path}
    Assume that~$\obj$ and~$h$ are twice differentiable and that~$\nabla^2 \obj$ is Lipschitz continuous in a neighborhood of~$\bar{\iter}$.
    Let~$\iter(t)$ be a feasible path starting at~$\bar{\iter}$ and parametrized by a nonnegative scalar~$t$, i.e.,~$h(\iter(t)) = 0$ and~$\iter(t) \ge 0$ for~$t \ge 0$, and~$\iter(0) = \bar{\iter}$.
    Assume that~$\iter$ is twice differentiable for~$t \ge 0$ and that~$\iter''$ is Lipschitz continuous in a neighborhood of~$0$.
    Then, there exist constants~$\nu \ge 0$ and~$\epsilon > 0$ such that
    \begin{equation*}
        \abs{\obj(\iter(t)) - Q(x'(0) t)} \le \bigg( \nu t + \frac{1}{2}\abs{\iter''(0)^{\T} [\nabla \obj(\bar{\iter}) + \nabla h(\bar{\iter})^{\T} \bar{\lm}]} \bigg) t^2 \quad \text{for all~$t \in [0, \epsilon]$.}
    \end{equation*}
\end{theorem}

\begin{proof}
    Define~$\phi(t) = \obj(x(t))$ for~$t \ge 0$.
    For any~$t \ge 0$, we have
    \begin{subequations}
        \label{eq:sqp-path-proof-1}
        \begin{empheq}[left=\empheqlbrace]{alignat=1}
            & \phi'(t) = \iter'(t)^{\T} \nabla \obj(\iter(t)),\\
            & \phi''(t) = \iter'(t)^{\T} \nabla^2 \obj(\iter(t)) \iter'(t) + \iter''(t)^{\T} \nabla \obj(\iter(t)),
        \end{empheq}
    \end{subequations}
    and by assumption, there exists~$\epsilon > 0$ such that~$\phi''$ is Lipschitz continuous in~$[0, \epsilon]$.
    Let~$\widehat{\phi}$ be the second-order Taylor expansion of~$\phi$ at~$0$.
    We then have
    \begin{equation}
        \label{eq:sqp-path-proof-2}
        \abs{\obj(\iter(t)) - Q(\iter'(0) t)} \le \abs{\phi(t) - \widehat{\phi}(t)} + \abs{\widehat{\phi}(t) - Q(\iter'(0) t)}.
    \end{equation}
    Due to the Lipschitz continuity of~$\phi''$, there exists a constants~$\nu \ge 0$ such that
    \begin{equation}
        \label{eq:sqp-path-proof-3}
        \abs{\phi(t) - \widehat{\phi}(t)} \le \nu t^3 \quad \text{for~$t \in [0, \epsilon]$.}
    \end{equation}
    We now bound~$\abs{\widehat{\phi}(t) - Q(\iter'(0) t)}$.
    According to~\cref{eq:sqp-path-proof-1}, only the second-order terms of~$\widehat{\phi}(t)$ and~$Q(\iter'(0) t)$ differ, and
    \begin{subequations}
        \label{eq:sqp-path-proof-4}
        \begin{align}
            \abs{\widehat{\phi}(t) - Q(x'(0) t)}    & = \frac{t^2}{2} \abs{\phi''(0) - \iter'(0)^{\T} \nabla_{x, x}^2 \lag(\bar{\iter}, \bar{\lm}) \iter'(0)}\\
                                                    & = \frac{t^2}{2} \abs[\bigg]{\iter''(0)^{\T} \nabla \obj(\bar{\iter}) - \sum_{i = 1}^m \bar{\lm}_i \iter'(0) \nabla^2 h_i(\bar{\iter}) \iter'(0)}.
        \end{align}
    \end{subequations}
    Moreover, since~$h(\iter(t)) = 0$ for all~$t \ge 0$, we have
    \begin{equation*}
        0 = \frac{\du^2 h_i(\iter(t))}{\du t^2}\bigg\vert_{t = 0} = \iter'(0)^{\T} \nabla^2 h_i(\bar{\iter}) \iter'(0) + \iter''(0)^{\T} \nabla h_i(\bar{\iter}), \quad \text{for~$i \in \set{1, 2, \dots, m}$.}
    \end{equation*}
    Therefore,~$\iter'(0)^{\T} \nabla^2 h_i(\bar{\iter}) \iter'(0) = -\iter''(0)^{\T} \nabla h_i(\bar{\iter})$ for each~$i$ and hence,~\cref{eq:sqp-path-proof-4} leads to
    \begin{equation}
        \label{eq:sqp-path-proof-5}
        \abs{\widehat{\phi}(t) - Q(x'(0) t)} \le \frac{t^2}{2} \abs{\iter''(0)^{\T} [\nabla \obj(\bar{\iter}) + \nabla h(\bar{\iter})^{\T} \bar{\lm}]}.
    \end{equation}
    Plugging~\cref{eq:sqp-path-proof-3,eq:sqp-path-proof-5} into~\cref{eq:sqp-path-proof-2}, we obtain the desired result.
\end{proof}
We observe that the magnitude of~$\nabla \obj(\bar{\iter}) + \nabla h(\bar{\iter})^{\T} \bar{\lm}$ affects the error term in \cref{thm:sqp-path} in general.
If~$(\bar{\iter}, \bar{\lm})$ is close to \gls{kkt} pair, then~$\norm{\nabla \obj(\bar{\iter}) + \nabla h(\bar{\iter})^{\T} \bar{\lm}}$ is small.
On the other hand, if we defined~$\nabla^2 Q$ by~$\nabla^2 \obj(\bar{x})$ instead of~$\nabla_{x, x}^2 \lag(\bar{\iter}, \bar{\lm})$, then under the assumptions of \cref{thm:sqp-path}, we would have
\begin{equation*}
    \abs{\obj(\iter(t)) - Q(x'(0) t)} \le \bigg( \nu t + \frac{1}{2}\abs{\iter''(0)^{\T} \nabla \obj(\bar{\iter})} \bigg) t^2 \quad \text{for all~$t \in [0, \epsilon]$,}
\end{equation*}
which can be obtained by setting~$\bar{\lm} = 0$ in \cref{thm:sqp-path}.
However, since we are considering constrained optimization, we cannot expect~$\norm{\nabla \obj(\bar{\iter})}$ to be small even if~$\bar{\iter}$ is close to a solution, unless no constraint is active at this solution.
This explains why the second-order term of the \gls{sqp} subproblem should be defined by the Hessian matrix of the Lagrangian, rather than that of~$\obj$.

\subsection{Lagrangian and augmented Lagrangian of the \glsfmtshort{sqp} subproblem}

In this section, we study the Lagrangian and the augmented Lagrangian of the \gls{sqp} subproblem, and observe their relations with those of the original optimization problem.
We also point out that the augmented Lagrangian of the \gls{sqp} subproblem is exactly the approximate augmented Lagrangian used in~\cite{Niu_Yuan_2010,Wang_Yuan_2014}.

For simplicity, instead of the problem~\cref{eq:problem-cobyqa-sqp}, we consider still the problem~\cref{eq:problem-cobyqa-auglag}.
Let~$\iter[k] \ge 0$ and~$\lm[k] \in \R^m$ be given.
Correspondingly, the \gls{sqp} subproblem of the problem~\cref{eq:problem-cobyqa-auglag} is
\begin{subequations}
    \label{eq:sqp-subproblem-auglag}
    \begin{align}
        \min        & \quad \obj(\iter[k]) + \nabla \obj(\iter[k])^{\T} \step + \frac{1}{2} \step^{\T} H^k \step \label{eq:sqp-subproblem-auglag-obj}\\
        \text{s.t.} & \quad h(\iter[k]) + \nabla h(\iter[k]) \step = 0,\\
                    & \quad \iter[k] + \step \ge 0, ~ \step \in \R^n,
    \end{align}
\end{subequations}
with~$H^k \approx \nabla_{x, x}^2 \lag(\iter[k], \lm[k])$.
Note that the constant term~$\obj(\iter[k])$ in~\cref{eq:sqp-subproblem-auglag-obj} may be excluded, as in~\cref{eq:sqp-subproblem-obj}, but including it facilitates the discussion in the sequel.

Recall that the augmented Lagrangian~\cite{Hestenes_1969,Powell_1969,Rockafellar_1973} of the problem~\cref{eq:problem-cobyqa-auglag} is
\begin{equation}
    \label{eq:augmented-lagrangian-inequality}
    \lag[\mathsf{A}](\iter, \lm) \eqdef \lag(\iter, \lm) + \frac{\gamma}{2} \norm{h(\iter)}^2, \quad \text{for~$\iter \ge 0$ and~$\lm \in \R^m$,}
\end{equation}
where~$\gamma \ge 0$ is a penalty parameter.
Denote by~$\lagalt$ the Lagrangian of the \gls{sqp} subproblem~\cref{eq:sqp-subproblem-auglag}, i.e.,
\begin{align*}
    \lagalt(\step, \lm) & \eqdef \obj(\iter[k]) + \nabla \obj(\iter[k])^{\T} \step + \frac{1}{2} \step^{\T} H^k \step\\
                        & \qquad + \lm^{\T} [h(\iter[k]) + \nabla h(\iter[k]) \step], \quad \text{for~$\step \ge -\iter[k]$ and~$\lm \in \R^m$,}
\end{align*}
and by~$\lagalt[\mathsf{A}]$ the augmented Lagrangian of the \gls{sqp} subproblem~\cref{eq:sqp-subproblem-auglag}, i.e.,
\begin{equation*}
    \lagalt[\mathsf{A}](\step, \lm) \eqdef \lagalt(\step, \lm) + \frac{\gamma}{2} \norm{h(\iter[k]) + \nabla h(\iter[k]) \step}^2, \quad \text{for~$\step \ge -\iter[k]$ and~$\lm \in \R^m$.}
\end{equation*}

We now present some relations between~$\lag$ and~$\lagalt$, and also between~$\lag[\mathsf{A}]$ and~$\lagalt[\mathsf{A}]$.

\begin{theorem}
    \label{thm:auglag-sqp-1}
    Assume that~$\obj$ and~$h$ are twice differentiable.
    If~$H^k = \nabla_{x, x}^2 \lag(\iter[k], \lm[k])$, then~$\lagalt(\step, \lm[k])$ is the second-order Taylor expansion of~$\lag(\iter[k] + \step, \lm[k])$ with respect to~$d$ at~$0$.
\end{theorem}

\begin{proof}
    This theorem can be verified by a straightforward calculation.
\end{proof}

\begin{theorem}
    \label{thm:auglag-sqp-2}
    Assume that~$\obj$ and~$h$ are twice differentiable.
    If
    \begin{equation}
        \label{eq:auglag-sqp-2}
        H^k = \nabla^2 \obj(\iter[k]) + \sum_{i = 1}^m [\lm[k]_i + \gamma h_i(\iter[k])] \nabla^2 h_i(\iter[k]),
    \end{equation}
    then~$\lagalt[\mathsf{A}](\step, \lm[k])$ is the second-order Taylor expansion of~$\lag[\mathsf{A}](\iter[k] + d, \lm[k])$ with respect to~$d$ at~$0$.
\end{theorem}

\begin{proof}
    By direct calculations, we have
    \begin{equation*}
        \nabla_x \lag[\mathsf{A}](\iter[k], \lm[k]) = \nabla_x \lag(\iter[k], \lm[k]) + \gamma \nabla h(\iter[k])^{\T} h(\iter[k]),
    \end{equation*}
    and
    \begin{equation*}
        \nabla_{x, x}^2 \lag[\mathsf{A}](\iter[k], \lm[k]) = \nabla_{x, x}^2 \lag(\iter[k], \lm[k]) + \gamma \bigg[ \nabla h(\iter[k])^{\T} \nabla h(\iter[k]) + \sum_{i = 1}^m h_i(\iter[k]) \nabla^2 h_i(\iter[k]) \bigg].
    \end{equation*}
    Therefore, the second-order Taylor expansion of~$\lag[\mathsf{A}](\iter[k] + d, \lm[k])$ with respect to~$d$ at~$0$ is
    \begin{equation*}
        \lag[\mathsf{A}](\iter[k] + d, \lm[k]) = \lag(\iter[k], \lm[k]) + \nabla_x \lag(\iter[k], \lm[k])^{\T} d + \frac{1}{2} d^{\T} H^k d + \frac{\gamma}{2} \norm{h(\iter[k]) + \nabla h(\iter[k]) d}^2 + \smallo(\norm{d}^2),
    \end{equation*}
    where~$H^k$ is defined by~\cref{eq:auglag-sqp-2}.
\end{proof}

Intriguingly, an augmented Lagrangian method for solving the problem~\cref{eq:problem-cobyqa-auglag} updates traditionally the dual variable~$\lm[k]$ by
\begin{equation*}
    \lm[k + 1] = \lm[k] + \gamma h(\iter[k]).
\end{equation*}
Therefore, the second-order Taylor expansion of~$\lag[\mathsf{A}]$ can be interpreted as the augmented Lagrangian of the \gls{sqp} subproblem~\cref{eq:sqp-subproblem-auglag} with~$H^k = \nabla_{x, x}^2 \lag(\iter[k], \lm[k + 1])$.

There is an interesting connection between the augmented Lagrangian of the \gls{sqp} subproblem~\cref{eq:sqp-subproblem-auglag} and the trust-region augmented Lagrange methods studied in~\cite{Niu_Yuan_2010,Wang_Yuan_2014}.
These methods employ the approximation
\begin{equation}
    \label{eq:niu-yuan-auglag}
    \lag[\mathsf{A}](\iter[k] + \step, \lm[k]) \approx \lag(\iter[k], \lm[k]) + \nabla_x \lag(\iter[k], \lm[k])^{\T} d + \frac{1}{2} d^{\T} H^k d + \frac{\gamma}{2} \norm{h(\iter[k]) + \nabla h(\iter[k]) \step}^2,
\end{equation}
which is a quadratic approximation obtained by replacing~$\lag(\iter[k] + d, \lm[k])$ with a quadratic approximation, and replacing~$h(\iter[k] + d)$ in the penalty term with its first-order Taylor expansion.
This approximation turns out to be the augmented Lagrangian of the \gls{sqp} subproblem~\cref{eq:sqp-subproblem-auglag}, as shown by \cref{thm:auglag-sqp-3}.

\begin{theorem}
    \label{thm:auglag-sqp-3}
    Assume that~$\obj$ and~$h$ are differentiable.
    For any matrix~$H^k \in \R^{n \times n}$, the right-hand side of~\cref{eq:niu-yuan-auglag} equals~$\lagalt[\mathsf{A}](\step, \lm[k])$.
\end{theorem}

\begin{proof}
    Simarlarly to \cref{thm:auglag-sqp-1}, this theorem can be verified by a straightforward calculation.
\end{proof}

Several possibilities of~$H^k$ are proposed in~\cite{Niu_Yuan_2010,Wang_Yuan_2014}.
For example,~$H^k$ can be set to~$\nabla_{x, x}^2 \lag(\iter[k], \lm[k])$ or an approximation.
As pointed out by~\cite[\S~2.1]{Niu_Yuan_2010}, if~$H^k$ is defined as in~\cref{eq:auglag-sqp-2}, then the right-hand side of~\cref{eq:niu-yuan-auglag} is the second-order Taylor expansion of~$\lag[\mathsf{A}](\iter[k] + d, \lm[k])$ with respect to~$d$ at~$0$, which agrees with \cref{thm:auglag-sqp-2}.
However, in the numerical experiments,~\cite{Niu_Yuan_2010,Wang_Yuan_2014} choose~$H^k = \nabla_{x, x}^2 \lag(\iter[k], \lm[k])$.

Suppose that an algorithm defines a step~$\step[k]$ based on the minimization of the right-hand side of~\cref{eq:niu-yuan-auglag}.
Then, \cref{thm:auglag-sqp-3} tells us that the algorithm can be regarded as an \gls{sqp} method that approximately solves the \gls{sqp} subproblem~\cref{eq:sqp-subproblem-auglag} by minimizing~$\lagalt[\mathsf{A}](\step, \lm[k])$, i.e., by applying one single iteration of an augmented Lagrangian method.

\section{Merit functions for the \glsfmtshort{sqp} method and the Maratos effect}

In unconstrained optimization, a point~$x \in \R^n$ is normally considered to be better than another point~$y \in \R^n$ if~$\obj(x) < \obj(y)$.
However, this is not true in constrained optimization, because the feasibility of the points~$x$ and~$y$ must be taken into account.
This is usually done using \emph{merit functions}.
A merit function assesses the quality of a point by considering both~$\obj$ and~$\con{i}$, with~$i \in \iub \cup \ieq$.
We present in what follows some classical merit functions.

\subsection{The Courant merit function}

Perhaps the most classical merit function is the Courant merit function\footnote{\citeauthor{Courant_1943} proposed this merit function when dealing with boundary conditions of equilibrium and vibration problems. See~\cite[Pt.~II, \S~3]{Courant_1943} for details.}~\cite{Courant_1943}, defined by
\begin{equation*}
    \merit[\gamma](\iter) \eqdef \obj(\iter) + \gamma \bigg( \sum_{i \in \iub} [\con{i}(\iter)]_+^2 + \sum_{i \in \ieq} \con{i}(\iter)^2 \bigg), \quad \text{for~$x \in \R^n$ and~$\gamma \ge 0$,}
\end{equation*}
where~$[\cdot]_+$ denotes the positive-part operator.
The advantage of such a merit function is that it is differentiable if~$\obj$ and~$\con{i}$, for~$i \in \iub \cup \ieq$, are differentiable.
However, normally, a global minimizer of~$\merit[\gamma]$ is not a solution to~\cref{eq:problem-cobyqa-sqp} when~$\gamma$ is finite.
This phenomenon can be seen on the simple example of minimizing~$x$ subject to~$x \ge 0$.

\subsection{Nonsmooth merit functions}

Others examples merit functions are the~$\ell_p$-merit functions, defined by
\begin{equation*}
    \merit[\gamma](\iter) \eqdef \obj(\iter) + \gamma \bigg( \sum_{i \in \iub} [\con{i}(\iter)]_+^p + \sum_{i \in \ieq} \abs{\con{i}(\iter)}^p \bigg)^{1/p}, \quad \text{for~$x \in \R^n$ and~$\gamma \ge 0$,}
\end{equation*}
Such merit functions enjoy the properties of being exact under some mild assumptions.
Roughly speaking, this means that a solution\footnote{Here, the term \enquote{solution} may be interpreted in different ways. It may be a global solution, a local solution, or a stationary point.} to the constrained problem~\cref{eq:problem-cobyqa-sqp} can be obtained by minimizing~$\merit[\gamma]$ when~$\gamma$ is big enough.
We will not discuss this concept further, but only provide the following theorem for later reference.
Interested readers may refer to~\cite{Han_Mangasarian_1979,Pillo_Grippo_1989}\todo{Add more references}.

\begin{theorem}[{\cite[Thm.~14.5.1]{Conn_Gould_Toint_2000}}]
    \label{thm:exact-merit-function}
    Assume that the functions~$\obj$ and~$\con{i}$ are twice continuously differentiable for all~$i \in \iub \cup \ieq$.
    Let~$(\iter[\ast], \lm[\ast])$ be a \gls{kkt} pair to the problem~\cref{eq:problem-cobyqa-sqp} that satisfies the second-order sufficient condition of \cref{thm:second-order-sufficient-conditions}.
    If~$\gamma \ge \norm{\lm[\ast]}_q$, where~$q$ is the H{\"{o}}lder conjugate of~$p$, then~$\iter[\ast]$ satisfies the second-order sufficient condition for the minimization of~$\merit[\gamma]$.
    Moreover, if~$\gamma > \norm{\lm[\ast]}_q$, then the two second-order sufficient conditions are equivalent.
\end{theorem}

An inconvenience of the~$\ell_p$-merit functions is that, even if~$\obj$ and~$\con{i}$ are differentiable for all~$i \in \iub \cup \ieq$,~$\merit[\gamma]$ is likely not differentiable at the points~$\iter \in \R^n$ where~$\con{i}(\iter) = 0$ for all~$i \in \iub \cup \ieq$.
However, there do exist smooth merit functions that enjoy exactness, as shown in the next section.

\subsection{The augmented Lagrangian merit function}

For simplicity, we shall assume here that~$\iub = \emptyset$.
The augmented Lagrangian~\cite{Hestenes_1969,Powell_1969,Rockafellar_1973} of the problem~\cref{eq:problem-cobyqa-auglag} is then
\begin{equation*}
    \lag[\mathsf{A}](\iter, \lm) \eqdef \lag(\iter, \lm) + \frac{\gamma}{2} \sum_{i \in \ieq} \con{i}(\iter)^2, \quad \text{for~$\iter \ge 0$ and~$\lm = [\lm_i]_{i \in \ieq}$,}
\end{equation*}
where~$\lag$ denotes the Lagrangian function of the problem~\cref{eq:problem-cobyqa-sqp}.
The augmented Lagrangian merit function is then defined as
\begin{equation*}
    \merit[\gamma](\iter) = \lag[\mathsf{A}](\iter, \lm(\iter)),
\end{equation*}
where~$\lm(\iter)$ denotes the least-squares solution to
\begin{align*}
    \min        & \quad \norm[\bigg]{\nabla \obj(\iter) + \sum_{i \in \ieq} \lm_i \nabla \con{i}(x)}\\
    \text{s.t.} & \quad \lm = [\lm_i]_{i \in \ieq}.
\end{align*}
To some extend, the least-squares Lagrange multipliers attempts to satisfy the \gls{kkt} conditions as much as possible.
If~$\iter[\ast]$ is a solution to the problem~\cref{eq:problem-cobyqa-sqp}, then~$(\iter[\ast], \lm(\iter[\ast]))$ is a \gls{kkt} pair.
When~$\iub \neq \emptyset$, to achieve the same propery, the complamentary slackness conditions must be taken into account.

A clear drawback of such a merit function is that it is expensive to evaluate, as one evaluation necessitates to solve a linear least-squares problem that involve the gradients of~$\obj$ and~$\con{i}$, with~$i \in \ieq$.
Nonetheless, this merit function offers several advantages.
First of all, if~$\obj$ and~$\con{i}$, for~$i \in \ieq$, are differentiable, and if~$\set{\nabla \con{i}(\iter)}_{i \in \ieq}$ are linearly independent for all~$\iter \in \R^n$, then~$\merit[\gamma]$ is differentiable.
Moreover, under some mild assumptions, when~$\gamma$ is large enough, the second-order sufficient conditions of~\cref{eq:problem-cobyqa-sqp} and of minimizing~$\merit[\gamma]$ are equivalent~\cite[Thm.~14.6.1]{Conn_Gould_Toint_2000}.

\subsection{Maratos effect and second-order correction}

In practice, to have global convergence, the \gls{sqp} method needs to be modified, normally using a merit function to decide whether to accept a step or not.
However, the merit function may jeopardize the fast local convergence of the \gls{sqp} method.
This is known as the \emph{Maratos effect}~\cite{Maratos_1978}.

The reason behind the effect is that, for certain problems, the \gls{sqp} method can generate a step that increases both the objective function and the constraint violation, no matter how close is the current iterate to a solution.
Such a step would be rejected by any merit function that is increasing with respect to both the objective function and the constraint violation.
See examples of this phenomenon in~\cite[\S~3.5]{Maratos_1978} and~\cite{Powell_1987}.

\subsubsection{Second-order correction}

\begin{itemize}
    \item \citeauthor{Fletcher_1982}~\cite{Fletcher_1982}~\cite[\S~14.4]{Fletcher_1987}.
    \item \citeauthor{Mayne_Polak_1982}~\cite{Mayne_Polak_1982}.
\end{itemize}

\subsubsection{A note on the augmented Lagrangian merit function}

\begin{itemize}
    \item Find a reference why it is not subject to the Maratos effect.
\end{itemize}

\section{The trust-region \glsfmtshort{sqp} method}

\subsection{Overview of the method}

\subsection{Composite-step approaches}

\subsubsection{Vardi-like approach}

\subsubsection{Byrd-Omojokun-like approach}

\begin{itemize}
    \item Compare with the Vardi-like approach.
\end{itemize}

\subsubsection{\Glsfmtlong{cdt}-like approach}
