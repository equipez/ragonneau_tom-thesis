%%
%% This is mathematics.tex.
%% The main TeX-file of the Ph.D. thesis of Tom M. Ragonneau.
%% It follows the requirements of The Hong Kong Polytechnic University.
%%
%% It may be distributed and/or modified under the conditions of the
%% LaTeX Project Public License, either version 1.3 of this license or
%% (at your option) any later version. The latest version of this
%% license is in
%%     http://www.latex-project.org/lppl.txt
%% and version 1.3 or later is part of all distributions of LaTeX
%% version 2003/12/01 or later.
%%
%% This work has the LPPL maintenance status 'maintained'.
%%
%% The Current Maintainer of this work is Tom M. Ragonneau.
%%
\usepackage{amsmath}
\usepackage{amsthm}
\usepackage{amsfonts}
\usepackage{amssymb}
\usepackage{mathrsfs}
\usepackage{mathtools}
\usepackage{dsfont}
\usepackage{empheq}
\usepackage{bm}
\usepackage{xargs}

% General aspect of the mathematical expressions and environments
\numberwithin{equation}{section}
\renewcommand{\qedsymbol}{\ensuremath{\blacksquare}}

% Mathematical constants, sets, and notations
\def\eu{\ensuremath{\mathrm{e}}}
\def\iu{\ensuremath{\mathrm{i}}}
\def\du{\ensuremath{\mathrm{d}}}
\def\C{\ensuremath{\mathds{C}}}
\def\F{\ensuremath{\mathds{F}}}
\def\N{\ensuremath{\mathds{N}}}
\def\Q{\ensuremath{\mathds{Q}}}
\def\R{\ensuremath{\mathds{R}}}
\def\Z{\ensuremath{\mathds{Z}}}
\def\T{\ensuremath{\mathsf{T}}}

% Mathematical operators
\DeclareMathOperator*{\argmax}{arg\,max}
\DeclareMathOperator*{\argmin}{arg\,min}
\def\eqdef{\ensuremath{\mathbin{\stackrel{\mathsf{def}}{=}}}}

% Mathematical macros
\newcommand{\abs}[2][]{#1\lvert#2#1\rvert}
\newcommand{\ceil}[2][]{#1\lceil#2#1\rceil}
\newcommand{\floor}[2][]{#1\lfloor#2#1\rfloor}
\newcommand{\norm}[2][]{#1\lVert#2#1\rVert}
\newcommand{\set}[2][]{#1\{#2#1\}}
\newcommand{\inner}[2][]{#1\langle#2#1\rangle}

% Mathematical environments (theorems, proofs, ...)
\theoremstyle{definition}
\newtheorem{assumption}{Assumption}[section]
\newtheorem{condition}{Condition}[section]
\newtheorem{definition}{Definition}[section]
\newtheorem{example}{Example}[section]
\newtheorem{problem}{Problem}[section]
\theoremstyle{plain}
\newtheorem{conjecture}{Conjecture}[section]
\newtheorem{proposition}{Proposition}[section]
\newtheorem{theorem}{Theorem}[section]
\newtheorem{corollary}{Corollary}[theorem]
\newtheorem{lemma}[theorem]{Lemma}
\theoremstyle{remark}
\newtheorem*{annotation}{Annotation}
\newtheorem*{claim}{Claim}
\newtheorem*{note}{Note}
\newtheorem*{remark}{Remark}

% Fine-tuning spacing in formulas
% Make @ behave as per catcode 13 (active). The TeXbook, p. 155.
\mathcode`@="8000{\catcode`\@=\active\gdef@{\mkern1mu}}

% Dedicated mathematical macros
