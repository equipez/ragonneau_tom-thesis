%% utils/profiles.tex
%% Copyright 2021-2022 Tom M. Ragonneau
%
% This work may be distributed and/or modified under the
% conditions of the LaTeX Project Public License, either version 1.3
% of this license or (at your option) any later version.
% The latest version of this license is in
%   http://www.latex-project.org/lppl.txt
% and version 1.3 or later is part of all distributions of LaTeX
% version 2005/12/01 or later.
%
% This work has the LPPL maintenance status `maintained'.
%
% The Current Maintainer of this work is Tom M. Ragonneau.
\begin{tikzpicture}
    \pgfplotstableread[col sep=comma]{\selectcsv}\selectcsvread
    \newcommand{\getxmaxcsv}[1]{%
        \pgfplotstablegetrowsof{\selectcsvread}%
        \pgfmathtruncatemacro{\LastRowNo}{\pgfplotsretval-1}%
        \pgfplotstablesort[sort key={#1}]{\csvsorted}{\selectcsvread}%
        \pgfplotstablegetelem{\LastRowNo}{#1}\of{\csvsorted}%
        \pgfmathsetmacro{\selectxmaxcsv}{\pgfplotsretval}%
    }
    \pgfmathparse{\selectsolvers[0]}
    \let\selectsolver\pgfmathresult\relax
    \getxmaxcsv{x\selectprofile_\selectsolver}
    \begin{axis}[%
        width=205pt,%
        xmin=0,%
        xmax=0.55*\selectxmaxcsv,%
        ymin=0,%
        ymax=1,%
        minor y tick num=1,%
        yminorticks=true,%
        ytick={0,0.2,...,1},%
        cycle list name=profiles,%
        legend pos=south east,%
        xlabel={\selectxlabel},%
        ylabel={\selectylabel},%
        xticklabel style={/pgf/number format/1000 sep={}},%
    ]
        \pgfmathparse{dim{\selectsolvers}-1}
        \let\selectlastindex\pgfmathresult\relax
        \foreach \i in {0,1,...,\selectlastindex} {%
            \pgfmathparse{\selectsolvers[\i]}%
            \let\selectsolver\pgfmathresult\relax%
            \StrSubstitute{\selectsolver}{-}{~}[\selectsolverescaped]%
            \addplot table[%
            x=x\selectprofile_\selectsolver,%
            y=y\selectprofile,%
            col sep=comma,%
            ]{\selectcsvread};%
            \addlegendentryexpanded{\selectsolverescaped}%
        }
    \end{axis}
\end{tikzpicture}
